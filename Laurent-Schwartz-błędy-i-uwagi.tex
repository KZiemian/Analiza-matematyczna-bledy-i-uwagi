% Autor: Kamil Ziemian

% ---------------------------------------------------------------------
% Podstawowe ustawienia i pakiety
% ---------------------------------------------------------------------
\RequirePackage[l2tabu, orthodox]{nag}  % Wykrywa przestarzałe i niewłaściwe
% sposoby używania LaTeXa. Więcej jest w l2tabu English version.
\documentclass[a4paper,11pt]{article}
% {rozmiar papieru, rozmiar fontu}[klasa dokumentu]
\usepackage[MeX]{polski}  % Polonizacja LaTeXa, bez niej będzie pracował
% w języku angielskim.
\usepackage[utf8]{inputenc}  % Włączenie kodowania UTF-8, co daje dostęp
% do polskich znaków.
\usepackage{lmodern}  % Wprowadza fonty Latin Modern.
\usepackage[T1]{fontenc}  % Potrzebne do używania fontów Latin Modern.



% ---------------------------------------
% Podstawowe pakiety (niezwiązane z ustawieniami języka)
% ---------------------------------------
\usepackage{microtype}  % Twierdzi, że poprawi rozmiar odstępów w tekście.
% \usepackage{graphicx}  % Wprowadza bardzo potrzebne komendy do wstawiania
% % grafiki.
\usepackage{xcolor}
% \usepackage{verbatim}  % Poprawia otoczenie VERBATIME.
% \usepackage{textcomp}  % Dodaje takie symbole jak stopnie Celsiusa,
% % wprowadzane bezpośrednio w tekście.
\usepackage{vmargin}  % Pozwala na prostą kontrolę rozmiaru marginesów,
% za pomocą komend poniżej. Rozmiar odstępów jest mierzony w calach.
% ---------------------------------------
% MARGINS
% ---------------------------------------
\setmarginsrb
{ 0.7in} % left margin
{ 0.6in} % top margin
{ 0.7in} % right margin
{ 0.8in} % bottom margin
{  20pt} % head height
{0.25in} % head sep
{   9pt} % foot height
{ 0.3in} % foot sep



% ---------------------------------------
% Często przydatne pakiety
% ---------------------------------------
% \usepackage{csquotes}  % Pozwala w prosty sposób wstawiać cytaty do tekstu.
% \usepackage{xcolor}  % Pozwala używać kolorowych czcionek (zapewne dużo
% więcej, ale ja nie potrafię nic o tym powiedzieć).



% ---------------------------------------
% Pakiety do tekstów z nauk przyrodniczych
% ---------------------------------------
\let\lll\undefined  % Amsmath gryzie się z pakietami do języka
% polskiego, bo oba definiują komendę \lll. Aby rozwiązać ten problem
% oddefiniowuje tę komendę, ale może tym samym pozbywam się dużego Ł.
\usepackage[intlimits]{amsmath}  % Podstawowe wsparcie od American
% Mathematical Society (w skrócie AMS)
\usepackage{amsfonts, amssymb, amscd, amsthm} % Dalsze wsparcie od AMS
% \usepackage{siunitx}  % Do prostszego pisania jednostek fizycznych
\usepackage{upgreek}  % Ładniejsze greckie litery
% Przykładowa składnia: pi = \uppi
\usepackage{slashed}  % Pozwala w prosty sposób pisać slash Feynmana.
\usepackage{calrsfs}  % Zmienia czcionkę kaligraficzną w \mathcal
% na ładniejszą. Może w innych miejscach robi to samo, ale o tym nic
% nie wiem.



% ##########
% Tworzenie otoczeń "Twierdzenie", "Definicja", "Lemat", etc.
\newtheorem{twr}{Twierdzenie}  % Komenda wprowadzająca otoczenie
% ,,twr'' do pisania twierdzeń matematycznych
\newtheorem{defin}{Definicja}  % Analogicznie jak powyżej
\newtheorem{wni}{Wniosek}



% ---------------------------------------
% Pakiety napisane przez użytkownika.
% Mają być w tym samym katalogu to ten plik .tex
% ---------------------------------------
\usepackage{SchwartzBooksCommands}  % Pakiet napisany m.in. dla tego pliku.
\usepackage{latexshortcuts}
% \usepackage{mathshortcuts}
\usepackage{mathcommands}
\usepackage{calculuscommands}



% ---------------------------------------------------------------------
% Dodatkowe ustawienia dla języka polskiego
% ---------------------------------------------------------------------
\renewcommand{\thesection}{\arabic{section}.}
% Kropki po numerach rozdziału (polski zwyczaj topograficzny)
\renewcommand{\thesubsection}{\thesection\arabic{subsection}}
% Brak kropki po numerach podrozdziału



% ---------------------------------------
% Ustawienia różnych parametrów tekstu
% ---------------------------------------
\renewcommand{\arraystretch}{1.2}  % Ustawienie szerokości odstępów między
% wierszami w tabelach.



% ---------------------------------------
% Pakiet „hyperref”
% Polecano by umieszczać go na końcu preambuły.
% ---------------------------------------
\usepackage{hyperref}  % Pozwala tworzyć hiperlinki i zamienia odwołania
% do bibliografii na hiperlinki.










% ---------------------------------------------------------------------
% Tytuł, autor, data
\title{Dzieła Laurenta Schwartza~-- błędy i~uwagi}

% \author{}
% \date{}
% ---------------------------------------------------------------------










% ####################################################################
\begin{document}
% ####################################################################



% ######################################
\maketitle % Tytuł całego tekstu
% ######################################


% ,,P\ldots'' oznacza, że w wydaniu ,,\ldots'' błąd został poprawiony. \\



% ############################
\Work{ % Autor i tytuł dzieła
  Laurent Schwartz \\
  „Kurs analizy matematycznej. Tom~I”,
  \cite{SchwartzKursAnalizyMatematycznejVolI1979} }


% ##################
\CenterBoldFont{Uwagi ogólne}

To zadziwiające, że~w~tak ogromnym traktacie nie~udowodniono
twierdzenia o trzech ciągach, które ma tak fundamentalne znaczenie
w~analizie matematycznej.

\vspace{\spaceThree}



Udowodnić, że domknięcie podprzestrzeni, też jest podprzestrzenią. Czy
nie można tego po prostu skwitować, że wynika to od razu, z tego że
taka podprzestrzeń nie ma punktów izolowanych?

\vspace{\spaceThree}



Duża ilość nieomówionych pojęć pojawiających się w rozdziale 16
sugeruje, że~książce został pominięty rozdział poświęcony omówieniu
własności przestrzeni Banacha. Albo, że~taki rozdział powinien się
w~niej znaleźć.

\vspace{\spaceThree}



Rozdział II.16. jakkolwiek prezentuje teorię przestrzeni Hilberta
pięknie i głęboko, jest w~pewnym stopniu chaotyczny i~przydałaby mu
się dodatkowa edycja.

\vspace{\spaceThree}



Nie zdefiniowano pojęcia izometrii. To pojęcie jest tak oczywiste, że
nie będziemy go tu przytaczać.

\vspace{\spaceThree}



\noindent
\textbf{Rozdział IV.}

\vspace{\spaceThree}

\start W~podanej w tym rozdziale teorii całki Lebesgue’a wychodzącej
od całki Radona należy rozróżnić dwa jej aspekty. Po pierwsze
$\mu^{ * }$ i $\mu_{ * }$ są zapewne miarami zewnętrznymi (co trzeba
wykazać) których obcięcia do zbiorów mierzalnych~są miarami (co jest
już jasne z teorii przedstawionej w~książce). Ponadto wiele dowodów
nie wykorzystuje w ogóle funkcjonału $\mu( \varphi )$ lecz jedynie
fakt, że~dana miara, zgodnie z terminologią
z~\cite{RudinAnalizaRzeczywistaIZespolona1998}, jest regularną miarą
borelowską. Można więc tą część teorii uogólnić dla tego przypadku,
jeśli tylko uda się znaleźć sposób na obejście ograniczenia, że
przestrzeń $X$ jest przeliczalna w~nieskończoności. Z~punktu widzenia
praktyki nie jest to bardzo pilne, bowiem przypadek przestrzeni
przeliczalnej w~nieskończoności obejmuje większość ważnych przypadków.

Z~drugiej strony, pojęcie takie jak nośnik miary, choć można go
zdefiniować dla dowolnej całki na $X$ (wystarczy rozpatrzyć nośnik
funkcjonału $\int \varphi \, \dPL \mu$) to bardziej naturalne jest
spojrzenie na niego w świetle teorii dystrybucji. Rzeczywiście, jeżeli
$X$ jest rozmaitością, to każda miara Radona definiuje sposób
dystrybucję, bowiem jeśli ciąg funkcji $\varphi_{ n }$ zmierza do 0
w~sensie funkcji próbnych tzn. zachowując nośnik w ustalonym zbiorze
zwartym, jednostajnie wraz ze~wszystkimi swymi pochodnymi to wprost z~
definicji całki Radona $\mu( \varphi_{ n } ) \to 0$. Twierdzenie
odwrotne może nie zachodzić, bowiem znając wartość dystrybucji na
funkcjach gładkich o nośniku zwartym, należałoby ją przedłużyć do
wszystkich funkcji ciągłych o~nośniku zwartym. Może to być wykonalne,
zależy to w~pierwszym rzędzie od~tego, czy~funkcje gładkie są gęste w
zbiorze funkcji ciągłych, ale na te wszystkie pytania nie jestem
w~stanie odpowiedzieć.

\vspace{\spaceFour}



\start \textbf{2.} Podrozdział~6. „Obraz miary w~odwzorowaniu”
korzysta jawnie z~całki Radona i~funkcji ciągłych o~nośniku zwartym
więc jego uogólnienie na regularne miary borelowskie może nie być
możliwe (choć nie wydaje się to prawdopodobne). Ponieważ jednak
formuły tutaj są znacznie ogólniejsze niż w~większości książek się
rozważa przy twierdzeniu o zmianie zmiennych w~całce z~miarą
Lebesgue’a na $\Rbb^{ n }$, a~do tego ostatecznie dążymy.

\vspace{\spaceFour}



\start Rozdział IV. Uwaga 3. W~tym rozdziale często jest żądane by
dany zbiór $A$ był zwarty lub borelowski o~domknięciu zwartym. Na
podstawie dzieła \cite{BogachevMeasureTheory2007}?? należy dojść do
wniosku, że~wielu wypadkach, jeśli nie~we wszystkich, można zamiast
tego przyjąć, że~$A$ jest mierzalny o~mierze skończonej (co zapewne
motywuje ten wybór u Schwartza) i~teoria nie ulegnie zmianie.
W~szczególności można to zrobić w~definicji całek na stronie 454.

\vspace{\spaceFour}



\start \Str{16} Dyskusja zmiany zmiennej i funkcji jest tak piękna, że
warto jeszcze dodać następującą uwagę aby uczynić ją odrobinę bardziej
pełną. Rozważmy dane funkcje z $G_{ 1 }$, $G_{ 2 }$ z $E_{ 1 }$ w $F$
i niech $g$ będzie bijekcją z $E_{ 1 }$ na $E_{ 2 }$. Wtedy równanie
$G_{ 1 }( f( x ), x ) = G_{ 2 }( f( x ), x )$ ma rozwiązanie
$f : E_{ 1 } \to F$ wtedy i tylko wtedy gdy istnieje rozwiązanie
$f_{ 1 } : E_{ 2 } \to F$ równania
$G_{ 1 }( f_{ 1 }( y ), g^{ -1 }( y ) ) = G_{ 2 }( f_{ 1 }( y ), g^{
  -1 }( y ) )$. Rozwiązania te są powiązane związkiem
$f_{ 1 }( y ) = f \circ g^{ -1 }( y )$. Analogiczne rozważanie można
przeprowadzić dla zmiany funkcji.

\vspace{\spaceFour}


\start \Str{17} Jeżeli $f : E \to F$ nie jest suriekcją to rodzina
zbiorów postaci $f^{ -1 } ( \{ z \} )$ dalej stanowi rozwarstwienie
przestrzeni $E$, z tą różnicą, że zawiera teraz zbiór $\emptyset$.
Ponadto dla suriekcji, jeżeli $x \neq y$ to
$f^{ -1 }( \{ x \} ) \neq f^{ -1 } ( \{ y \} )$, teraz oczywiście może
zajść $f^{ -1 }( \{ x \} ) = f^{ -1 }( \{ y \} ) = \emptyset$. W
szczególności $f^{ -1 }$ nie musi być odwzorowaniem na zbiór klas
abstrakcji. W tej sytuacji nie można uważać $F$ za „model” dla
$E / R$. Można to jednak naprawić zawężając zbiór końcowy do $f( E )$.

\vspace{\spaceFour}



\start \Str{46} Wywodząc bezpośrednio z punktu a) punkt c), Schwartz
popełnia, chyba, pewną nonszalancje. Wydaje się, że rozumował on tak:
jeżeli $F$ jest otoczeniem w $E$, zaś $\Vbb$ otoczeniem w $F$, to
otwarty względem $F$ zbiór zawarty w $A$, jest też otwarty w $E$. Jest
to absolutnie nieprawda. Świadczy o tym następujący kontrprzykład:
rozpatrzmy w $\Rbb$ odcinek $X = [ -2, 2 ]$ jest to otoczenie 0. Teraz
zbiór $( -1, 2 ]$, jest zbiorem otwartym
w $X$, jako ślad zbioru $( -1, 3 )$, nie jest jednak otwarty w~$\Rbb$. \\
Dowód punktu c): \\
Jeżeli $F$ jest otoczeniem $a$ w $E$, to zwiera zbiór otwarty $A$ w
$E$, zaś $\Vbb$ jest otoczeniem $a$ w $F$, to zawiera zbiór $B$,
zawierający $a$, będący śladem zbioru otwartego $B_{ 1 }$ z~$E$. Teraz
$A \cap B_{ 1 } \subset \Vbb$, jest otwarty zarówno w $E$ jak i~w~$F$.

\vspace{\spaceFour}



\start \Str{30} Funkcja $f( x ) = \frac{ \log( x ) }{ 1 - x }$ nie
jest bijekcją odcinka $] 0, 1 [$ na $\Rbb$, bowiem w całym tym
przedziale $f( x ) < -1$.

\vspace{\spaceFour}



\start \Str{51} Pierwsze sformułowanie Twierdzenia 2 z rozdziału II,
obowiązuje tylko w przestrzeniach metrycznych, jednak pozostałem dwa
obowiązują w dowolnej przestrzeni metrycznej.

\vspace{\spaceFour}



\start \Str{53} Ostatnie zdanie przypis 5 jest napisany zbyt
lekkomyślnie, sugeruje bowiem, że dla zbieżności danej funkcji nie ma
znaczenia czy $a \in A$, czy też $a \notin A$. W rzeczywistości jest
wręcz przeciwnie.

\vspace{\spaceFour}



\start \Str{60} W dowodzie Twierdzenia 13, można pominąć zdanie po
„z~dostatecznie dużymi wskaźnikami;”. Poprawia to ponadto spójność
rozumowania.

\vspace{\spaceFour}



\start \Str{60} Poprawić dowód Twierdzenia 20. \red{???!!!}

\vspace{\spaceFour}



\start \Str{65} Dowód pierwszej części twierdzenia 27 jest bez sensu.

\vspace{\spaceFour}



\start \Str{96} Należałoby udowodnić równoważność podanych tu określeń
normy.

\vspace{\spaceFour}



\start \Str{106} Uzupełnić dowód.

\vspace{\spaceFour}



\start \Str{108} Wspomniano tu analogiczne wyniki dla ciągów w
przestrzeni skończenie wymiarowej, które nie zostały chyba jednak
nigdzie podane. Jakkolwiek centralny wynik jest oczywisty, warto
go tu przytoczyć. \\
Twierdzenie. \\
Aby ciąg elementów skończenie wymiarowej przestrzeni unormowanej
$v_{ n }$, był zbieżny, potrzeba i wystarczy, by w pewnej (a tym samy
dowolnej) bazie był zbieżny każdy ciąg jego
współrzędnych. \\
Dowód. \\
Załóżmy, że w pewnej bazie $e_{ 1 }, \ldots, e_{ n }$ każdy ciąg
współrzędnych jest zbieżny. Mamy wtedy:
\begin{equation}
  \label{eq:SchwartzKAMVolI-01}
  \lim_{ n \to \infty } v_{ n }
  =
  \lim_{ n \to \infty } \sum_{ i = 1 }^{ n } x_{ n }^{ i } e_{ i }
  =
  \sum_{ i = 1 }^{ n } \lim_{ n \to \infty }( x_{ n }^{ i } e_{ i } )
  =
  \sum_{ i = 1 }^{ n } \lim_{ n \to \infty }( x_{ n }^{ i } ) e_{ i }
\end{equation}
na mocy ciągłości działań w przestrzeni liniowo topologicznej.
Własność ta obowiązuje teraz w dowolnej bazie, ze względu na liniowy
związek pomiędzy współrzędnymi wektora w różnych bazach. Przeciwnie,
wynik jest natychmiastowy na podstawie oszacowania:
\begin{equation}
  \label{eq:SchwartzKAMVolI-02}
  \Vert x_{ 0 }^{ i } - x_{ n }^{ i } \Vert \leq \Vert x_{ 0 } - x_{ n } \Vert
\end{equation}

\vspace{\spaceFour}



\start \Str{141} Warto byłoby, tak jak w rozdziale III, termin
„iloczyn skalarny” zarezerwować dla form dodatnio określonych.
Wymagałoby to przeformułowania Wniosku, co jest niewielkim problemem
bowiem, jego obecna forma jest wysoce niesatysfakcjonująca. Przede
wszystkim twierdz on, że iloczyn skalarny jest ciągły, to stwierdzenie
wymaga jednak by $\vec{ E }$ była przestrzenią topologiczną. Jeżeli
jednak iloczyn skalarny generuje półnormę, należy się spytać o jaką
topologię chodzi.

\vspace{\spaceFour}



\start \Str{141} W dowodzie twierdzenia 84. jest użyte twierdzenie o
uzupełnianiu przestrzeni metrycznych (w szczególnym przypadku
unormowanych), jednak to twierdzenie nie jest nigdzie podane ani
udowodnione! Nie udowadnianie tego twierdzenia i powoływanie się na
nie, jest chyba jakąś tradycją książek do matematyki. Zastanowić się
czy można to twierdzenie udowodnić korzystając z kanonicznego
izomorfizmu ze strony 106.

\vspace{\spaceFour}



\start \Str{144} Do pełnej poprawności dowodu Twierdzenia 89 potrzeba
jeszcze udowodnić, że domknięcie zbioru wypukłego jest wypukłe.

\vspace{\spaceFour}



\start Str{145} Poprawić ten punkt.

\vspace{\spaceFour}



\start \Str{146} Twierdzenie 91 i Uwaga do niego wprowadzają pewien
logiczny zamęt. Dopiero w Uwadze jest zauważone, ze to odwzorowanie
jest liniowe, co należałoby zauważyć w miejscu gdzie stwierdza się, że
są to dopełnienia algebraiczne, a bez tego pojęcie normy odwzorowanie
jest pozbawione sensu, nie wprowadza bowiem żadnego pojęcia zbieżności
które dałoby się obronić. Dla zilustrowania tego rozważmy dwie funkcje
rzeczywiste które w świetle tej definicji mają normę równą 1: niech
pierwsza będzie równa $x$, dla $x \geq 0$ i 0 na ujemnej półosi, drugą
niech będzie $| x |$. Twierdzić, że jedna jest zbieżna do drugiej jest
czystym nonsensem pozbawionym matematycznej treści. Zauważmy ponadto,
że żadna z nich nie jest liniowa, więc z faktu, że
$\Vert u( x ) \Vert \leq k \Vert x \Vert$, nie można wnioskować o
liniowości funkcji. Trzeba szerzej przedyskutować problem dopełnień
topologicznych.

\vspace{\spaceFour}


\start \Str{148} Przy dyskusji ilorazów przestrzeni Hilberta, jest
wspomniana analogiczna procedura dla ilorazu przestrzeni Banach,
pomimo, że nie została w książce omówiona. \\
Pobieżne przedstawienie tej konstrukcji.???!!!

\vspace{\spaceFour}



\start \Str{149} Aby dowód tego twierdzenia był kompletny należy
skorzystać z poniższego lematu. \\
Lemat. \\
Jeżeli $\alpha$ jest niezerową formą nad przestrzenią Hilberta, a
$\vec{ H }$
jest jej jądrem, to $\vec{ H }^{ \bot }$ jest jednowymiarowa. \\
Dowód. \\
Jeżeli $\vec{ H }^{ \bot }$ nie jest jednowymiarowa to możemy znaleźć
w niej dwa linowa niezależne wektory $u$, $v$. Rozważmy teraz
niezerowy wektor $x = u -\frac{ \alpha( u ) }{ \alpha( v ) }v$. Jest
oczywiście $\alpha( x ) = 0$,
co jednak przeczy temu, że $\vec{ H } \cap \vec{ H }^{ \bot } =  0$. \\
Uwaga. W przestrzeni skończenie wymiarowej, wynika to od razu z
pięknego wzoru $\dim\image f = \dim X - \dim\ker f$, pozwalającego
obliczyć wymiar jądra. Ponieważ jednak przestrzeń Hilberta może być
nieskończenie wymiarowa, wzoru ten oczywiście nie zachodzi.

\vspace{\spaceFour}



\start \Str{150} Nie zdefiniowano użytych tu pojęć wzajemnych
odwzorowań, refleksywności przestrzeni i słabej zwartości. Do tego
pojęcie kanonicznej iniekcji podane jest zbyt zwięźle.

\vspace{\spaceFour}



\start \Str{150} Poprawić uwagę na dole strony.

\vspace{\spaceFour}



\start \Str{160} W dowodzie tego arcyważnego twierdzenia jest pewna
skrótowość. Należy bowiem skorzystać z ważnego związku:
\begin{equation}
  \label{eq:SchwartzKAMVolI-03}
  \Vert x \Vert = \sup_{ \Vert y^{ * } \Vert \leq 1 } | x( y^{ * } ) |
  =
  \sup_{ \Vert y^{ * } \Vert \leq 1 } | y^{ * }( x ) |.
\end{equation}
Jest to w istocie wniosek z refleksywności przestrzeni Hilberta,
bowiem na jego mocy $x$ można traktować jako odwzorowanie liniowe
ciągłe z przestrzeni dualnej w~$K$. Korzystając dodatkowo
z~izomorfizmu przestrzeni Hilberta z~jej dualną, mamy zależność:
\begin{equation}
  \label{eq:SchwartzKAMVolI-04}
  \Vert x \Vert = \sup_{ \Vert y \Vert \leq 1 } | ( y | x ) |.
\end{equation}
Zauważenie tych związków nie jest banalne, ich udowodnienie, gdy już
wiadomo z jakich twierdzeń skorzystać, już jest.

\vspace{\spaceFour}



\start \Str{169} Zastanowić się czy ten dowód wymaga uzupełnienia.
Chodzi o warunek by domknięcie zbioru było zwarte.

\vspace{\spaceFour}



\start \Str{221}. Twierdzenia 16 i 17, mądrzej byłoby umieścić na
stronie 217, na końcu podparagrafu „Pochodna zupełna, czyli
odwzorowanie pochodne”.

\vspace{\spaceFour}



\start \Str{229} W tym miejscu powinien być komentarz odnośnie tego
jak podstępnym obiektem są pochodne cząstkowe w wybranych
współrzędnych, choć problem ten się nie pojawia przy mechanicznym
wykonywaniu rachunków.
Omówienie problemu: \\
Na początku musimy się spytać co oznacz symbol $\partial_{ x^{ i } }$?
Zaczniemy od przypadku ogólnego: jeżeli mamy funkcję
$f : E \to \vec{ F }$, gdzie $E$ jest zbiorem na którym można określić
układ współrzędnych (rozmaitością), ten symbol oznacza „oblicz
pochodną w kierunku w którym zmienia się tylko $x$”. Linie na których
zmienia się tylko jedna współrzędna nazywamy liniami
izoparametrycznymi. Jest więc jasne, że ponieważ różne układy
współrzędnych mają różne linie izoparametryczne eg. dla układu
kartezjańskiego na płaszczyźnie są to proste, dla współrzędnych
radialnych, są to półproste wychodzące z początku układu współrzędnych
(dla współrzędnej kątowej) i okręgi bez punktu (dla współrzędnej
radialnej), pochodne w tych współrzędnych są różne. Oczywiście jeśli
jakaś transformacja pozostawia linie izoparametryczne jednej
współrzędnej niezmienione (przynajmniej w pewnym obszarze) o pochodna
względem tej zmiennej się niezmienna. Lecz nawet tu sytuacja jest
zwodnicza, co zademonstrujemy na wyjątkowo uderzającym
transformacji liniowych.

\vspace{\spaceFour}



\start \Str{233} Dyskusja o zmianie zmienny jest tak piękna, że warto
ją jeszcze rozszerzyć. Aby uczynić ją pełną, wydaje mi się, że warto
zauważyć następujący fakt: jeśli $f : E \to F$ jest bijekcją to
równanie $g( x ) = h( x )$, dla funkcji określonych na $F$ (równianie
to wygląda trywialnie, zauważmy jednak, że możemy przyjąć
$g( x ) = \frac{ d h }{ d x }$), jest spełnione wtedy i tylko wtedy
gdy spełnione jest równanie $g \circ f^{ -1 } = h \circ f^{ -1 }$.
Niech $E$ i $F$ będą teraz unormowanymi przestrzeniami wektorowymi.

\vspace{\spaceFour}



\start \Str{243} Poprawić dowód uwagi 2.

\vspace{\spaceFour}



\start \Str{249} Zrób coś z wzorem (3.6.7).

\vspace{\spaceFour}



\start \Str{387} Przedyskutuj to oszacowanie. Jest poprawne, nie jest
tylko wspomniane, że jest przyjęte tak by łatwo było liczyć, nie by
uzyskać optymalne oszacowanie, co w praktycznych rachunkach jest i tak
bez znaczenia. To, że takich przedziałów jest nie więcej niż $2N$
wynika chyba z tego, iż tych punktów jest $N + 1$, więc mogą się one
zawierać w co najwyżej $N + 1$ przedziałach. Zachodzi przy tym
$N + 1 \leq 2 N, N \geq 1$.

\vspace{\spaceFour}



\start \Str{399} Uwagi 1 i 2 należy przepisać, w obecnej formie są
zbyt chaotyczne.

\vspace{\spaceFour}



\start \Str{401} Zdanie w linii siódmej „ponieważ funkcja wykładnicza
rośnie szybciej niż dowolny wielomian”, jest sformułowane niezbyt
szczęśliwie. Spróbuj je poprawić.

\vspace{\spaceFour}



\start \Str{403} Dowód faktu, że suma przedstawiająca funkcje $\beta$
jest lokalnie skończona jest chyba następujący. Dla dowolnego kompaktu
$K$ z własności zbiorów $U_{ n }$ jednie dla skończonej ilość
wskaźników $n$ funkcje $\beta_{ i n }$ są niezerowe, z drugiej strony
dla każdego ustalonego $n$ jednie skończona ilość funkcji jest
niezerowa. Na tym kompakcie nie znika więc tylko skończona ilość
funkcji.

\vspace{\spaceFour}



\start \Str{405} Uzupełnienie dowodu Wniosku 3. Z tego że, nośnik
$\alpha_{ i }$ znika na pewnym otoczeniu sumy $F_{ i }'$, wynika w
szczególności, iż znika ona na pewnym otoczeniu $F_{ j }, j \neq i$.
Przecinając te wszystkie otoczenia dla $j \neq i$ otrzymujemy
otoczenie $F_{ j }$ na którym jest niezerowa tylko funkcja
$\alpha_{ j }$.

\vspace{\spaceFour}



\start \Str{406} Nie jestem wcale pewien czy z Wniosku 4 wynikają
wszystkie pozostałe wnioski.

\vspace{\spaceFour}



\start \Str{419} Udowodnienie nierówności (4.2.61) jest raczej zbyt
ascetyczne, więc tutaj podamy bardziej rozbudowaną formę. Z
nierówności (4.2.63) wynika, że:
\begin{equation}
  \label{eq:SchwartzKAMVolI-04}
  \big| \mu( \varphi ) - \sum_{ i = 0 }^{ N } g_{ i } \mu( \psi_{ i } ) \big|
  \leq
  \Vert \mu \Vert_{ \Vcal } \frac{ \varepsilon }{ 2 \Vert \mu \Vert_{ \Vcal } } = \frac{ \varepsilon }{ 2 }.
\end{equation}
Z drugiej strony:
\begin{equation}
  \label{eq:SchwartzKAMVolI-05}
  | \mu( \varphi ) | - \sum_{ i = 0 }^{ N } | g_{ i } | \mu( \psi_{ i } )
  \leq
  | \mu( \varphi ) | - | \sum_{ i = 0 }^{ N } g_{ i } \mu( \psi_{ i } ) |
  \leq
  \bigg| \mu( \varphi ) - | \sum_{ i = 0 }^{ N }
  g_{ i } \mu( \psi_{ i } ) | \bigg| \leq \bigg| \mu( \varphi )
  - \sum_{ i = 0 }^{ N } g_{ i } \mu( \psi_{ i } ) \bigg|.
\end{equation}
Z powyższych otrzymujemy w prosty sposób:
\begin{equation}
  \label{eq:SchwartzKAMVolI-06}
  | \mu( \varphi ) |
  \leq
  \sum_{ i = 0 }^{ N } | g_{ i } | \mu( \psi_{ i } ) + \frac{ \varepsilon }{ 2 }.
\end{equation}
Z nierówności (4.2.64) wynika teraz (zauważmy, że tu wszystkie funkcje
są rzeczywiste):
\begin{equation}
  \label{eq:SchwartzKAMVolI-07}
  \sum_{ i = 0 }^{ N } | g_{ i } | \mu( \psi_{ i } ) - \mu( | \varphi | )
  \leq
  \bigg| \sum_{ i = 0 }^{ N } | g_{ i } | \mu( \psi_{ i } ) - \mu( | \varphi | ) \bigg|
  =
  \bigg| \mu( | \varphi | ) - \sum_{ i = 0 }^{ N } | g_{ i } | \mu( \psi_{ i } ) \bigg|
  \leq
  \Vert \mu \Vert_{ \Vcal } \frac{ \varepsilon }{ 2 \Vert \mu \Vert_{ \Vcal } } = \frac{ \varepsilon }{ 2 },
\end{equation}
stąd
\begin{equation}
  \label{eq:SchwartzKAMVolI-08}
  \sum_{ i = 0 }^{ N } | g_{ i } | \mu( \psi_{ i } )
  \leq
  \mu( | \varphi | ) + \frac{ \varepsilon }{ 2 }.
\end{equation}
To pozwala już w prosty sposób otrzymać żądaną nierówność.

\vspace{\spaceFour}



\start \Str{419} Tu na pewno nie chodzi o wzory (4.2.35).

\vspace{\spaceFour}



\start \Str{421} Uzupełnienie dowodu, że
$\inf( e^{ + }, e^{ - } ) = 0$. Analogicznie jak dla supremum
zachodzi:
\begin{equation}
  \label{eq:SchwartzKAMVolI-09}
  \inf( e - h, f - h ) = \inf( e, f ) - h.
\end{equation}
Stąd mamy
\begin{equation}
  \label{eq:SchwartzKAMVolI-10}
  \inf( e^{ + }, e^{ - } )
  =
  e^{ + } + \inf( e^{ + } - e^{ + }, e^{ - } - e^{ + } ).
\end{equation}
Łącząc to z podany w książce wywodem dowodzimy już bezpośrednio
rozważanej własności.

\vspace{\spaceFour}



\start \Str{422} Czy siatka zupełna według podanej tu definicji jest
też siatką? Wydaje mi się, że~nie bowiem wtedy dla każdej pary
$( e, 0 )$ musiałby istnieć element ją majoryzujący, a~w~definicji
siatki zupełnej nic się o~tym nie zakłada. Wydaje się więc, że trzeba
dodatkowo założyć, iż~siatka zupełna jest siatką.

\vspace{\spaceFour}



\start \Str{422} Nie rozumiem twierdzenia 19 w tej części która odnosi
się do wzoru (4.2.71). Czy nie należy w nim zastąpić $\mu( \psi )$
przez $| \mu( \psi ) |$?

\vspace{\spaceFour}



\start \Str{423} W~dowodzie jest luka, trzeba bowiem wykazać,
że~$\psi_{ 1 } = \inf( \psi, \varphi )$ jest ciągła. Można to zrobić
tak: jeżeli $\psi( x ) > \varphi( x )$ (analogicznie dla
$\psi( x ) < \varphi( x )$) to z~ciągłości tych funkcji wynika,
że~istnieje otoczenie punktu $x$ na którym $\psi > \varphi$. W~tym
otoczeniu $\psi_{ 1 } = \varphi$ więc funkcja ta jest ciągła. Jeśli
$\psi( x ) = \varphi( x ) = a$ to dla każdego $\epsilon > 0$ istnieją
otoczenia tego punktu $A$ i $B$, że
$\psi( A ) \subset [ a - \epsilon, a + \epsilon ]$
i~$\varphi( B ) \subset [ a - \epsilon, a + \epsilon ]$. W~takim razie
na $A \cup B$
$\inf( \psi( x ), \varphi( x ) ) \subset [ a - \epsilon, a + \epsilon
]$, co dowodzi ciągłości. Wynik ten nie przenosi~się na infinimum już
przeliczalnego zbioru funkcji. Można bowiem dla funkcji
charakterystycznej zbioru $[ 0, 1 ]$ rozważyć ciąg zmierzających do niej
trapezów zawartych w przedziale
$[-\frac{ 1 }{ n }, 1 + \frac{ 1 }{ n } ]$. Infinimum tego zbioru
funkcji jest tą właśnie funkcją charakterystyczną.n

\vspace{\spaceFour}



\start \Str{429} Tę nierówność da się chyba uzyskać nie korzystając z
dodatniości miary.

\vspace{\spaceFour}


\start \Str{430} Zadanie stwierdzające, że
$( \Ocal_{ i } )_{ i \in I }$ jest pokryciem $K$ jest dość niezręczne.
Może da się to lepiej sformułować?

\start Str. 439. Warto podać dowód tego, że $\complement A$ jest
mierzalny. Zauważmy, że dla każdego kompaktu zachodzi
$\complement A \cap K = K \setminus ( A \cap K )$. Zbiór $A \cap K$
jest mierzalny i jako podzbiór kompaktu ma miarę skończoną.
Mierzalność $\complement A$ wynika więc z rozważań na stronie 436.

\vspace{\spaceFour}



\start \Str{446} Zarówno to stwierdzenie, że zbiory $A$ dla który
$f^{ -1 }( A )$ jest mierzalne stanowią $\sigma$-algebrę jak i~dowód
twierdzenia 24 wymagają posłużenie się relacjami
\begin{equation}
  \label{eq:SchwartzKAMVolI-05}
  f^{ -1 }( \bigcap_{ i } A_{ i } ) = \bigcap_{ i } f^{ -1 }( A_{ i } ), \quad
  f^{ -1 }( \bigcup_{ i } A_{ i } ) = \bigcup_{ i } f^{ -1 }( A_{ i } ).
\end{equation}
W~wypadku twierdzenia 24 wystarczy tylko pierwsza z~nich.

\vspace{\spaceFour}



\start \Str{449} W twierdzeniu 26. Schwartz konstruuje ciąg funkcji
piętrowych który jest tylko prawie wszędzie zbieżny do $f$. Z drugiej
strony dla przypadku $F = \Rbb$ Rudin
\cite{RudinAnalizaRzeczywistaIZespolona1998} konstruuje bez problemu
ciąg zbieżny punktowo. Czy wskazuje to na to, iż twierdzenie da się
poprawić lub znacznie uprościć dowód? Jeśli nie jest to prawdą, to
należy zbadać sprawę zbioru $E$ który jest traktowany jak
pseudomierzalny, acz nie wiem z czego to wynika.

\vspace{\spaceFour}



\start \Str{453} Dla każdego z~zbiorów $A \cap K_{ n }$ istnieją
zbiory borelowskie o zadanej własności i~mierze skończonej
$A_{ i }^{ * }$ oraz~$A_{ * i }$. Ich sumy są zbiorami mierzalnymi
(a~nawet borelowskimi) trzeba jeszcze pokazać, że miara ich różnicy
jest równa~0. Wynika to z~faktu, że~jeżeli $B_{ i } \subset A_{ i }$
to $\bigcup A_{ i } \setminus \bigcup B_{ i } \subset \bigcup ( A_{ i } \setminus B_{ i } )$ (łatwo to pokazać
rozpisując tą zależność za pomocą kwantyfikatorów). Teraz:
\begin{equation}
  \label{eq:SchwartzKAMVolI-06}
  \mu( A^{ * } \setminus A_{ * } ) = \mu\left( \bigcup A^{ * }_{ i } \setminus \bigcup A_{ * i } \right)
  \leq \mu\left( \bigcup ( A^{ * }_{ i } \setminus A_{ * i } ) \right)
  \leq \sum \mu( A^{ * }_{ i } \setminus A_{ * i } ) = 0.
\end{equation}

\vspace{\spaceFour}



\start \Str{456} Dowód wzoru (4.3.30) jest prosty, jeśli tylko uda się
go pokazać dla zbioru $A$~o~domknięciu zwartym. Wbrew uwadze na dole
strony nie jest to dla mnie oczywiste, poza przypadkiem gdy $A$~jest
mierzalny.

\vspace{\spaceFour}



\start \Str{457} Poprawić uwagi odnośnie przeniesienia twierdzeń z
całki Riemanna na całkę Lebesgue’a.

\vspace{\spaceFour}



\start \Str{458} W wniosku 1 nie da się chyba uniknąć założenia, że
zbiory $Y$ i $Z$ są mierzalne. W przeciwnym razie biorąc zbiór
niemierzalny i jego dopełnienie uzyskalibyśmy wniosek, że każda
funkcja całkowalna jest całkowalna na zbiorze niemierzalnym.

\vspace{\spaceFour}



\start \Str{458} Dla pełniejszego dowodu wniosku 2. warto pokazać, że
ciąg $\vec{ f } \varphi_{ A \cap K_{ n } }$ jest ciągiem
aproksymacyjny do funkcji $\vec{ f } \varphi_{ A }$. Wystarczy
zauważyć, że:
\begin{equation}
  \label{eq:SchwartzKAMVolI-07}
  \int^{ * } \Vert \vec{ f } \varphi_{ A } - \vec{ f } \varphi_{ A \cap K_{ n } } \Vert
  = \int^{ * } \Vert \vec{ f } \Vert \; | \varphi_{ A } - \varphi_{ A \cap K_{ n } } |
  = \Vert \vec{ f } \Vert \; \mu( A \setminus A \cap K_{ n } ) \to 0.
\end{equation}

\vspace{\spaceFour}



\start \Str{462} Całe rozumowanie prowadzące do wzoru $(4.3.44)$ jest
absolutnie poprawne, tylko że ten wzór z niego nie wynika. I nie wiem
jak to poprawić.

\vspace{\spaceFour}



\start \Str{465} W twierdzeniu Jegorowa założenie, że~$K$ jest zwarty
jest chyba nadmiarowe (patrz Uwaga 1 do rozdziału IV). W~zupełności
powinno wystarczyć, że~jest zbiorem mierzalnym o~mierze skończonej,
zobacz \cite{RudinAnalizaRzeczywistaIZespolona1998}.

\vspace{\spaceFour}



\start \Str{469} Rozumowanie które prowadzi do zależności
$\int \Vert \vec{ f } - \vec{ f }_{ n } \Vert \to 0$ wydaje się
wyglądać następująco. Skoro pokazaliśmy, że $\vec{ f }$ jest
całkowalna, całkowalna jest funkcja $\vec{ f } - \vec{ f }_{ n }$ z
nią zaś $\Vert \vec{ f } - \vec{ f }_{ n } \Vert$. Oznacza to w
szczególności
$\int^{ * } \Vert \vec{ f } - \vec{ f }_{ n } \Vert = \int \Vert \vec{
  f } - \vec{ f }_{ n } \Vert$. \textit{To rozumowanie można chyba mocno
  poprawić.}

\vspace{\spaceFour}



\start \Str{471} W dowodzie twierdzenia Fatou używa się faktu, że góra
i dolna obwiednia dwóch funkcji całkowalnych jest funkcją
całkowalną, jednak ten fakt nie jest chyba nigdzie dowiedziony. \\
Pomysł na dowód: rozważyć ciągi aproksymacyjne $f_{ n }$, $g_{ n }$
dla funkcji $f$ i~$g$ złożone z funkcji schodkowych i~pokazać,
że~$\sup( f_{ n }, g_{ n } )$ jest ciągiem aproksymacyjnym funkcji
schodkowych dla $\sup( f, g )$. \start Str. 473. Drugi punkt wniosku 2
można wzmocnić. Zauważmy bowiem, że dla ciągu rosnącego mamy zawsze
$\int f_{ 0 } \leq \int f_{ n }$, czyli dla ciągu rosnącego musi być
$\int f_{ n } \to +\infty$ i $\int f_{ n } \to -\infty$ dla
malejącego. \textit{Możliwe, że ten dowód da się poprawić.}

\vspace{\spaceFour}



\start \Str{477} W założeniach twierdzenia 42 nie jest podane, że
przestrzeń jest przeliczalna w nieskończoności, a wydaje się to
potrzebne w dowodzie dla otrzymania wstępującego ciągu kompaktów
pokrywających całą przestrzeń.

\vspace{\spaceFour}



\start \Str{479} W dowodzie twierdzenia wniosku 4, jest przywołane
twierdzenie 26 i uwaga 2 do niego. Jednak twierdzenie 26 nie ma uwagi
numer 2, nie chodzi też o wniosek 2 z tego twierdzenia. Jednak sam
dowód faktu dla którego odwołano się do tej uwagi nie będzie raczej
trudny.

\vspace{\spaceFour}



\start \Str{490} Jestem w stanie udowodnić, ale chyba mało elegancko,
nierówności (4.4.61), za to nierówność (4.4.58) z trójmianu (4.4.59)
jestem w stanie uzyskać tylko stosując rozumowanie które Schwartz
przedstawił na stronie 195.

\vspace{\spaceFour}



\start \Str{491} Funkcje podane w punkcie $1^{ \circ }$ mogą być
nieokreślone dla 0. Dla spójności paragrafu trzeba je tam dookreślić,
ponieważ jednak punkt ma miarę 0, więc konkretny sposób rozszerzenia
tej funkcji nie ma znaczenia.

\vspace{\spaceFour}



\start \Str{491} W punkcie $2^{ \circ }$ który dotyczy funkcji na
$\Rbb$ jest przywoływany przykład funkcji z punktu $1^{ \circ }$
określonej tylko na odcinku $[ 0, 1 ]$ (zobacz poprzednią uwagę).
Należy to chyba rozumieć, że tą funkcję przedłużamy do funkcji
wynoszącej 0 poza tym odcinkiem.

\vspace{\spaceFour}



\start \Str{493} Odniesienie do rozdziału I strona 8, jest błędne.
Strona 8 jest wstęp od tłumacza.

\vspace{\spaceFour}



\start \Str{495} Część twierdzenia 55 która mówi, że~funkcje z~danej
przestrzeni można przybliżyć funkcjami ciągłymi rozkładalnymi
o~zwartym nośniku, jak wykazano nie zachodzi, jednak nic nie zostało
powiedziane o funkcjach piętrowych.

\vspace{\spaceFour}



\start \Str{498} Zupełnie nie rozumiem jak z warunku 3) wynika warunek
3').

\vspace{\spaceFour}



\start \Str{502} Na tej stronie pierwszy raz pojawia się miara zbioru
$\mu( A )$, gdzie $\mu$ nie musi być miarą dodatnią, jednak wielkość
ta nie zostaje zdefiniowana. Ze względu na definicję rozszerzenia
borelowskiego miary, dla zbioru borelowskiego $A$ o domknięciu zwartym
za miarę tego zbioru należy przyjąć wartość tego rozszerzenia na jego
funkcji charakterystycznej. Z tego powodu należy dodać obstrzyżenie
przy wzorach (4.4.68), że zbiory $B$ są borelowskie w $X$ (dopóki $A$
jest borelowskie, równoważne można powiedzieć, że mają być borelowskie
w~$A$).

\vspace{\spaceFour}



\start \Str{504} W definicji miary całkowicie pozytywnej (zerowej,
negatywnej) nie~jest jasne czy zbiory $B$ mają być borelowskie w~$A$
czy w~$X$. Należy chyba przyjąć taką definicję. Niech $A$ będzie
\textit{dowolnym} zbiorem w $X$. Miara $\mu$ jest całkowicie pozytywna
(zerowa, negatywna) na $A$, jeśli dla każdego zbioru $B$ borelowskiego
w~$X$ zawartemu w~$A$ o~domknięciu zwartym zachodzi $\mu( B ) \geq 0$
($= 0$, $\leq 0$).

\vspace{\spaceFour}



\start \Str{508} Udowodnić twierdzenie 61.

\vspace{\spaceFour}



\start \Str{511} Wyjaśnienie dlaczego $A_{ h, n } \cap A_{ k, n }$ ma
miarę zerową. Jeżeli $h + 1 = k$ to zgodnie z~tym co napisano na tej
stronie pod koniec drugiego akapitu,
$X^{ - }_{ h + 1, n } \cap X^{ + }_{ k, n } = \emptyset$. Jeżeli
$h + 1 < k$ to $X^{ - }_{ h + 1, n } \cap X^{ + }_{ k, n } \subset N_{ n }$.

\start \Str{512} Nie jest wcale dla mnie oczywiste czemu zbiory
$A_{ 2 k,\, n+ 1 }$ i~$A_{ 2 k + 1,\, n+ 1 }$ dają cały
zbiór~$A_{ k,\, n }$, ale można chyba pokazać, że~dają prawie cały ten
zbiór. Mamy bowiem
$A_{ 2 k,\, n + 1 } = X^{ + }_{ k,\, n } \cap X^{ - }_{ 2 k + 1,\, n +
  1 }$ i
$A_{ 2 k + 1,\, n + 1 } = X^{ + }_{ 2 k + 1,\, n } \cap X^{ - }_{ k,\,
  n }$. Co dalej?

\vspace{\spaceFour}



\start \Str{514} Nie rozumiem dowodu nie wprost (albo przez
kontrapozycję, kto to rozróżnia) na dole strony. Należy go w miarę
możliwości poprawić.

\vspace{\spaceFour}



\start \Str{531} W dowodzie zakłada się, że~przestrzeń jest
przeliczalna w~nieskończoności choć dowód nie~wydaje~się z~tego
korzystać. Jednak w~definicji odwzorowania $\mu$\dywiz właściwego
zakłada się tylko, że~miara przeciwobrazu każdego kompaktu jest
skończona, a~naturalne byłoby sformułowanie, iż~miara przeciwobrazu
dowolnego zbioru o~mierze skończonej jest skończona.
Najprawdopodobniej w~wypadku przestrzeni przeliczalnej w
nieskończoności oba te pojęcia pokrywają się.

\vspace{\spaceFour}



\start \Str{532} Udowodnić twierdzenie 73.

\vspace{\spaceFour}



\start \Str{532} Coś mi nie pasuje w~rozumowaniu prowadzącym do wzoru
(4.6.31). Wynik jest poprawny, jednak jego dowód zdaje się zawierać
lukę.

\vspace{\spaceFour}



\start \Str{544} Udowodnić twierdzenie 77.

\vspace{\spaceFour}



\start \Str{576} W~tym paragrafie nieuważnie zostały potraktowany
problem tego, czy rozpatrujemy przedziały na $\Rbb$ czy na
$\overline{\Rbb}$. Jeżeli chcemy dyskutować ogólne miary o~bazie
dodatniej to należy się ograniczyć do $\Rbb$, w~przeciwnym razie
udowodnione rezultaty nie będą obejmowały miary Lebesgue'a pochodzącej
od całki Riemanna: miara zbioru $[ -\infty, 0 ]$ jest dla niej
nieskończona (warto zauważyć, że~zbiór ten jest zwarty w~topologii
$\overline{\Rbb}$). Wtedy jednak należy dodać, że~dla szczególnych
przypadków np.~gdy masa miary jest skończona, jak dla miary Diraca,
można rozpatrywać przedziały $\overline{\Rbb}$ z~topologią tej
przestrzeni.

\vspace{\spaceFour}



\start \Str{594} W~6~linii od~dołu jest odniesienie do twierdzenia 24
z~rozdziału III. Jednak jest to~błąd, musi chodzić o~inne twierdzenie
z~tego rozdziału.

\vspace{\spaceFour}



\start \Str{600} Dlaczego całka z~funkcji $f$ o~której mowa
w~przypisie, istnieje?

\vspace{\spaceFour}



\start \Str{625} Zdanie że~jesteśmy daleko od~ścisłego dowodu wniosku
7 jest niepoprawne. Przedstawiony dowód jest~zupełnie ścisły.

\vspace{\spaceFour}



\start \Str{628} Nie~widzę z~czego wynika wniosek 2.

\vspace{\spaceFour}



\start \Str{636} W~linii 14 od~dołu jest odwołanie do~wniosku 1
z~twierdzenia 36. Musi chodzić o~jakieś inne twierdzenie.

\vspace{\spaceFour}



\start \Str{654} W~przypisie 2 należy zaznaczyć, że~pojęcie niemal
jednostajnej zbieżności na $\Rbb$ jest identyczne z~pojęciem
zbieżności lokalnej na $\Rbb$. Zapewne jest tak na ogólniejszej klasie
przestrzeni topologicznych, jednak tu trzeba~się chwilę zastanowić. %










% ##################
\newpage
\CenterBoldFont{Błędy}

\begin{center}

  \begin{tabular}{|c|c|c|c|c|}
    \hline
    & \multicolumn{2}{c|}{} & & \\
    Strona & \multicolumn{2}{c|}{Wiersz} & Jest
                              & Powinno być \\ \cline{2-3}
    & Od góry & Od dołu & & \\
    \hline
    10  &  9 & & $E$ i jeżeli $A \supset B$, to & $E$, to \\
    10  & &  6 & zawierają $n$ elementów & zawierają $n$-ty element \\
    11  & 10 & & element z $F$ & dokładnie jeden element z $F$ \\
    12  & 18 & & nazywa się \textit{stałe} & nazywa się \textit{stałym} \\
    13  & & 11 & $d^{ -1 }( \{ y \} )$ & $f^{ -1 }( \{ y \} )$ \\
    14  &  3 & & Mamy & Mamy ponadto \\
    16  &  2 & & funckji & funkcji \\
    18  &  9 & & właśnie liczba & liczba \\
    19  & &  8 & jeżeli$( x, y ) \in R$ & jeżeli $( x, y ) \in R$ \\
    22  &  6 & & $b_{ 1 } \leq x \leq b$ & $b_{ 1 } < x < b$ \\
    22  & 18 & & że są dwa & że istnieją dwa \\
    23  & & 18 & zbiór $E$ & zbiór $R$ \\
    23  & &  6 & uporządkowanym & chaotycznie uporządkowanym \\
    25  & & 11 & Oczywiste~są & Nie oczywiste~są \\
    26  & &  1 & tłnm)$\cdot$ & tłum.). \\
    27  &  5 & & i skoro
           & i skoro, na mocy iniekcji $x \mapsto \{ x \}$, \\
    27  &  7 & & $f_{ 0 }( x ) \neq \emptyset$
           & $f_{ 0 }( x ) = \emptyset$ \\
    31  &  3 & & gdy & gdyż \\
    34  &  5 & & $( \forall a \in \Rbb ) ( \forall \epsilon > 0 )$
           & $( \forall \epsilon > 0 ) ( \forall a \in \Rbb )$ \\
    36  & &  8 & funkcję & funkcję rzeczywistą \\
    37  &  1 & & $\Vert 0 \Vert = \vecZero$ & $\Vert \vec{ 0 } \Vert = 0$ \\
    37  &  5 & & $\Vert \vecx_{ 1 } + \vecx_{ 2 } + \ldots + \vecx_{ n } |$
           & $\Vert \vecx_{ 1 } + \vecx_{ 2 } + \ldots + \vecx_{ n } \Vert$ \\
    40  & 14 & & est & jest \\
    40  & & 18 & Przechodzą & Przechodząc \\
    40  & & 18 & dotyczący h & dotyczących \\
    41  & & 11 & (a) & (a$^{ \prime\prime }$) \\
    43  & 10 & & to nie należy do żadnego & to nie zawiera się w żadnym \\
    43  & 18 & & $\overline{A,}$ & $\overline{A},$ \\
    43  & & 15 & \textit{zbioru $E$} & \textit{zbioru $A$} \\
    43  & &  3 & nie przecinającym się & przecinającym się \\
    44  & 19 & & metryczna & metryczną \\
    44  & 20 & & \textit{ośrodkowa} & \textit{ośrodkową} \\
    44  & 21 & & wszędzie gęstą & gęstą \\
    45  &  5 & & $B_{ 0 }( a, R )$ & $B_{ o }( a, R )$ \\
    45  &  6 & & $\beta_{ 0 }( a, R ) = B_{ 0 }( a, R ) \cap F$
           & $\beta_{ o }( a, R ) = B_{ o }( a, R ) \cap F$ \\
    45  &  8 & & $\beta( a_{ i }, R_{ i } )_{ i \in I }$
           & $\beta_{ o }( a_{ i }, R_{ i } )_{ i \in I }$ \\
    \hline
  \end{tabular}

\end{center}





\begin{center}

  \begin{tabular}{|c|c|c|c|c|}
    \hline
    & \multicolumn{2}{c|}{} & & \\
    Strona & \multicolumn{2}{c|}{Wiersz} & Jest
                              & Powinno być \\ \cline{2-3}
    & Od góry & Od dołu & & \\
    \hline
    45  &  8 & & $B_{ 0 }( a_{ i }, R_{ i } )_{ i \in I }$
           & $B_{ o }( a_{ i }, R_{ i } )_{ i \in I }$ \\
    45  & 13 & & $C$ & $C = C_{ 1 } \cap F$ \\
    45  & 16 & & $\dot{ C }_{ 1 } \cap F = C$ & $C_{ 1 } \cap F = C$ \\
    45  & 18 & & części $A_{ 1 }$ & części $A$ \\
    45  & 19 & & zbioru $A$ & zbioru $A_{ 1 }$ \\
    46  & 16 & & $f( \: \Ucal ) \subset \Vcal$ & $f( \Ucal ) \subset \Vcal$ \\
    48  &  6 & & jedynie z & z \\
    50  & 18 & & $d' = \inf( d, 1 )$ & $d' = \min( d, 1 )$ \\
    51  &  9 & & lecz także & lecz \\
    51  & & 14 & musi być to & istnieć \\
    51  & & 13 & topologiczna nazywa się & topologiczną nazywa się \\
    & & & \textit{regularna} & \textit{regularną} \\  % popraw
    52  &  1 & & regularn a & regularna. \\
    52  &  2 & & otoczeń & otoczeń domkniętych \\
    52  & &  1 & $l \in$ & $x_{ n } \in$ \\
    53  &  7 & & metryczną; zbieżność & metryczną, zbieżność \\
    56  & 12 & & $E_{ 1 } \times E_{ 2 }$ & $E_{ 1 }$ i $E_{ 2 }$ \\
    56  & 21 & & przedział & zbiór \\
    59  & &  8 & \textit{na pewno nigdy} & \textit{nigdy} \\
    61  &  8 & & $[ -\frac{ 1 }{ 2 } \pi, \frac{ 1 }{ 2 } \pi [$
           & $] -\frac{ 1 }{ 2 } \pi, \frac{ 1 }{ 2 } \pi [$ \\
    63  &  6 & & $| \vecx | = \max_{ i = 1, \ldots, n} | \vecx_{ i } |$
           & $| \vecx | = \max_{ i = 1, \ldots, n} | x_{ i } |$ \\
    65  &  8 & & zbiór jego punktów skupienia & jego domknięcie \\
    68  & &  3 & $f^{ -1 } \{ b \}$ & $f^{ -1 }( \{ b \} )$ \\
    70  &  8 & & \textit{kompalitu} & \textit{kompaktu} \\
    71  & & 16 & $F$ & $E$ \\
    71  & & 10 & $B$ $\cap F$ & $B \cap F$ \\
    72  &  5 & & $d = \inf_{ \substack{ k_{ 1 } \in F_{ 2 } } }
                 d( x_{ 1 }, F_{ 2 } )$
           & $d = \inf_{ \substack{ k_{ 1 } \in F_{ 2 } } }
             d( x_{ 1 }, k_{ 1 } )$ \\
    72  &  8 & & $F$ & $E$ \\
    72  & 11 & & odwzorowania & przypadku \\
    72  & & 13 & wewnątrz okręgu & na zewnątrz okręgu \\
    75  & &  4 & części & niepuste części \\
    75  & &  3 & części & niepuste części \\
    76  &  3 & & $F$ & że $F$ \\
    77  & 19 & & dwie & dwie niepuste, \\
    77  & &  7 & to nie jest nic innego & to coś innego \\
    77  & &  2 & lub & Lub \\
    78  & &  1 & się zbiory & się niepuste zbiory \\
    79  &  5 & & się i & się, niepustych i \\
    \hline
  \end{tabular}

\end{center}





\begin{center}

  \begin{tabular}{|c|c|c|c|c|}
    \hline
    & \multicolumn{2}{c|}{} & & \\
    Strona & \multicolumn{2}{c|}{Wiersz} & Jest
                              & Powinno być \\ \cline{2-3}
    & Od góry & Od dołu & & \\
    \hline
    79  & 21 & & ponieważ $B$ & ponieważ $B''$ \\
    80  &  2 & & \textit{częścią} & \textit{częścią spójną} \\
    80  & 12 & & do której należy & zawierająca \\
    82  &  4 & & $F_{ x }$ & $E_{ x }$ \\
    83  & & 12 & $] f^{ - 1 }( \alpha ), f^{ -1 }( \alpha' ) [$
           & $] a, f^{ -1 }( \alpha ) [$ \\
    86  &  4 & & $\frac{ {} }{ 2 }$ & $\frac{ 1 }{ 2 }$ \\
    93  & & 11 & $d[ f_{ \lambda }( a_{ \lambda } ), df_{ \lambda_{ 0 } }( a_{ \lambda_{ 0 } } ) ]$
           & $d[ f_{ \lambda }( a_{ \lambda } ), f_{ \lambda_{ 0 } }( a_{ \lambda_{ 0 } } ) ]$ \\
    94  & 17 & & $\big( \sum_{ i = 1 }^{ n } \Vert u( \vec{ e }_{ i } ) \Vert$
           & $\big( \sum_{ i = 1 }^{ n } \Vert u( \vec{ e }_{ i } ) \Vert \big)$ \\
    98  & & 17 & $\vecx_{ n } + \vecx_{ n }$
           & $\vecx_{ n } + \vecy_{ n }$ \\
    104 &  1 & & $| \vecy \Vert / \eta$ & $\Vert \vecy \Vert / \eta$ \\
    125 & &  1 & twierdzenie 44 & twierdzenie 53 \\
    128 & &  8 & $x \in \epsilon [ a, b ]$ & $x \in [ a, b ]$ \\
    129 & &  3 & nie jest & nie musi \\
    134 &  2 & & $d[ \tilde{ f }, \tilde{ f }_{ n } ]$
           & $d( \tilde{ f }, \tilde{ f }_{ n } )$ \\
    136 & 15 & & $ln( 1 + x )$ & $\ln( 1 + x )$ \\
    137 &  7 & & będz e & będzie \\
    139 & &  4 & półtoraliniową & półtoraliniową hermitowską \\
    140 &  4 & & $( \vecx | \vecy )^{ 1 / 2 }$
           & $( \vecx | \vecx )^{ 1 / 2 }$ \\
    141 &  9 & & z ciągłości & wynika z ciągłości \\
    141 &  9 & & której można łatwo dowieść & co można łatwo pokazać \\
    141 & 18 & & \textit{hilbertowską} & \textit{prehilbertowską} \\
    143 & & 11 & $d[ \vecb, \frac{ 1 }{ 2 } ( \vec{ \alpha }, \vec{ \beta }) ]$
           & $d[ \vecb, \frac{ 1 }{ 2 } ( \vec{ \alpha } + \vec{ \beta }) ]$ \\
    146 & & 14 & \textit{jest dopełnieniem} & \textit{jest domknięciem} \\
    146 & & 13 & \textit{dopełnieniem} & \textit{domknięciem} \\
    146 & &  9 & dopełnieniami & dopełnieniami algebraicznymi \\
    147 & &  1 & $\vec{ \dot{ x } }$ (lub $\vec{ \dot{ y } }$)
           & $\vecx$ (lub $\vecy$) \\
    148 & &  3 & Stąd wiemy, że & Ponadto zachodzi \\
    149 &  6 & & \textit{dwuciągłą} & \textit{ciągłą} \\
    149 & 16 & & innymi słowy & stąd mamy \\
    151 & &  3 & jest & fakt, że jest \\
    152 &  8 & & $\Vert ( \vecx_{ i } | \vecy_{ i } )_{ \vecE_{ i } } \Vert$
           & $| ( \vecx_{ i } | \vecy_{ i } )_{ \vecE_{ i } } |$ \\
    153 &  2 & & $\Vert \vecx_{ ni } \Vert_{ \vecE } \leq \Vert \vecx_{ n } \Vert_{ \vecE }$
           & $\Vert \vecx_{ ni } \Vert_{ \vecE_{ i } } \leq \Vert \vecx_{ n } \Vert_{ \vecE }$ ??? \\
    153 & &  9 & $\Vert \vecx_{ i } \Vert_{ \vecE }$
           & $\Vert \vecx_{ i } \Vert_{ \vecE_{ i } }$ \\
    153 & &  7 & $\leq \Vert \vecx_{ n } \Vert_{ \vecE }$
           & $\Vert \vecx_{ n } \Vert_{ \vecE }$ \\
    154 & 19 & & $\vecx - ( \vecx_{ i } )_{ i \in I }$
           & $\vecx = ( \vecx_{ i } )_{ i \in I }$ \\
    155 & &  6 & $x_{ i } \bar{ y }$ & $x_{ i } \bar{ y }_{ i }$ \\
    157 &  4 & & izomorfizmem & izometrycznym izomorfizmem \\
    160 & &  6 & ponieważ & ponieważ wtedy \\
    \hline
  \end{tabular}

\end{center}





\begin{center}

  \begin{tabular}{|c|c|c|c|c|}
    \hline
    & \multicolumn{2}{c|}{} & & \\
    Strona & \multicolumn{2}{c|}{Wiersz} & Jest
                              & Powinno być \\ \cline{2-3}
    & Od góry & Od dołu & & \\
    \hline
    163 & 10 & & $\Vert U \Vert$ & $\Vert U \Vert =$ \\
    163 & 16 & & $\Vert i\vecy \Vert^{ 2 }$ & $\Vert \vec{ y } \Vert^{ 2 }$ \\
    164 & & 14 & $k \geq 0$ & $k > 0$ \\
    165 & 16 & & $k \geq 0$ & $k > 0$ \\
    165 & & 18 & odwracalne obustronnie & jest elementem odwrotnym \\
    165 & &  5 & $\big{ | } \big( ( u + \sigma I ) \vecx |
                 \vecx \big) + i \tau \Vert \vecx \Vert^{ 2 }$
           & $\big{ | } \big( ( u + \sigma I ) \vecx | \vecx \big)
             + i \tau \Vert \vecx \Vert^{ 2 } \big{ | }$ \\
    169 &  5 & & $u_{ k }( B )$ & $u( B )$ \\
    171 & & 15 & $C_{ [ 0, 1 ] } \{ \lambda \}$
           & $C( [ 0, 1 ] \backslash \{ \lambda \} )$ \\
    172 &  9 & & zwarte & zawarte \\
    176 &  9 & & $\lambda_{ 0 } \in \overline{ \Lcal }$
           & $\lambda_{ 0 } \notin \overline{ \Lcal }$ \\
    180 &  1 & & $\prod^{ \infty }_{ n = 1 } \frac{ 1 }{ n }$
           & $\sum^{ \infty }_{ n = 1 } \frac{ 1 }{ n }$ \\
    180 &  2 & & twierdzeniu 82 & twierdzeniu 115 \\
    181 & & 14 & twierdzenia 82 & twierdzenia 115 \\
    201 & 16 & & lub & i \\
    206 & &  6 & $a \leq x \leq a + \eta$ & $a < x \leq a + \eta$ \\
    207 & 12 & & $\leq 0$ & $\geq 0$ \\
    213 &  1 & & $E$ także & $E$ \\
    214 & &  3 & był & obecny był \\
    221 & &  2 & $\vec{ \alpha }_{ m } \Vert \overrightarrow{ dx } \Vert$
           & $\vec{ \alpha }_{ m } \Vert \overrightarrow{ dx } \Vert )$ \\
    224 & & 12 & $| B'( \vecx ) \cdot \vecX \Vert$
           & $\Vert B'( \vecx ) \cdot \vecX \Vert$ \\
    226 & 10 & & swoimi pochodnymi & swą pochodną \\
    226 & 11 & & swoimi pochodnymi & swą pochodną \\
    \hline
  \end{tabular}

\end{center}





\begin{center}

  \begin{tabular}{|c|c|c|c|c|}
    \hline
    & \multicolumn{2}{c|}{} & & \\
    Strona & \multicolumn{2}{c|}{Wiersz} & Jest
                              & Powinno być \\ \cline{2-3}
    & Od góry & Od dołu & & \\
    \hline
    226 & 15 & & $\max_{ x \in \Omega }$ & $\sup_{ x \in \Omega }$ \\
    226 & 16 & & $\L( \vec{ E }; \vec{ F } )$
           & $\L( \vec{ E }; \vec{ F } )$) \\
    228 & 16 & & szmo & samo \\
    233 & & 12 & $y = f( x )$ & $y = f( x )$, $f : E \to F$ \\
    237 & & 11 & rozwiązalne & rozwiązywalne \\
    242 & &  7 & nie musi & nie może \\
    245 &  6 & & liczbie & odwzorowaniu \\
    245 &  9 & & $x \mapsto
                 \frac{ \overrightarrow{ f( x + \vec{ h } ) - f( x ) } }{ \vec{ h } }$
           & $x \mapsto \frac{ \overrightarrow{ f( x + h ) - f( x ) } }{ h }$ \\
    250 & 11 & & $f'( a \Big)$ & $f'( a ) \Big)$ \\
    253 & 17 & & $m \}$ & $n \}$ \\
    254 & 13 & & $\vec{ X }^{ p_{ 1 } }_{ 1 }$ & $X^{ p_{ 1 } }_{ 1 }$ \\
    254 & &  3 & $m \geq 2$ & dla $m - 1$, gdzie $m \geq 2$ \\
    259 & 12 & & $\frac{ f^{ ( m + 1 ) }( x + \theta h ) }{ ( m + 1 )! }$
           & $\frac{ f^{ ( m + 1 ) }( x + \theta \vec{ h } ) }{ ( m + 1 )! }$
    \\
    259 & & 15 & $[ 0, 1 ]$ & $[ 0, 1 ]$) \\
    260 &  9 & & przyjmując,~że & zauważając,~że \\
    265 & 15 & & $\frac{ \vec{ h }^{ \vec{ p } } }{ p! }$
           & $\frac{ \vec{ h }^{ \vec{ p } } }{ \vec{ p }! }$ \\
    269 &  9 & & $\Lbb( \overrightarrow{ E }, \Rbb )$
           & $\Lbb( \overrightarrow{ E }, \Rbb )$ \\
    276 &  6 & & $\vec{ c }$ & $c$ \\
    280 &  5 & & $Q^{ - } \circ P$ & $Q^{ -1 } \circ P$ \\
    280 & &  9 & ciągłe & różniczkowalne \\
    281 & 18 & & $V \mapsto W = u_{ 0 } W$ & $V \mapsto W = u_{ 0 } V$ \\
    281 & 19 & & ciągłym$W$ & ciągłym $W$ \\
    282 & &  8 & $K^{ n^{ 2 } }$
           & $\Ucal$, bo $\det( \Mfrak^{ -1 } ) = 1 / \det( \Mfrak )$, \\
    283 &  5 & & \textit{afinicznej} & \textit{afinicznej unormowanej} \\
    284 &  7 & & zaś jest & jest więc \\
    288 & 11 & & odwzorowanie $B$ jest otwarte & zbiór $B$ jest otwarty \\
    288 & 11 & & odwzorowanie $A$ jest otwarte & zbiór $A$ jest otwarty \\
    289 & 23 & & \textit{w przestrzeń} & \textit{w zupełną przestrzeń} \\
    291 & &  6 & twierdzenia~49 & twierdzenia~50 \\
    \hline
  \end{tabular}

\end{center}





\begin{center}

  \begin{tabular}{|c|c|c|c|c|}
    \hline
    & \multicolumn{2}{c|}{} & & \\
    Strona & \multicolumn{2}{c|}{Wiersz} & Jest
                              & Powinno być \\ \cline{2-3}
    & Od góry & Od dołu & & \\
    \hline
    297 &  8 & & $\Lcal( \overrightarrow{ K }^{ n }, E )$
           & $\Lcal( K^{ n }, E )$ \\
    297 & & 17 & w $V$ & w $E$ \\
    298 & & 20 & $\Phi( A )$ & $\Phi( \Acal )$ \\
    299 & & 16 & $K^{ n }$ & $K^{ N }$ \\
    299 & & 15 & z $K^{ n }$ & $K^{ n }$ \\
    300 & 12 & & $\Phi$ & $\tilde{ \Phi }$ \\
    373 & &  7 & i (4.1.4) & (4.1.4) \\
    374 &  1 & & $( d_{ l + 1 } - d_{ l } )\; )$
           & $( d_{ l + 1 } - d_{ l } )$ \\
    374 & 16 & & $\int ( b - a )$ & $( b - a )$ \\
    374 & &  5 & $f > 0$ & $f \geq 0$ \\
    374 & &  2 & & $\int^{ * } \! f$ \\
    375 &  2 & & & $\int^{ * } \! f$ \\
    375 & & 12 & Biorąc kres górny prawej & Biorąc kres dolny prawej \\
    379 & &  3 & $\int\limits_{ [ b, c ]  } \Vert \vec{ f } - \vec{ f }_{ n } \Vert$
           & $\int^{ * }_{ [ b, c ]  } \Vert \vec{ f } - \vec{ f }_{ n } \Vert$ \\
    380 & & 11 & $\int\limits_{ [ a, b ] }$
           & $\int\limits_{ [ a, b ] } \vec{ f } $ \\
    381 & 11 & & & $\int^{ * } \! f$ \\
    382 & & 13 & $||| \vec g_{ n } ||| \leq 3 ||| \vec{ \; \; } |||$
           & $||| \vec{ g_{ n } } ||| \leq 3 ||| \vec{ g } |||$ \\
    382 & &  1 & $3 || \vec{ g } |||$ & $3 ||| \vec{ g } |||$ \\
    385 & 13 & & $[ c_{ i }, c_{ i + 1 } ]$ & $[ c_{ i }, c_{ i + 1 } [$ \\
    387 & 14 & & jeden & co najmniej jeden \\
    392 & & 15 & punktów $a_{ \nu }$ & punktom $a_{ \nu }$ \\
    \hline
  \end{tabular}

\end{center}





\begin{center}

  \begin{tabular}{|c|c|c|c|c|}
    \hline
    & \multicolumn{2}{c|}{} & & \\
    Strona & \multicolumn{2}{c|}{Wiersz} & Jest
                              & Powinno być \\ \cline{2-3}
    & Od góry & Od dołu & & \\
    \hline
    392 & & 13 & $d''_{ i + }$ & $d''_{ i + 1 }$ \\
    393 & & 15 & $q_{ i } / | q_{ i } |$ & $\bar{ q }_{ i } / | q_{ i } |$ \\
    395 &  6 & & forma liniowa & formę liniową \\
    395 &  7 & & określona & określoną \\
    395 & &  2 & zeru (każde & zeru, bowiem każde \\
    396 &  1 & & punktów & punkty \\
    396 &  2 & & uwzględnić, że & pominąć, tak że \\
    396 & & 15 & $\int\limits_{ [ b', b ] } | \varphi( x ) | \, dx$
           & $\int\limits_{ [ b', b ] } | p( x ) | \, dx$ \\
    398 & &  3 & równa $\sqrt{ 2 }$ & wynosi $\sqrt{ 2 }$ \\
    399 &  5 & & otoczenia & pewnego otoczenia \\
    399 &  8 & & tym przypadku & tej sytuacji \\
    399 & & 18 & \textit{zwartym} & \textit{zwartym zawartym} \\
    399 & & 10 & $\eta > 0$ & $1 \geq \eta > 0$ \\
    401 & &  9 & \textit{zeru} & \textit{tożsamościowo zeru} \\
    404 & &  4 & dopełnień $F'_{ i }$ & zbiorów $F'_{ i }$ \\
    405 & 11 & & \textit{ciągłą na $\vec{ F }$} & \textit{ciągłą na $F$} \\
    406 & 16 & & przedłu & przedłu- \\
    407 & &  7 & na $K$ o & o nośniku $K$~i \\
    408 & 14 & & $\Vert \vec{ f }( a_{ i } ) \Vert \alpha_{ i }( x ) \big| \leq$
           & $\Vert \vec{ f }( a_{ i } ) \Vert \alpha_{ i }( x ) \big| =$ \\p
    409 & 14 & & całym $K$ & całym $X$ \\
    410 & 17 & & Ale należąc do & Należąc więc do \\
    411 & &  7 & zawartym & zawartym \\
    413 & 12 & & punktu & punktów \\
    415 & 16 & & o zwartym nośniku & o nośniku \\
    416 & 17 & & Nierówność & Zachodzi nierówność \\
    416 & & 12 & $\Omega_{ i }$ & $\Omega_{ j }$ \\
    417 & &  9 & (4.2.56) & (4.2.56)) \\
    419 & 15 & & $\mu( \varphi ) + \varepsilon$ & $\mu( | \varphi | ) + \varepsilon$ \\
    420 & &  4 & nieciągła & ciągła \\
    421 & 18 & & Mamy oczywiście & Mamy bowiem \\
    421 & &  9 & Mało tego, każdy & Wynika stąd, że każdy \\
    422 & & 10 & $| \Psi | $ & $| \psi |$ \\
    422 & & 10 & $| \Psi | \in$ & $| \psi | \in$ \\
    424 & &  4 & $| \mu^{ + }( \varphi ) \Vert$ & $| \mu^{ + }( \varphi ) |$ \\
    \hline
  \end{tabular}

\end{center}





\begin{center}

  \begin{tabular}{|c|c|c|c|c|}
    \hline
    & \multicolumn{2}{c|}{} & & \\
    Strona & \multicolumn{2}{c|}{Wiersz} & Jest
                              & Powinno być \\ \cline{2-3}
    & Od góry & Od dołu & & \\
    \hline
    425 & 10 & & normy są addytywne & $\mu = \mu^{ + } - \mu^{ - }$ \\
    425 & 12 & & $\Vert \mu^{ + } \Vert$ & $\Vert \mu^{ + } \Vert_{ K }$ \\
    425 & &  2 & \textit{na $\Ccal( X ) \geq 0$ nad $\Ccal^{ + }( X )$}
           & \textit{dodatnia na $\Ccal( X )$} \\
    431 & & 12 & $\geq$ & $=$ \\
    432 &  1 & & $\varphi \geq 1$ & $\varphi = 1$ \\
    433 &  8 & & mierzalne & pseudomierzalne \\
    433 & & 15 & kres górny & kres dolny \\
    433 & &  9 & Domknięcia & Przypomnijmy, że domknięcia \\
    435 &  1 & & $\varphi( x ) \geq 1$ & $\varphi( x ) = 1$ \\
    435 & 12 & & $=$ & $\leq$ \\
    436 & 15 & & str. 440 & str. 441 \\
    436 & 18 & & zbiór jest & zbiór ten jest \\
    437 &  2 & & $\sum\limits^{ N }_{ n = 0 } \mu( O_{ n } ??????
                 K_{ n } ) \leq$
           & $\sum\limits^{ N }_{ n = 0 } \mu( O_{ n } \backslash K_{ n } ) =$ \\
    437 &  3 & & $\leq \sum\limits^{ N }_{ n = 0 } [ \mu( O_{ n } )
                 - \mu( K_{ n } ) ]$
           & $= \sum\limits^{ N }_{ n = 0 } [ \mu( O_{ n } )
             - \mu( K_{ n } ) ]$ \\
    437 &  8 & & $\overline{ A_{ 0 } \cup A_{ 1 } \cup \ldots \cup A_{ n } }$
           & $A_{ 0 } \cup A_{ 1 } \cup \ldots \cup A_{ n }$ \\
           % Tu może być błąd.
    439 & 19 & & wówczas & więc \\
    440 & 14 & & mierzalny & metryzowalny \\
    440 & & 13 & przecinających & nieprzecinających \\
    440 & & 13 & (zob. str. 438) & (zob. str. 433) \\
    % 441 & & 11 & skończoną & przeliczalną \\ Może to być dobre
    % sformułowanie.
    441 & & 10 & $X$ & $\Zbb$ \\
    447 & & 13 & \textit{odwzorowań} & \textit{odwzorowań ciągłych} \\
    448 & &  7 & twierdzenie 22 & twierdzenie 23 \\
    450 &  7 & & $a$ & $a_{ 0 }$ \\
    450 & 11 & & $B( a_{ k, 1 / n } )$ & $B( a_{ k }, 1 / n )$ \\
    452 & & 13 & $V_{ n, m } = a \in F$ & $V_{ n, m } = a_{ 0 } \in F$ \\
    453 &  6 & & $\mu( A \setminus A_{ n } )$ & $\mu( A \setminus A_{ * } )$ \\
    454 & & 11 & dowolną funkcją & dowolną funkcją $\geq 0$ \\
    455 & & 18 & funkcjami & funkcjami $\geq 0$ \\
    456 & &  1 & $K \subset A$ & $K \subset X$ \\
    457 &  2 & & niezerowej & zerowej \\
    458 & &  8 & twierdzenia 32 & uwagi 1 do twierdzenia 20 \\
    \hline
  \end{tabular}

\end{center}





\begin{center}

  \begin{tabular}{|c|c|c|c|c|}
    \hline
    & \multicolumn{2}{c|}{} & & \\
    Strona & \multicolumn{2}{c|}{Wiersz} & Jest
                              & Powinno być \\ \cline{2-3}
    & Od góry & Od dołu & & \\
    \hline
    459 & &  9 & $\Vcal_{ a_{ i } }$ & $C_{ i }$ \\
    460 & 12 & & $K$ & $K_{ n }$ \\
    460 & &  9 & $\gamma$ & $\vec{ \gamma }$ \\
    462 & &  6 & $\mu( [ a - \epsilon, b + \epsilon [ )$
           & $\mu( ] a - \epsilon, b + \epsilon [ )$ \\
    462 & &  6 & $\mu( ] a - \epsilon, b + \epsilon ] )$
           & $\mu( [ a - \epsilon, b + \epsilon ] )$ \\
    464 & 17 & & $\int \Vert \vec{ f } - \vec{ g } \Vert$
           & $\int^{ * } \Vert \vec{ f } - \vec{ g } \Vert$ \\
    464 & 19 & & $\int \Vert \vec{ f } - \vec{ \gamma } \Vert$
           & $\int^{ * } \Vert \vec{ f } - \vec{ \gamma } \Vert$ \\
    467 &  1 & & kompaktów $K_{ \delta }$ & kompaktów $K_{ i }$ \\
    467 &  1 & & Zbiór $K_{ i }$ & Zbiór $K_{ \delta }$ \\
    467 & 12 & & $\mu( K_{ n } \setminus K_{ \delta }' )$
           & $\mu( K_{ \delta }' \setminus K_{ n } )$ \\
    468 &  9 & & 22 & 23 \\
    469 &  9 & & $\vec{ f }$ & $\int \vec{ f }$ \\
    469 &  9 & & $\int\:\; \Vert \vec{ f } - \vec{ f_{ n } } \Vert$
           & $\int \Vert \vec{ f } - \vec{ f_{ n } } \Vert$ \\
    470 &  9 & & $] 0, 1 ]$ & $[ 0, 1 [$ \\
    474 & 13 & & $\int^{ * } | f - f_{ 0 } |$
           & $\int^{ * } | f - f_{ 0 } | = \int ( f - f_{ 0 } )$ \\
           % 475 & 6 & & Wniosek
           % 4
           % & Wniosek 5 \\
    475 & 13 & &  $\sum_{ n = 0 }^{ \infty }$ & $\sum_{ n = 0 }^{ m }$ \\
    476 & & 15 & $= \sum^{ \infty }_{ n = m + 1 }$
           & $\leq \sum^{ \infty }_{ n = m + 1 }$ \\
    478 &  6 & & wnioskiem 4 & wnioskiem 5 \\
    478 & 14 & & wnioskiem 4 & wnioskiem 5 \\
    480 & 10 & & są zazwyczaj & są \\
    481 & 10 & & $] a, b [$ & $] a, b]$ \\
    482 & 14 & & $| g_{ i } |$ & $g_{ i }$ \\
    483 & &  9 & $\Ccal( X ) \otimes F$ & $\Ccal( X ) \otimes \vec{ F }$ \\
    487 & 10 & & \textit{jest zapewniona} & \textit{nie jest zapewniona} \\
    487 & 13 & & 32 przyjmując $p = 1$. & 32. \\
    487 & 16 & & $\Vert \vec{ f } \Vert^{ p } )$
           & $\Vert \vec{ f } \Vert^{ p }$ \\
    487 & & 10 & twierdzenia 46 & nierówności \\
    492 &  8 & & równoważność & nierównoważność \\
    496 & 17 & & pierwsza & druga \\
    497 & &  4 & zwarty & zawarty \\
    498 & 12 & & domknięcia zwartego & o~domknięciu zwartym \\
    499 & & 12 & $\sigma$-algebrę & $\sigma$-algebrę borelowską \\
    500 & &  3 & $\Ccal_{ + }( X )$ & $\Ccal( X )$ \\
    500 & &  2 & $\Gamma_{ + }( X )$ & $\Gamma( X )$ \\
    500 &  1 & & $\Gamma_{ + }( X )$ & $\Gamma( X )$ \\
    501 &  1 & & $\Ccal_{ + }( X )$ & $\Ccal( X )$ \\
    501 &  2 & & $\Ccal_{ + }( X )$ & $\Ccal( X )$ \\
    501 & & 15 & zbiorów charakterystycznych & charakterystycznych
                                               zbiorów \\
    502 & & 14 & $| \mu |$ & $\mu^{ + }$ \\
    502 & & 13 & zdefiniowaną w sposób ciągły & ciągłą zdefiniowaną \\
    502 & & 10 & $| \mu |$ & $\mu^{ + }$ \\
    504 &  4 & & zbioru & w $X$ zawartych w zbiorze \\ \hline
  \end{tabular}

\end{center}





\begin{center}

  \begin{tabular}{|c|c|c|c|c|}
    \hline
    & \multicolumn{2}{c|}{} & & \\
    Strona & \multicolumn{2}{c|}{Wiersz} & Jest
                              & Powinno być \\ \cline{2-3}
    & Od góry & Od dołu & & \\
    \hline
    505 & &  1 & $a$ & $a_{ \nu }$ \\
    506 &  5 & & wymiernych & niewymiernych \\
    508 & & 18 & $\int$ & $\int^{ * }$ \\
    511 & 18 & & $0, 1, 2, \ldots, m, \ldots$ & $0, 1, 2, \ldots$ \\
    511 & & 16 & o~nośniku & o~domknięciu \\
    511 & &  4 & $\sum\limits_{ n \geq 0 } \int\limits_{ K \cap A_{ k,\, n } }$
           & $\sum\limits_{ k \geq 0 } \int\limits_{ K \cap A_{ k,\, n } }$ \\
    511 & &  4 & $\sum\limits_{ n \geq 0 } \nu( K \cap A_{ k,\, n } )$
           & $\sum\limits_{ k \geq 0 } \nu( K \cap A_{ k,\, n } )$ \\
    511 & &  4 & $\leq \nu( K )$ & $= \nu( K )$ \\
    512 &  1 & & $M_{ n }$ & $M_{ n } \cup N_{ n }$ \\
    512 &  3 & & $M_{ n }$ & $M_{ n } \cup N_{ n }$ \\
    512 & & 16 & $\varphi\, \dPL \mu$ & $\int \varphi\, \dPL \mu$ \\
    512 & & 13 & $\nu = \mu$ & $\nu \leq \mu$ \\
    517 & 15 & & majorantą & majorantą) \\
    522 &  4 & & Tak samo można & Można też \\
    522 & &  3 & $B( \vec{ p }_{ 1 }, g_{ 1 }, f )$
           & $B( \vec{ p }_{ 1 } g_{ 1 }, \vec{ f } )$ \\
    522 & &  1 & $B( \vec{ p }_{ 1 }, g_{ 1 }, \vec{ f } )$
           & $B( \vec{ p }_{ 1 } g_{ 1 }, \vec{ f } )$ \\
    523 &  5 & & dwie & zatem dwie \\
    523 &  6 & & isnieją bowiem & istnieją \\
    524 & 14 & & $d \mu$ & $d \vec{ \mu }$ \\
    524 & &  5 & bowiem & więc \\
    525 &  6 & & lub także & co implikuje \\
    525 & & 11 & $\langle \mu, \varphi \rangle$
           & $\langle u, \varphi \rangle$ \\
    526 & 16 & & 0 & $M$ \\
    526 & &  1 & $\overrightarrow{ u }_{ \dot{ } }$
           & $u_{ \overrightarrow{ \dot{ h } } }$ \\
    527 & &  9 & jest ona przeciwobrazem & posiada ona przeciwobraz \\
    527 & &  9 & będącym & będący \\
    529 & 17 & & ta kula & to otoczenie \\
    530 &  4 & & nośnika & nośnikiem \\
    530 &  4 & & domknięcie & jest domknięcie \\
    530 & 10 & & $H^{ -1 }( A )$ & $H^{ -1 }( K )$ \\
    531 &  3 & & $\vec{ \mu }( H^{ * } \varphi )$ & $H^{ * } \varphi$ \\
    531 &  2 & & funkcji & nośnika funkcji \\
    % 531 & & 16 & zespoloną & rzeczywistą $\geq 0$ \\
    531 & & 10 & Ale & Teraz \\
    531 & & 10 & równa & mniejsza niż \\
    532 &  2 & & \textit{nierówność} & \textit{równość} \\
    532 &  4 & & \textit{nierównością} & \textit{równością} \\
    537 & &  2 & & można \\
    538 &  9 & & $\Vert \vec{ p }_{ n } + \vec{ p } \Vert$
           & $\Vert \vec{ p }_{ n } - \vec{ p }\, \Vert$ \\
    538 & 10 & & $\Vert p_{ n } - p \Vert$
           & $\Vert \vec{ p }_{ n } - \vec{ p }\, \Vert$ \\
    \hline
  \end{tabular}

\end{center}





\begin{center}

  \begin{tabular}{|c|c|c|c|c|}
    \hline
    & \multicolumn{2}{c|}{} & & \\
    Strona & \multicolumn{2}{c|}{Wiersz} & Jest
                              & Powinno być \\ \cline{2-3}
    & Od góry & Od dołu & & \\
    \hline
    538 & & 11 & miary & normy \\
    545 &  6 & & unormowagą & unormowaną \\
    556 & &  1 & zwarte & lokalnie zwarta \\
    559 & &  5 & $\Ccal_{ H \times K }$ & $\Gamma_{ H \times K }$ \\
    563 & 14 & & z~dołu & z~góry \\
    563 & & 10 & twierdzenie & wynik twierdzenia \\
    563 & &  5 & twierdzenia 59 & twierdzenia 45 \\
    563 & &  5 & Z~twierdzenia & Z~twierdzenia tego \\
    564 &  2 & & można & że można \\
    564 &  2 & & i~wtedy & mamy \\
    564 & 14 & & $g_{ i n }$ & $\vec{ g }_{ i n }$ \\
    564 & &  2 & i~dla & dla \\
    566 & &  6 & nie~można & można \\
    566 & &  1 & (~\textit{o} & (\textit{o} \\
    571 & &  5 & \textit{$\mu$-całkowalny}
           & \textit{$\mu$-mierzalny} \\
    571 & &  5 & \textit{$\nu$-całkowalny}
           & \textit{$\nu$-mierzalny} \\
    573 &  3 & & $( \lambda \otimes \mu )$-prawie & $\lambda$-prawie \\
    575 & 10 & & $\vec{ P }_{ l }$ & $P_{ l }$ \\
    575 & 11 & & $d_{ p }$ & $\vec{ d }_{ p }$ \\
    575 & & 17 & sumie wektorowej & domknięciu \\
    576 & &  9 & zwartych & o~domknięciu zwartym \\
    577 & & 10 & $R_{ 1 }$ & $\Rbb_{ 1 }$ \\
    578 & 13 & & funkcją & majoryzowana funkcją \\
    578 & &  6 & $R_{ 1 }$ & $\Rbb_{ 1 }$ \\
    578 & &  4 & Ponieważ & Ponadto \\
    585 & &  9 & \textit{lokalnie} & \textit{lokalnym} \\
    585 & &  3 & $\sum\limits_{ i = 0 }^{ n - 1 } =$
           & $\sum\limits_{ i = 0 }^{ n - 1 }$ \\
    590 & & 18 & Zatem $M$ & $M$ \\
    590 & & 18 & mającą masę & nie mającą masy \\
    591 &  1 & & $\sum\limits_{ \Delta } \Vert \overrightarrow{ B - A } \Vert$
           & $\sum\limits_{ \Delta } = \Vert \overrightarrow{ B - A } \Vert$ \\
    591 & &  6 & dokładnie jedną & jedyną \\
    592 & 11 & & ma & może mieć \\
    593 &  8 & & ($ t $!) & $ t $! \\
    593 & & 15 & \textit{niezerową} & \textit{zerową} \\
    593 & & 10 & $\Rbb_{ 2 }$ & $\Rbb_{ 1 }$ \\
    593 & &  5 & $M_{ [ c_{ i }, } c_{ i + 1 ] }$
           & $M_{ [ c_{ i }, c_{ i + 1 } ] }$ \\
    594 &  2 & & $\Rbb^{ 2 }$ & $\Rbb_{ 1 }$ \\
    594 & &  3 & funkcji & funkcji ciągłej \\
    599 &  7 & & (4.9.38) & (4.8.38) \\
    609 &  3 & & $\int \Vert \vec{ f } \Vert \varphi_{ b } \, d\lambda$
           & $\int^{ * } \Vert \vec{ f } \Vert \varphi_{ b } \, d\lambda$ \\
    609 &  5 & & $\int \Vert \vec{ f } \Vert \varphi_{ b' } \, d\lambda$
           & $\int^{ * } \Vert \vec{ f } \Vert \varphi_{ b' } \, d\lambda$ \\
    609 &  5 & & $\int\limits_{ | a, b' [ } \Vert \vec{ f } \Vert \, d\lambda$
           & $\int\limits_{ | a, b' [ }{^{ * }} \Vert \vec{ f } \Vert \, d\lambda$ \\
    \hline
  \end{tabular}

\end{center}





\begin{center}

  \begin{tabular}{|c|c|c|c|c|}
    \hline
    & \multicolumn{2}{c|}{} & & \\
    Strona & \multicolumn{2}{c|}{Wiersz} & Jest
                              & Powinno być \\ \cline{2-3}
    & Od góry & Od dołu & & \\
    \hline
    612 & 13 & & po prawej stronie & w~powyższym wzorze \\
    617 & &  4 & jest & nie jest \\
    620 & 11 & & $x_{ n }$ & $x_{ n } )$ \\
    620 & 16 & & $t_{ n }$ & $t_{ n } )$ \\
    630 & 11 & & \textit{objętość} & \textit{objętość $V$} \\
    631 & &  8 & $i, j$ & $i < j$ \\
    636 & 11 & & twierdzenia 109 & twierdzenia 119 \\
    638 & &  8 & \textit{$dS$-całkowalnej} & \textit{$dS$-mierzalnej} \\
    642 & 12 & & $v( u_{ 1 }, \ldots, u_{ N - 1 } )$
           & $v = ( u_{ 1 }, \ldots, u_{ N - 1 } )$ \\
    643 & 11 & & $\omega$ & $w$ \\
    643 & & 10 & $\left( = \frac{ 1 }{
                 \partialDerivative{}{ h }{ { x_{ N } } }( x ) } \right)$
           & $= \left( \frac{ 1 }{
             \partialDerivative{}{ h }{ { x_{ N } } }( x ) } \right)$ \\
    648 & & 12 & \textbf{Całkowalność}
           & \textbf{Różniczkowalność} \\
    649 & &  1 & $\overrightarrow{ f_{ m n }( x ) - f_{ m n }( x ) }$
           & $\vec{ f }_{ m n }( x ) - \vec{ f }_{ m n }( x )$ \\
    % 649 & &  9 & \raisebox{\depth}{\scalebox{1}[-1]{\textit{j}}}\textit{ednostajnie}
    %        & \textit{jednostajnie} \\
    654 &  4 & & $1 )$ & $1$ \\
    649 & &  8 & \textit{ewentualnie} & \textit{odpowiednio} \\
    666 &  4 & & wyprowadzony & podany \\
    & & & & \\
    564 &  8 & & $\vec{ E }$.
           & $\vec{ E }$, gdzie $\vec{ x }_{ n } \in B( a_{ n }, \varphi )$. \\
           % Coś tu jest nie tak
    \hline
  \end{tabular}

\end{center}

% \vspace{1em}

\noindent
\StrWd{20}{19} \\
\Jest  Należy zawsze używać $\preceq$ \\
\Powin Należy używać $\preceq$ raczej niż \\
\StrWd{10}{6} \\
\Jest  nie zawierają $n$ elementów \\
\Powin nie zawierają $n$-tego elementu \\
\StrWg{41}{5} \\
\Jest  rozważany zbiór jest skończony, a~kiedy nieskończony \\
\Powin rozważana ilość zbiorów jest skończona, a~kiedy nieskończona \\
\StrWg{34}{9} \\
\Jest  $+\infty$ tak wybraną, że $p / 10^{ b } \leq \eta$ \\
\Powin $+\infty$, zaś $b$ jest liczbą rzeczywistą tak wybraną,
że~$1 / 10^{ b } \leq \eta$ \\
\StrWd{46}{15} \\
\Jest dopełnionej w~początku współrzędnych prostą rzeczywistą \\
\Powin będącej dopełnieniem początku układu współrzędnych do~prostej
rzeczywistej \\
\StrWd{49}{17} \\
\Jest $B_{ 1 }( 1 )$ zawiera także pewien zbiór, otwarty w~sensie
drugiej
metryki, do~którego także należy to zero przestrzeni \\
\Powin ten zbiór jest też względem drugiej metryki i~zawiera zero
przestrzeni \\ % to jakoś źle brzmi
\StrWd{63}{18} \\
\Jest podzbiorów domkniętych \\
\Powin  zstępujących podzbiorów domkniętych \\
\StrWd{63}{17} \\
\Jest Na~mocy twierdzenia~20 \\
\Powin Na~mocy wniosku~1 z~twierdzenia~21 \\
\StrWd{70}{19} \\
\Jest współrzędnych w~przedziale zwartym \\
\Powin  układu współrzędnych do~przedziału zwartego \\
\StrWd{83}{13} \\
\Jest $\alpha' > \alpha$ \\
\Powin  $\alpha' > \alpha$, zawartego w~przedziale $| \alpha, \beta |$ \\
\StrWd{139}{9} \\
\Jest pierwszy wyraz po~prawej stronie we~wzorze \\
\Powin  lewa strona wzoru \\
\StrWg{142}{13} \\
\Jest dochodzimy wprost do~wzoru \\
\Powin  otrzymujemy pierwszy ze~wzorów \\
\StrWg{146}{13} \\
\Jest algebraicznymi \\
\Powin algebraicznymi; w~szczególności oznacza to, że~rzutowania na~te
podprzestrzenie jest~liniowe \\
\StrWd{163}{1} \\
\Jest
$\Vert \vec{ x } + \vec{ y } \Vert = \Vert \vec{ x } \Vert + \Vert \vec{ y } \Vert$
$+ 2 \mathrm{Re}( \vec{ x } | \vec{ y } )$ \\
\Powin $\Vert \vec{ x } + \vec{ y } \Vert^{ 2 } = \Vert \vec{ x } \Vert^{ 2 }
+ \Vert \vec{ y } \Vert^{ 2 }$ $+ 2 \mathrm{Re}( \vec{ x } | \vec{ y } )$ \\
\StrWg{165}{9} \\
\Jest i \\
\Powin i~wraz z~pierwszą częścią dowodu wykazuje, \\
\StrWg{200}{18} \\
\Jest obie te granice nie są równe \\
\Powin  jeśli co~najmniej jedna z~tych granic nie~jest równa \\
\StrWg{245}{8} \\
\Jest funkcja \\
\Powin  funkcja $x \mapsto f'( x )$ jest jednostajnie ciągła \\
\StrWd{259}{12} \\
\Jest \textit{$\Omega \subset E$ w zbiór $F$} \\
\Powin $\Omega \subset E$ przestrzeni afinicznej unormowanej
w~przestrzeń afiniczną unormowaną~$F$ \\
\StrWd{280}{10} \\
\Jest Odwzorowanie to~jest ciągłe \\
\Powin  Wynika stąd, że~odwzorowanie to~jest ciągłe \\
\StrWd{374}{10} \\
\Jest Punkty te~nie muszą \\
\Powin  Punkty te~można umieścić znajdować~się \\
\StrWg{379}{7} \\
\Jest  \textit{Riemanna} (niekoniecznie\ldots zwartym) \\
\Powin \textit{Riemanna} \\
\StrWd{387}{14} \\
\Jest  po~całym tym przedziale \\
\Powin po~wszystkich tych przedziałach \\
\StrWd{390}{2} \\
\Jest Przekonamy~się \\
\Powin  O~jej istnieniu przekonamy~się \\
\StrWd{393}{12} \\
\Jest (4.2.21)~i~biorąc pod~uwagę (3.2.19), \\
\Powin (4.2.21), \\
\StrWd{396}{14} \\
\Jest ciągła na przedziale $[ a', b' ]$ funkcja $\varphi$ \\
\Powin  funkcja $\varphi$ ciągła na przedziale $[ a', b' ]$ \\
\StrWg{399}{7} \\
\Jest Gdy przestrzeń~$X$ jest zwarta, wtedy~dochodzimy tego \\
\Powin  Z~własności zbiorów zwartych wynika \\
\StrWg{407}{5} \\
\Jest pojęcie \\
\Powin  pojęcie i~udowodnimy związane z~nim twierdzenie \\
\StrWg{413}{12} \\
\Jest \textit{przy ciągłym $\vec{ p }$ jest sumą mnogościową domknięcia
  zbioru punktów $a_{ \nu }$ i~nośnika
  funkcji $\vec{ p }$.} \\
\Powin \textit{zawiera~się w~sumie mnogościowej domknięcia zbioru
  punktów $a_{ \nu }$ i~nośnika funkcji $\vec{ p }$, zaś~przy
  $\vec{ p }$
  ciągłym jest ich sumą.} \\
\StrWd{415}{14} \\
\Jest  \textit{lokalnie zwartą\ldots} \\
\Powin \textit{lokalnie zwartą, przeliczalną w~nieskończoności\ldots} \\
\StrWg{436}{17} \\
\Jest $\complement ( [ a, b ] )$ \\
\Powin $\complement ( [ a, b ] )$ i~niech~$D$ będzie wspomnianym wyżej
zbiorem. \\
\StrWd{438}{14} \\
\Jest pseudomierzalny (niekoniecznie o mierze skończonej). \\
\Powin pseudomierzalny. Zbiór ten~ma miarę skończoną ze~względu
na~własność~(4.3.12) i~skończoną miarę kompaktu $K$. \\
\StrWd{439}{4} \\
\Jest \textit{jeżeli wszystkie $\mu( A_{ n } )$~są nieskończone.} \\
\Powin \textit{jeżeli dla~nieskończenie wielu $A_{ n }$ zachodzi
  $\mu( A_{ n } ) = +\infty$.} \\
\StrWg{451}{18} \\
\Jest zbioru~$X$ w~mierzalny, zgodnie z~wnioskiem~1,
zbiór $\vec{ F } \times \vec{ F }$ \\
\Powin zbioru $X$ w~zbiór $\vec{ F } \times \vec{ F }$, mierzalnego
zgodnie
z~wnioskiem~1 \\
\StrWg{467}{4} \\
\Jest w odwzorowaniu $f_{ \delta }$ jest sumą kompaktów $K_{ i }$,
a~zatem jest domknięty, więc~obcięcie $f_{ \delta }$ funkcji $f$
do~$K_{ \delta }$
jest ciągłe na~$K_{ \delta }$. \\
\Powin przez obcięcie $f_{ \delta }$ funkcji $f$ do~$K_{ \delta }$
jest sumą kompaktów $K_{ i }$, a~zatem jest domknięty,
więc~$f_{ \delta }$
jest ciągłe na~$K_{ \delta }$. \\
\StrWd{481}{14} \\
\Jest zbiór otwarty $\mathcal{V}$ kompaktu $K$\ldots \\
\Powin otwarte otoczenie $\mathcal{V}$ kompaktu $K$ o~domknięciu
zwartym\ldots \\
\StrWg{510}{5} \\
\Jest Pierwsza część twierdzenia nie została jednak jeszcze
udowodniona\ldots \\
\Powin  Nie kończy to jednak dowodu pierwszej części twierdzenia\ldots \\
\StrWd{511}{14} \\
\Jest dolna $\mu$\dywiz miara zbioru $M_{ n }$ jest bowiem równa zeru. \\
\Powin  więc miara wewnętrzna zbioru $M_{ n }$ jest równa zeru. \\
\StrWg{529}{16} \\
\Jest kulę zwartą o~środku w~punkcie~$b$ \\
\Powin  otoczenie zwarte punktu~$b$ \\
\StrWd{529}{2} \\
\Jest każda kula o~środku w~punkcie~$b$ \\
\Powin każde otoczenie punktu~$b$
\StrWd{532}{11} \\
\Jest z~nierówności wypukłości\ldots\\
\Powin  z~własności wypukłości i~wartości średniej\ldots \\
\StrWd{532}{18} \\
\Jest \textit{oraz, przy założeniach twierdzenia 72, istniej
  $K( H \mu )$\ldots} \\
\Powin \textit{oraz $H$~i~$K$ spełniają założenia twierdzenia~72,
  tak~że~istniej $K( H \mu )$\ldots}\\
\StrWg{539}{10} \\
\Jest  W~konsekwencji\ldots \\
\Powin Pokażemy, że~w~konsekwencji tego\ldots \\
\StrWd{563}{12} \\
\Jest  obie funkcje~są równe zeru. \\
\Powin funkcja $\psi$ jest równa zeru. \\
\StrWd{563}{11} \\
\Jest jest majoryzowany przez funkcję $f$, równą $f( a )$. \\
\Powin  jest równy $f( a )$.\\
\StrWg{563}{14} \\
\Jest lokalnie zwartymi \\
\Powin  lokalnie zwartymi, przeliczalnymi w~nieskończoności \\
\StrWd{578}{2} \\
\Jest a~więc funkcja regularna\ldots \\
\Powin  więc każda funkcja regularna jest borelowska\ldots \\
\StrWd{586}{4} \\
\Jest funkcją dowolną\ldots \\
\Powin  dowolną funkcją lokalnie $\mu$\dywiz całkowalną\ldots \\
\StrWg{592}{18} \\
\Jest rosnącą i~w~ogólny przypadku silnie rosnącą. \\
\Powin  rosnącą, lecz~w~ogólny przypadku nie silnie rosnącą. \\
\StrWg{593}{6} \\
\Jest sprawa polega na tym, że nawet\ldots \\
\Powin  nawet\ldots \\
\StrWg{605}{8} \\
\Jest $[ a, b ]$ \\
\Powin  $[ a', b' ]$, zawierającym zbiór $\xi( [ \alpha, \beta ] )$ \\
\StrWd{607}{10} \\
\Jest nie miałoby sensu jako całka funkcji ciągłej. \\
\Powin  nie miałoby sensu. \\
\StrWg{610}{10} \\
\Jest całek posługiwaliśmy się\ldots \\
\Powin  niewłaściwej całek posłużymy~się\ldots \\
\StrWg{637}{18} \\
\Jest  \textit{zdefiniowaną}\ldots \\
\Powin \textit{gdzie $D( u )$ jest~zdefiniowane}\ldots \\
\StrWg{646}{9} \\
\Jest  \textit{funkcją pierwotną pola sfery jest $S_{ N } R^{ N - 1 }$,
  znikającą przy $R = 0$}\ldots \\
\Powin \textit{jest to~funkcja pierwotna pola sfery
  $S_{ N } R^{ N - 1 }$,
  znikająca przy $R = 0$}\ldots \\
\StrWg{651}{16} \\
\Jest twierdzenia~13\ldots \\
\Powin twierdzenia~26, rozdział~III\ldots \\

\vspace{\spaceTwo}
% ############################
















\start Str. 195. \ldots iloczyn skalarny
$\langle \cdot \,, \cdot \rangle$\ldots

\start Str. 198. Odwzorowanie $\gamma$ jest więc także iniektywne.

\start Str. 160. \ldots taki, że
$( u\vec{ x } | \vec{ y } )_{ \vec{ F } } = ( \vec{ x } | \vec{ Y }
)_{ \vec{ E } }$\ldots

\start Str. 142. Dowiedliśmy\ldots

\start Str. 142. \ldots dążą do $d^{ 2 }$\ldots

\start Str. 143. \textit{\ldots$F$ jest domknięty niepusty i
  wypukły\ldots}

\start Str. 145. \ldots jest domknięta niepusta i wypukła,\ldots

\start Str. 145. \ldots jest ortogonalny, na mocy ciągłości iloczynu
skalarnego, do całej przestrzeni\ldots

\start Str. 146. Ale także, jak to już pokazaliśmy, również do jej
domknięcia.

\start Str. 146. \ldots\textit{domknięcie ortogonalne przekroju}\ldots

\start Str. 147. \ldots jeżeli
$\vec{ F } = \bigcup_{ i \in I } \vec{ F }_{ i }^{ + }$\ldots

\start Str. 149. Jest ono ciągłe\ldots

\start Str. 73. Niemniej\ldots

\start Str. 76. Jeżeli bowiem $x \in E$\ldots

\start Str. 90. \large{(Riesza($^{ 1 }$))}.

\start Str. 98. \ldots przestrzeń wektorową form liniowych\ldots

\start Str. 102.
$$u(\vec{x}_{1},\vec{x}_{2})=u_{1}(\vec{x}_{1})+u_{2}(\vec{x}_{2})$$
\start Str. 104.
$$\Vert u \Vert = \sup_{ \substack{ \vec{ x } \neq \vec{ 0 } },
  \,\vec{ y } \neq \vec{ 0 } } \frac{ \Vert u( \vec x, \vec y ) \Vert
}{ \Vert \vec{ x } \Vert \, \Vert \vec y \Vert } = \sup_{ \substack{
    \Vert \vec{ x } \Vert \leq 1, \, \Vert \vec{ y } \Vert \leq 1 } }
\Vert u( \vec x, \vec y ) \Vert \, .$$

\start Str. 106. Niech $E_{ 1 }$, $E_{ 2 }$,\ldots,$E_{ n }$, $F$
będą\ldots

\start Str. 107. \ldots poprzedzają twierdzenie 65\ldots

\start Str. 112. Rzeczywiście, zauważmy,\ldots

\start Str. 43. \ldots znajdowałyby się w zewnętrzu $A$; jest to
przeciwne do założenia $a \in \overline{ A }$

\start Str. 58. \ldots układu w $\mathbb{R}^{ 2 }$. Ponadto jest
ona\ldots

\start Str. 60. \ldots z pokrycia $\mathcal{R}$ \ldots

\start Str. 43. \ldots znajdowałyby się w zewnętrzu $A$; jest to
przeciwne do założenia $a \in \overline{ A }$


\start Str. 171. \large{(Riesza)}.
% \start[--] Str. \start[--] Str.










% ##################
\newpage
\Work{ % Autor i tytuł dzieła
  Laurent Schwartz \\
  „Kurs analizy matematycznej. Tom II”,
  \cite{SchwartzKursAnalizyMatematycznejVolII1980} }


% Uwagi:
% \begin{itemize}
%   \start

%   \startize}


% ##################
\CenterBoldFont{Błędy}

\begin{center}

  \begin{tabular}{|c|c|c|c|c|}
    \hline
    & \multicolumn{2}{c|}{} & & \\
    Strona & \multicolumn{2}{c|}{Wiersz} & Jest
                              & Powinno być \\ \cline{2-3}
    & Od góry & Od dołu & & \\
    \hline
    61  & & 13 & $\epsilon_{ \sigma^{ 1 } \sigma }$
           & $\epsilon_{ \sigma^{ -1 } \sigma }$ \\
    65  & &  1 & forma \ldots & forma \\
    66  &  7 & & liczbie${ N \choose p } =  C^{ p }_{ N } $
           & liczbie ${ N \choose p } =  C^{ p }_{ N } $ \\
    \hline
  \end{tabular}

\end{center}

\noindent \\
\start Str. 68.
$$( \vec{ X }_{ 1 }, \vec{ X }_{ 2 }, \ldots, \vec{ X }_{ p })
\mapsto \sum_{\sigma \in \mathfrak{ S } } u_{ 1 }( \vec{ X }_{
  \sigma_{ 1 } } ) u_{ 2 }( \vec{ X }_{ \sigma_{ 2 } } ) \ldots u_{ N
}( \vec{ X }_{ \sigma_{ N } } ) = \det[ u_{ i }( \vec{ X }_{ j } ) ]
\textrm{.}$$

\start Str. 70. Formę stopnia $p$ nazywa się \textit{rozkładalną}\ldots

\start Str. 73. 2$^{\circ}$ Załóżmy przeciwnie\ldots

\start Str. 75. \ldots wyznaczników tych\ldots

\start Str. 75. \ldots jest równy jedynce\ldots

\start Str. 77. \ldots
$\bar{ \zeta }_{ j } = \xi_{ j } - i \eta_{ j }$.

\start Str. 78. \ldots drugiej $\mathbb{R}$\dywiz bazy, odpowiadającej
drugiej $\mathbb{C}$\dywiz bazie, względem\ldots

\start Str. 80. \ldots przeprowadzającej $1, 2, \ldots, N$ w
$k_{ 1 }, \, k_{ 2 }, \ldots, \, k_{ N - p }, \, j_{ 1 }, \, j_{ 2 },
\ldots, \, j_{ p }$.

\start Str. 80. \ldots (3.1.23)\ldots

\start Str. 81. \ldots to
$\alpha_{ \vec{ X }_{ 1 } \vec{ X }_{ 2 } \ldots \vec{ X }_{ N - 1 } }
( \vec{ Y } )$\ldots \start Str. 89. Niech $\Omega$ będzie zbiorem
otwartym przestrzeni af\mbox{}inicznej unormowanej $E$\ldots

\start Str. 91. \ldots co kończy dowód pierwszej części twierdzenia.

\start Str. 92. $$\ldots = \sum_{ \substack{ j_{ 1 } < j_{ 2 } < j_{ 3 } < \ldots < j_{ p } \\
    k_{ 1 } < k_{ 2 } < k_{ 3 } < \ldots < k_{ p } } } \omega_{ j_{ 1
  } j_{ 2 } \ldots j_{ p } } \big( H( x ) \big) \frac{ D( H_{ j_{ 1 }
  }, H_{ j_{ 2 } }, \ldots, H_{ j_{ p } } ) }{ D( x_{ k_{ 1 } }, x_{
    k_{ 2 } }, \ldots, x_{ k_{ p } } ) } \; dx_{ k_{ 1 } } \wedge dx_{
  k_{ 2 } } \wedge \ldots \wedge dx_{ k_{ p } } \textrm{.}$$

\start Str. 95.
$$d_{ \mathcal{ R } } \vec{ \omega }_{ J } = \sum_{ \alpha } dx_{
  \alpha} \wedge \frac{ \partial \vec{ \omega }_{ J } }{ \partial x_{
    \alpha } } \textrm{.}$$

\start Str. 100. \ldots tu pewna pozorna trudność, zresztą bez
znaczenia).

\start Str. 104. \ldots jest dopuszczalne, jest tak jeżeli obszar
$\Omega_{ 0 }$\ldots

\start Str. 107. \ldots\textit{fromą różniczkową zamknietą} (\textit{tzn.
  $d\vec{ \omega } = \vec{ 0 }$})\ldots

\start Str. 113. \ldots że jest ona stała w każdym spójnym otoczeniu
punktu $\alpha$.

\start Str. 114. \ldots\textit{to są one identyczne w całym otoczeniu
  spójnym punktu} $a$\ldots
% \start Str. \start Str. \start Str. \start Str.

\start Str. 135.
$$\ldots \vec{ \mu }_{ \vec{ \omega } \Phi } = \vec{ f }( u_{ 1 },
u_{ 2 }, \ldots, u_{ n } ) \theta( u_{ 1 }, u_{ 2 }, \ldots, u_{ n } )
\; du_{ 1 } \otimes du_{ 2 } \otimes \ldots \otimes du_{ n }
\textrm{,}$$

\start Str. 136.
$$\vec{ \mu }_{ 2 } = \Phi_{ 2 1 } \vec{ \mu }_{ 1 } \ldots$$
% \start Str. \start Str. \start Str. \start Str. \start Str.


\vspace{\spaceTwo}
% ############################







% ####################################################################
% ####################################################################
% Bibliografia
\bibliographystyle{alpha} \bibliography{MathComScienceBooks}{}



% ############################

% Koniec dokumentu
\end{document}
