% Autor: Kamil Ziemian

% --------------------------------------------------------------------
% Podstawowe ustawienia i pakiety
% --------------------------------------------------------------------
\RequirePackage[l2tabu, orthodox]{nag} % Wykrywa przestarzałe i niewłaściwe
% sposoby używania LaTeXa. Więcej jest w l2tabu English version.
\documentclass[a4paper,11pt]{article}
% {rozmiar papieru, rozmiar fontu}[klasa dokumentu]
\usepackage[MeX]{polski} % Polonizacja LaTeXa, bez niej będzie pracował
% w języku angielskim.
\usepackage[utf8]{inputenc} % Włączenie kodowania UTF-8, co daje dostęp
% do polskich znaków.
\usepackage{lmodern} % Wprowadza fonty Latin Modern.
\usepackage[T1]{fontenc} % Potrzebne do używania fontów Latin Modern.



% ----------------------------
% Podstawowe pakiety (niezwiązane z ustawieniami języka)
% ----------------------------
\usepackage{microtype} % Twierdzi, że poprawi rozmiar odstępów w tekście.
\usepackage{graphicx} % Wprowadza bardzo potrzebne komendy do wstawiania
% grafiki.
\usepackage{verbatim} % Poprawia otoczenie VERBATIME.
\usepackage{textcomp} % Dodaje takie symbole jak stopnie Celsiusa,
% wprowadzane bezpośrednio w tekście.
\usepackage{vmargin} % Pozwala na prostą kontrolę rozmiaru marginesów,
% za pomocą komend poniżej. Rozmiar odstępów jest mierzony w calach.
% ----------------------------
% MARGINS
% ----------------------------
\setmarginsrb
{ 0.7in} % left margin
{ 0.6in} % top margin
{ 0.7in} % right margin
{ 0.8in} % bottom margin
{  20pt} % head height
{0.25in} % head sep
{   9pt} % foot height
{ 0.3in} % foot sep



% ------------------------------
% Często przydatne pakiety
% ------------------------------
\usepackage{csquotes} % Pozwala w prosty sposób wstawiać cytaty do tekstu.
\usepackage{xcolor} % Pozwala używać kolorowych czcionek (zapewne dużo
% więcej, ale ja nie potrafię nic o tym powiedzieć).



% ------------------------------
% Pakiety do tekstów z nauk przyrodniczych
% ------------------------------
\let\lll\undefined % Amsmath gryzie się z językiem pakietami do języka
% polskiego, bo oba definiują komendę \lll. Aby rozwiązać ten problem
% oddefiniowuję tę komendę, ale może tym samym pozbywam się dużego Ł.
\usepackage[intlimits]{amsmath} % Podstawowe wsparcie od American
% Mathematical Society (w skrócie AMS)
\usepackage{amsfonts, amssymb, amscd, amsthm} % Dalsze wsparcie od AMS
% \usepackage{siunitx} % Do prostszego pisania jednostek fizycznych
\usepackage{upgreek} % Ładniejsze greckie litery
% Przykładowa składnia: pi = \uppi
\usepackage{slashed} % Pozwala w prosty sposób pisać slash Feynmana.
\usepackage{calrsfs} % Zmienia czcionkę kaligraficzną w \mathcal
% na ładniejszą. Może w innych miejscach robi to samo, ale o tym nic
% nie wiem.



% ##########
% Tworzenie otoczeń "Twierdzenie", "Definicja", "Lemat", etc.
\newtheorem{twr}{Twierdzenie} % Komenda wprowadzająca otoczenie
% ,,twr'' do pisania twierdzeń matematycznych
\newtheorem{defin}{Definicja} % Analogicznie jak powyżej
\newtheorem{wni}{Wniosek}



% ----------------------------
% Pakiety napisane przez użytkownika.
% Mają być w tym samym katalogu to ten plik .tex
% ----------------------------
% \usepackage{reedsimon} % Pakiet napisany konkretnie dla tego pliku.
\usepackage{latexshortcuts}
\usepackage{mathshortcuts}



% --------------------------------------------------------------------
% Dodatkowe ustawienia dla języka polskiego
% --------------------------------------------------------------------
\renewcommand{\thesection}{\arabic{section}.}
% Kropki po numerach rozdziału (polski zwyczaj topograficzny)
\renewcommand{\thesubsection}{\thesection\arabic{subsection}}
% Brak kropki po numerach podrozdziału



% ----------------------------
% Ustawienia różnych parametrów tekstu
% ----------------------------
\renewcommand{\arraystretch}{1.2} % Ustawienie szerokości odstępów między
% wierszami w tabelach.



% ----------------------------
% Pakiet "hyperref"
% Polecano by umieszczać go na końcu preambuły.
% ----------------------------
\usepackage{hyperref} % Pozwala tworzyć hiperlinki i zamienia odwołania
% do bibliografii na hiperlinki.








% ####################################################################
\begin{document}
% ####################################################################



% ##############################
\Main{Historia Polski} % Tytuł całego tekstu

\vspace{\spaceTwo} \vspace{\spaceThree}
% ##############################


% ####################
\Work{ % Autor i tytuł dzieła
  Stanisław Łojasiewicz \\
  ,,Wstęp do~teorii funkcji rzeczywistych'',
  \cite{LojasiewiczWstepDoFunkcjiRzeczywistych76} }

\CenterTB{Uwagi}

% \tb{Konkretne strony.}

% \vspace{\spaceFour}

% \start \Str{6} Stwierdzenie, że~wszystkie funkcje pierwotne
% różnią~się o~stałą, można zinterpretować w~błędny, acz bez~ustanku
% powtarzany, sposób. Dwie funkcje pierwotne różnią~się o~stałą
% dowolną na każdym przedziale spójnym, jednak jeśli dziedzina funkcji
% składa~się z~dwóch lub więcej przedziałów spójnych, na każdym z~nich
% można wybrać inną stałą dowolną.

\CenterTB{Błędy}
\begin{center}
  \begin{tabular}{|c|c|c|c|c|}
    \hline
    & \multicolumn{2}{c|}{} & & \\
    Strona & \multicolumn{2}{c|}{Wiersz}& Jest
                              & Powinno być \\ \cline{2-3}
    & Od góry & Od dołu & & \\ \hline
    9  & 15 & & $\abso{ \al }$ $\abso{ \be }$
           & $\abso{ \al }$, $\abso{ \be }$ \\
           % & & & & \\
           % & & & & \\
           % & & & & \\
           % & & & & \\
           % & & & & \\
    \hline
  \end{tabular}
\end{center}

\vspace{\spaceTwo}

% \vspace{\spaceFour}



% ####################
\Work{ % Autor i tytuł dzieła
  Grigorij Michajłowicz Fichtenholz \\
  ,,Rachunek różniczkowy i~całkowy. Tom~I'',
  \cite{FichtenholzRachunekRiCTomI05} }


% Uwagi:
% \begin{itemize}
% \item
% \item
% \item
% \item
% \end{itemize}


Powinno być:
\begin{itemize}
\item[--] Str. 303.
$$\sum_{ i = 1 }^{ n } a_{ i }^{2} \sum_{ i = 1 }^{ n } b_{ i }^{ 2 } - \{ \sum_{ i = 1 }^{ n } a_{ i } b_{ i } \}^{ 2 } \geq 0 \textrm{,}$$
\item[--] Str. 371.
  $$z'_{ x } = \frac{ x }{ p } \textrm{,} \quad z'_{ y } = \frac{ y }{
    q } \textrm{.}$$
\item[--] Str. 376. \ldots ale równa jest 0, gdy\ldots
\item[--] Str.
\end{itemize}

\vspace{\spaceTwo}





% ####################
\Work{ % Autor i tytuł dzieła
  G. M. Fichtenholz \\
  ,,Rachunek różniczkowy i całkowy. Tom~II'',
  \cite{FichtenholzRachunekRiCTomII04} }


\CenterTB{Uwagi}

% \tb{Konkretne strony.}

% \vspace{\spaceFour}

\start \Str{6} Stwierdzenie, że~wszystkie funkcje pierwotne różnią~się
o~stałą, można zinterpretować w~błędny, acz bez~ustanku powtarzany,
sposób. Dwie funkcje pierwotne różnią~się o~stałą dowolną na każdym
przedziale spójnym, jednak jeśli dziedzina funkcji składa~się z~dwóch
lub więcej przedziałów spójnych, na każdym z~nich można wybrać inną
stałą dowolną.

\vspace{\spaceFour}


\start \Str{34} Nie jestem w~stanie zrozumieć, dlaczego największym
wspólnym dzielnikiem wielomianów $Q$ i~$Q'$ jest $Q_{ 1 }$. Może to
jakiś znany fakt z~algebry? \Dok

\CenterTB{Błędy}
\begin{center}
  \begin{tabular}{|c|c|c|c|c|}
    \hline
    & \multicolumn{2}{c|}{} & & \\
    Strona & \multicolumn{2}{c|}{Wiersz}& Jest
                              & Powinno być \\ \cline{2-3}
    & Od góry & Od dołu &  &  \\ \hline
    8   & & 11 & $< | \Del P | <$ & $\leq | \Del P | \leq$ \\
    8   & &  7 & $< \fr{ | \Del P | }{ \Del x } <$
           & $\leq \frac{ | \Del P | }{ \Del x } \leq$ \\
    12  & &  5 & $a^{ n }$ & $a_{ n }$ \\
    13  &  7 & & $( x - a )^{ k }\, dx$ & $\int ( x - a )^{ -k }\, dx$ \\
    18  & &  4 & $\fr{ 1 }{ 3 }$ & $\fr{ 1 }{ 2 }$ \\
    18  & &  1 & więc$\cdot t$ & więc $t$ \\
    23  & 16 & & $n \_ 1$ & $n - 1$ \\
    23  & 17 & & $n \_ 2$ & $n - 2$ \\
    23  & & 16 & otrzvmujemy & otrzymujemy \\
    25  &  1 & & $e^{ \cdot (k + 1) t}$ & $e^{ (k + 1) t}$ \\
    28  & &  5 & z$\cdot$algebry & z~algebry \\
    29  & & 11 & $\fr{ P(x) }{ (x - a)^{k - 1} Q_{ 1 }(x) }$
           & $\fr{ P_{ 1 }(x) }{ (x - a)^{k - 1} Q_{ 1 }(x) }$ \\
    34  &  3 & & [lub (6) & [lub (6)] \\
    34  &  4 & & lub (6)] & [lub (6)] \\
    35  & & 17 & $\left[ \fr{ a x^{ 2 } + b x + c }
                 { x^{ 3 } + x^{ 2 } + x + 1 } \right]$
           & $\left[ \fr{ a x^{ 2 } + b x + c }
             { x^{ 3 } + x^{ 2 } + x + 1 } \right]'$ \\
    35  & & 12 & $x^{ 2 } + x^{ 2 } + x + 1$
           & $x^{ 3 } + x^{ 2 } + x + 1$ \\
    36  & 14 & & $x^{ \dot{ 2 } }$ & $x^{ 2 }$ \\
    39  &  3 & & $\sqrt[ m ]{ \fr{ \al x + \be }{ \ga x + \del } } dx$
           & $\sqrt[ m ]{ \fr{ \al x + \be }{ \ga x + \del } }$ \\
    40  &  2 & & $\fr{ 2t - 1 }{ \sqrt{ 3 } }$
           & $\fr{ 2t + 1 }{ \sqrt{ 3 } }$ \\
    40  & & 12 & ${ m + 1 \atop n }$ & $\fr{ m + 1 }{ n }$ \\
    44  & &  4 & $\sqrt{ ax }$ & $\sqrt{ a } x$ \\ \hline
  \end{tabular}

  \begin{tabular}{|c|c|c|c|c|}
    \hline
    & \multicolumn{2}{c|}{} & & \\
    Strona & \multicolumn{2}{c|}{Wiersz}& Jest
                              & Powinno być \\ \cline{2-3}
    & Od góry & Od dołu &  &  \\ \hline
    44  & &  1 & $\sqrt{ ax }$ & $\sqrt{ a } x$ \\
    45  & & 13 & $2 \sqrt{ a } t - b$ & $2 \sqrt{ c } t - b$ \\
    46  & &  8 & $( 2 ax + b^{ 2 } )$ & $( 2 ax + b )^{ 2 }$ \\
    % 479 & 2 & & $\IntL_{ 0 }^{ +\infty }$
    % & $\IntL_{ a }^{ +\infty }$ \\
    % 479 & 4 & & $\IntL_{ a }^{ A } \fr{ dx }{ x }$
    % & $\IntL_{ a }^{ A } \fr{ dx }{ x^{ \la } }$ \\
    % 483 & 4 & & $I^{ 3 }$ & $I^{ 2 }$ \\
    % & & & & \\
    \hline
  \end{tabular}
\end{center}
\noi \\
\StrWd{34}{3}
\tb{Jest:} Zróżniczkujmy (10) obustronnie\ld \\
\tb{Powinno być:} Wykonując jawnie różniczkowanie w~(10)\ld \\
\StrWd{71}{16} \Jest
$R\left( x, \sqrt{ ax^{ 4 } + bx^{ 3 } + cx^{ 2 } + dx + e }
\right.$ \\
\Pow $R\left( x, \sqrt{ ax^{ 4 } + bx^{ 3 } + cx^{ 2 } + dx + e }
\right)$ \\

\vspace{\spaceTwo}





% ####################
\Work{
  Włodzimierz Krysicki, L. Włodarski \\
  ,,Analiza matematyczna w zadaniach. Tom~I'', \cite{KW05} }


\CenterTB{Uwagi}

\start \Str{89} Problemy takie jak policzenie granicy
$\Lim_{ \xToZero } \fr{ \sin 3 x }{ x }$, sugerują następujące
postępowanie. Wiemy, że~$\Lim_{ \xToZero } 3x = 0$ oraz
że~$\Lim_{ \xToZero } \fr{ \sin x }{ x } = 1$, chcielibyśmy
przekształcić wyrażenie do postaci $3 \fr{ \sin 3x }{ 3x }$
i~skorzystać z~tego, że~$\Lim_{ \xToZero } \fr{ \sin 3x }{ 3x } = 0$,
pytanie jedna czy ta ostania równość zachodzi? Pozytywną odpowiedź
na~to~pytanie daje poniższe twierdzenie.

\vspace{\spaceFour}

% Funkcje tu użyte wciąż nie są zdefiniowane \start \Str{96} Można
% zauważyć, że~funkcje $\artanh$ i~$\arcoth$ mają taką samą pochodną,
% więc powinny różnią~się o~stałą, co~jest nonsensem. Wyjaśnieniem
% problemu jest fakt, że~te dwie funkcje~są określone na rozłącznych
% dziedzinach.

\vspace{\spaceFour}


\start \Str{125} Warto przedyskutować, dlaczego tak ważnej jest
założenie o~tym, że~$\dd{}{ x }{ t } \neq 0$. Jeżeli mamy dane dwie
funkcje $y( t )$ i~$x( t )$, to~nie musi istnieć funkcja $y( x )$.
Jeżeli jednak dla jakiegoś $t_{ 0 }$ mamy
$\dd{}{ x }{ t }( t_{ 0 } ) \neq 0$ to~na mocy twierdzenie
o~odwracaniu funkcji klasy $\Cj$ istnieje\footnote{Nie wiem czy
  założenie o~klasie różniczkowalności jest konieczne, bowiem
  w~przypadku funkcji rzeczywistej jednej zmiennej istnieje wiele
  wariantów twierdzenia o~funkcji uwikłanej i~funkcji odwrotnej, więc
  może~się okazać, iż~jeden z~nich pozwala osłabić ten warunek.
  Dyskusję tych twierdzeń można znaleźć w~książce Fichtenholza
  \cite{Fic05a}.} funkcja $t( x )$ w~pewnym otoczeniu $t_{ 0 }$
i~w~tym otoczeniu $y( x ) = y( t( x ) )$.

Co~jednak dzieje~się, gdy~$\dd{}{ x }{ t }( t_{ 0 } ) = 0$, czy
funkcja $y( x )$ wówczas nie istnieje? Następujący przypadek pokazuje,
że~tak nie musi być. Rozpatrzmy funkcje $x( t ) = t^{ 3 }$,
$y( t ) = t$. Pomimo, że~$\dd{}{ x }{ t }( 0 ) = 0$~to, można to
zauważyć np. rysując wykresy obu funkcji, można odwikłać $t( x )$
i~wówczas
\begin{equation*}
  y( x ) =
  \begin{cases}
    \sqrt[1 / 3]{ x } & x \geq 0, \\
    -\sqrt[1 / 3]{ x } & x < 0.
  \end{cases}
\end{equation*}
Funkcja ta jednak nie jest różniczkowalna dla $x = 0$. Nie potrafię
stwierdzić, czy znikanie pochodnej $\dd{}{ x }{ t }( t_{ 0 } )$ musi
pociągać za sobą, że~jeśli funkcja $y( x )$ istnieje to jest
nieróżniczkowalna po $x$ w~punkcie $x( t_{ 0 } )$. Wątpię jednak,
aby~tak było.

\vspace{\spaceFour}


\start \Str{130} Autorzy popełnili tu błąd przyjmując, że~moduł
pochodnej równy jest pochodnej modułu:
\begin{displaymath}
  \left| \dd{}{ f }{ t } \right| = \dd{}{ \abso{ f } }{ t }.
\end{displaymath}
Że~jest inaczej można~się przekonać rozważając znany przykłady ruchu
jednostajnego po okręgu, gdzie prędkość ma stałą długość, a~jednak
moduł przyśpieszenia nie jest równy 0, lecz $\fr{ v^{ 2 } }{ \rho }$,
gdzie $\rho$ to promień krzywizny.

Przeprowadzając proste obliczenia dostajemy poprawne wzory na składowe
i~moduł przyśpieszenia:
\begin{align*}
  a_{ x } &= 50 ( \cos 5t^{ 2 } - 100 t^{ 2 } \sin 5t^{ 2 } ), \\
  a_{ y } &= -50 ( \sin 5t^{ 2 } + 100 t^{ 2 } \cos 5t^{ 2 } ), \\
  a &= 50 \sqrt{ 1 + 100 t^{ 4 } }.
\end{align*}
Widzimy więc, że~moduł rośnie w~czasie. Należało~się tego spodziewać,
bowiem przyśpieszenie normalne do toru dane jest wzorem
$\fr{ v^{ 2 } }{ \rho }$, więc jeśli prędkość rośnie liniowo, to ta
składowa przyśpieszenia również musi rosnąć.

% Uwagi:
% \begin{itemize}
% \item
% \item
% \item
% \item
% \end{itemize}

\CenterTB{Błędy}
\begin{center}
  \begin{tabular}{|c|c|c|c|c|}
    \hline
    & \multicolumn{2}{c|}{} & & \\
    Strona & \multicolumn{2}{c|}{Wiersz}& Jest & Powinno być \\ \cline{2-3}
    & Od góry & Od dołu &  &  \\ \hline
    46  &  8 & & $\sqrt[n]{ u_{ n } } < p$ & $\sqrt[n]{ u_{ n } } \leq p$ \\
    79  & &  8 & $-b \, a$ & $-b / a$ \\
    80  & 10 & & $= 2$ & $x = 2$ \\
    88  & &  2 & $\fr{ x^{ 2 } - 1 }{ x - 2 }$ & $\fr{ x^{ 2 } - 4 }
                                                 { x - 2 }$ \\
    89  &  5 & & $\fr{ ( x - 3 )( -1 )^{ [ x ] } }{ x^{ 2 } - 9 }$
           & $\fr{ ( x + 3 )( -1 )^{ [ x ] } }{ x^{ 2 } - 9 }$ \\
    96  & &  3 & $\fr{ -1 }{ 1 - x^{ 2 } }$
           & $\fr{ 1 }{ 1 - x^{ 2 } }$ \\
    116 &  2 & & $\left[ 2 \tan \fr{ x }{ 3 } + 1 \right)$
           & $ \left( 2 \tan \fr{ x }{ 3 } + 1 \right)$ \\
    129 & &  5 & $a = 796$ & $a = 7.96$ \\
    231 & 11 & & $\bigg| \Sum_{ k = 0 }^{ n } u( x ) - S( x ) < \bigg|
                 \veps$
           & $\bigg| \Sum_{ k = 0 }^{ n } u( x )
             - S( x ) \bigg| < \veps$ \\
    234 & & 11 & $\sin{ 1 \atop n }$ & $\sin \fr{ 1 }{ n }$ \\
    241 & &  6 & $f( 0 )$ & $f'( 0 )$ \\
    241 & &  4 & $2^{ 3 }$ & $2^{ 2 }$ \\
    256 & &  4 & $\fr{ f'( x ) }{ {}'( x ) }$ & $\fr{ f'( x ) }
                                                { g'( x ) }$ \\
    258 & &  4 & ${ 1 \atop t^{ 2 } }$ & $\fr{ 1 }{ t^{ 2 } }$ \\
    259 &  8 & & oraz \emph{istnieje} & \emph{ale istnieje} \\
    441 & &  7 & $\arctan^{ 3 }x$ & $\arctan x^{ 3 }$ \\
    442 & &  6 & $\fr{ 1 }{ ( 1 - x ) \sqrt{ x } }$
           & $\fr{ 1 }{ x \ln x \ln( \ln x ) }$ \\
    436 & &  6 & $-4$ & $-2$ \\
    % & & & & \\
    % & & & & \\
    % & & & & \\
    \hline
  \end{tabular}
\end{center}

\begin{itemize}

\item[--] Str. 325. 16.32.
  $\int \frac{ 6 x - 13 }{ x^{ 2 } - \frac{ 7 }{ 2 } x + \frac{ 3 }{ 2
    } } dx$.

\item[--] Str. 378. 19.15.
  $\int \limits_{ 0 }^{ a } 3x \sqrt{ x^{ 2 } + 4 a^{ 2 } } dx, a > 0
  \textrm{.}$

\item[--] Str. 436. 5.38. $\frac{ 2 }{ 3 }$.

\item[--] Str. 438. 6.50.
  $y' = 7 x^{ 4 / 3 } - 13 x^{ 9 / 4 } - \frac{ 2 }{ 7 } x^{ -3 / 2 }
  \textrm{.}$

\item[--] Str. 438. 6.53.
  $y' = x^{ -2 / 3 } - 3 x^{ 2 } + \frac{ 1 }{ 2 } \frac{ 1 }{ \sqrt[
    4 ]{ x } }$.

\item[--] Str. 438. 6.56. \ldots
  $y' = \frac{ -5 }{ 7 \sqrt[ 7 ]{ x^{ 8 } } } - 14 x^{ 6 } - \frac{ 3
  }{ 4 \sqrt{ x^{ 3 } } }$.

\item[--] Str. 438. 6.89. \ldots
  $z' = \frac{ -2 a x }{ ( a^{ 2 } + x^{ 2 } ) \sqrt{ a^{ 4 } - x^{ 4
      } } }$.

\item[--] Str. 440. 6.113. $\cos x \neq 0$,
  $y' = \frac{ 7 \sin^{ 3 } x }{ \cos^{ 8 } x }$.

\item[--] Str. 441. 6.129. $x > 1$,
  $y' = \frac{ x \ln x }{ \sqrt{ ( x^{ 2 } - 1 )^{ 3 } } }$.

\item[--] Str. 441. 6.131.
  $y' = x^{ 4 } \arctan x + \frac{ x^{ 5 } - x }{ 5 ( 1 + x^{ 2 } ) }
  - \frac{ 1 }{ 5 } x^{ 3 } + \frac{ 1 }{ 5 } x$.
\item[--] Str. 478. \ldots
  $I = \frac{ 1 }{ 3 } \ln | a^{ 3 } + x^{ 3 } |$.

\item[--] Str. 480. 16.26. $I = \frac{ 1 }{ 8 } ( 2 x + 1 )^{ 4 }$.

\item[--] Str. 480. 16.27. $x \neq \frac{ 2 }{ 3 }$;
  $I = \frac{ -1 }{ 9 ( 3 x - 2 )^{ 3 } }$.

\item[--] Str. 480. 16.37. $x \neq \frac{ 2 }{ 3 }$,
  $x \neq \frac{ 3 }{ 2 }$;
  $I = \frac{ 1 }{ 5 } \ln | \frac{ 2 x - 3 }{ 3 x - 2 } |$.

\end{itemize}

\vspace{\spaceTwo}





% ####################
\Work{
  Włodzimierz Krysicki, L. Włodarski \\
  ,,Analiza matematyczna w~zadaniach. Tom~II'', \cite{KW04} }


\CenterTB{Uwagi}

\start \Str{199} \tb{Zadanie 7.2.} Rozwiązanie tego zadania jest
niepełne, w~następujący sensie. Równanie numer (2) w~tym zadaniu,
po~spierwiastkowaniu sprowadza~się do~równania:
\begin{equation*}
  \left| \dd{}{ y }{ x } \right| = \abso{ \cos x }.
\end{equation*}
Równania tego nie da~się przedstawić w~postaci Newtona, zaś poza
równaniami (3) z~tego zadania, można uzyskać z~niego nieskończoną
ilość innych, np. ograniczając~się do przedziału
$( -\fr{ \pi }{ 2 }, \frac{ 3 \pi }{ 2 })$ możemy rozpatrzyć:
\begin{equation*}
  \dd{}{ y }{ x } =
  \begin{cases}
    -\cos x, & x \in ( -\fr{ \pi }{ 2 }, \fr{ \pi }{ 2 } ), \\
    \cos x, & x \in ( \fr{ \pi }{ 2 }, \fr{ 3 \pi }{ 2 } ).
  \end{cases}
\end{equation*}
Dla dowolnych warunków początkowych w~rozpatrywanym przedziale, można
znaleźć rozwiązania tego równania różniczkowalne w~całym przedziale.
Na przykład dla $x_{ 0 } = \fr{ \pi }{ 2 }, y_{ 0 } = 0$, takim
rozwiązaniem jest:
\begin{equation*}
  y( x )
  \begin{cases}
    -\sin x + 1,  & x \in ( -\fr{ \pi }{ 2 }, \fr{ \pi }{ 2 } ), \\
    \sin x - 1, & x \in ( \fr{ \pi }{ 2 }, \fr{ 3 \pi }{ 2 } ).
  \end{cases}
\end{equation*}

\vspace{\spaceFour}


\start \Str{200} \tb{Zadanie 7.3.} Również tutaj problem znalezienia
wszystkich rozwiązań równania różniczkowego nie został w~pełni
rozwiązany. Ograniczając~się do przedziału $( 0, 2 \pi )$ możemy
zauważyć, że~funkcja:
\begin{equation*}
  y( x ) =
  \begin{cases}
    C_{ 1 } \sin x,& x \in ( 0, \pi ), \\
    0, & x = \pi, \\
    C_{ 2 } \sin x,& x \in ( \pi, 2 \pi ),
  \end{cases}
\end{equation*}
jest rozwiązaniem równania (1) z~tego zadania w~całym badanym
przedziale, choć nie jest różniczkowalne w~0.


\CenterTB{Błędy}
\begin{center}
  \begin{tabular}{|c|c|c|c|c|}
    \hline
    & \multicolumn{2}{c|}{} & & \\
    Strona & \multicolumn{2}{c|}{Wiersz}& Jest & Powinno być \\ \cline{2-3}
    & Od góry & Od dołu &  &  \\ \hline
    25  & &  7 & $( \sin z )^{ \tan z }$ & $( \sin x )^{ \tan z }$ \\
    26  & 10 & & $3 \cdot 3$ & $9$ \\
    26  & &  4 & $|\, \mr{n}$ & $\ln$ \\
    26  &  2 & & $e^{ x^{ 2 } \sin(x - y^{ 2 }) }$
           & $\sin^{ x^{ 2 } }(x - y^{ 2 })$ \\
    26  &  1 & & $e^{ x^{ 2 } \sin(x - y^{ 2 }) }$
           & $\sin^{ x^{ 2 } }(x - y^{ 2 })$ \\
           % & & & & \\
           % & & & & \\
    201 & &  1 & $+$ & $=$ \\
    203 &  7 & & $\fr{ a y }{ d x }$ & $\dd{}{ y }{ x }$ \\
    203 &  7 & & sprawdzają & spełniają \\
    207 &  2 & & $1 \fr{ 1 + \tfr{ 1 }{ C_{ 2 }^{ 2 } } }
                 { x^{ 2 } + 1 }$
           & $1 - \fr{ 1 + \fr{ 1 }{ C_{ 2 }^{ 2 } } }
             { x^{ 2 } + 1 }$ \\
             % & & & & \\
    431 & &  3 & $x^{ \fr{ 1 }{ y - 1 } }$ & $x^{ \fr{ 1 }{ y } - 1 }$ \\
    432 & &  9 & $\fr{\uppi R^{ 3 } }{ 3 }$ & $\fr{\uppi R^{ 2 } }{ 3 }$ \\
    449 & &  9 & $C - e^{ \fr{ 1 }{ x } }$ & $C e^{ -\fr{ 1 }{ x } }$ \\
    449 & &  9 & $c$ & $C$ \\
    % & & & & \\
    \hline
  \end{tabular}
\end{center}

\begin{itemize}
\item[--] Str. 432.
  $\fr{ \pr^{ 2 } u }{ \pr x \pr y } = \fr{ \pr^{ 2 } u }{ \pr y \pr x
  } = 6 x^{ 2 } - 30 x y^{ 2 } - 2 \sin 2y$.

\item[--] Str. 454. 9.5. $y = C \frac{ x - 2 }{ x + 2 }$.

\item[--] Str. 454. 9.5. $y = C x$.

  % \item[--] Str.
  % \item[--] Str.
  % \item[--] Str.
  % \item[--] Str.
  % \item[--] Str.
\end{itemize}

\vspace{\spaceTwo}





% ####################
\Work{ % Autor i tytuł dzieła
  P. J. Nahin \\
  ,,Inside Interesting Integrals'', \cite{Nahin15} }



% \noindent
% Uwagi:
% \begin{itemize}
% \item[] \tb{Konkretne strony.}
% \item[--] \Str{6} Stwierdzenie, że~wszystkie funkcje pierwotne
%   różnią~się o~stałą, można zinterpretować w~błędny, acz bez~ustanku
%   powtarzany, sposób. Dwie funkcje pierwotne różnią~się o~stałą
%   dowolną na każdym przedziale spójnym, jednak jeśli dziedzina
%   funkcji składa~się z~dwóch lub więcej przedziałów spójnych, na
%   każdym z~nich można wybrać inną stałą dowolną.
% \item[--] \Str{34} Nie jestem w~stanie zrozumieć, dlaczego
%   największym wspólnym dzielnikiem wielomianów $Q$ i~$Q'$ jest
%   $Q_{ 1 }$. Może to jakiś znany fakt z~algebry? \Dok
% \end{itemize}

\CenterTB{Błędy}
\begin{center}
  \begin{tabular}{|c|c|c|c|c|}
    \hline
    & \multicolumn{2}{c|}{} & & \\
    Strona & \multicolumn{2}{c|}{Wiersz}& Jest & Powinno być \\ \cline{2-3}
    & Od góry & Od dołu &  &  \\ \hline
    % & & & & \\
    % & & & & \\
    \hline
  \end{tabular}
\end{center}
\noi
\StrWg{vi}{3} \\
\Jest \tb{immediatly know they have encountered a~most interesting
  character}: \\
\Pow \tb{\emph{immediatly know they have encountered a~most
    interesting
    character}:} \\
\newpage





% ####################
\Work{
  B. W. Szabat \\
  ,,Wstęp do analizy zespolonej'', \cite{Szabat74} }


\CenterTB{Błędy}
\begin{center}
  \begin{tabular}{|c|c|c|c|c|}
    \hline
    & \multicolumn{2}{c|}{} & & \\
    Strona & \multicolumn{2}{c|}{Wiersz}& Jest & Powinno być \\ \cline{2-3}
    & Od góry & Od dołu &  &  \\ \hline
    17 & 5 & & dal & dal\dywiz \\
    25 & & 11 & $z_{ 2 }( t )$ & $z_{ 2 }( \tau( t ) )$ \\
    % & & & & \\
    \hline
  \end{tabular}
\end{center}

\vspace{\spaceTwo}





% ####################
\Work{
  W. Żakowski, W. Lesiński \\
  ,,Matematyka. Część IV'' % \cite{ZL78}
}


\CenterTB{Uwagi}


\noi \tb{Część~I.}

\vspace{\spaceFour}


\start Często w~tej części książki pojawia~się następująca sytuacja.
W~wyniku rachunków otrzymaliśmy całkę ogólną pewnego równania
różniczkowego $\vp( x, C ) = 0$, przy czym
$C \in ( a, b ) \sum ( b, c )$. W~każdym rozważanym przypadku
okazywało~się, że~funkcję $\vp( x, C )$ można w~naturalny sposób
przedłużyć do~$C = b$ i~$\vp( x, b ) = $ również jest rozwiązaniem
tego równania. Czy to jest zbieg okoliczności, czy~też przy pewnych
warunkach musi to zachodzić? \Prze

\vspace{\spaceThree}



\noi \tb{Konkretne strony.}

\vspace{\spaceFour}


\start \Str{23} \Dok

\start \Str{26} Rachunki prowadzące do równania (I.62) dowodzą,
że~przedstawia ono rozwiązanie wyjściowego równania różniczkowego
dla~każdej stałej $C_{ 2 }$ różnej od~zera. Aby~zasadnie twierdzić,
że~jest to rozwiązanie tego równania dla~każdej stałej rzeczywistej
$C_{ 2 }$ należy podstawić\footnote{W~tym momencie nie znam prostszego
  rozwiązania tego problemu, ale~w~uwagach do części~I tej książki,
  jest zawarta sugestia innego podejścia.} $C_{ 2 } = 0$, co prowadzi
do~$u = 1 \pm \sqrt{2}$, i~sprawdzić, że~otrzymana w~ten sposób
funkcja jest rozwiązaniem zadanego równania. Jak~się okazuje, tak
w~istocie jest.

\CenterTB{Błędy}
\begin{center}
  \begin{tabular}{|c|c|c|c|c|}
    \hline
    & \multicolumn{2}{c|}{} & & \\
    Strona & \multicolumn{2}{c|}{Wiersz}& Jest & Powinno być \\ \cline{2-3}
    & Od góry & Od dołu &  &  \\ \hline
    % & & & & \\
    % & & & & \\
    % & & & & \\
    \hline
  \end{tabular}
\end{center}





% ####################################################################
% ####################################################################
\bibliographystyle{alpha} \bibliography{Bibliography}{}



\end{document}
