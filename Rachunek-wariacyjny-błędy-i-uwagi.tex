% ---------------------------------------------------------------------
% Podstawowe ustawienia i pakiety
% ---------------------------------------------------------------------
\RequirePackage[l2tabu, orthodox]{nag}  % Wykrywa przestarzałe i niewłaściwe
% sposoby używania LaTeXa. Więcej jest w l2tabu English version.
\documentclass[a4paper,11pt]{article}
% {rozmiar papieru, rozmiar fontu}[klasa dokumentu]
\usepackage[MeX]{polski}  % Polonizacja LaTeXa, bez niej będzie pracował
% w języku angielskim.
\usepackage[utf8]{inputenc} % Włączenie kodowania UTF-8, co daje dostęp
% do polskich znaków.
\usepackage{lmodern}  % Wprowadza fonty Latin Modern.
\usepackage[T1]{fontenc}  % Potrzebne do używania fontów Latin Modern.



% ------------------------------
% Podstawowe pakiety (niezwiązane z ustawieniami języka)
% ------------------------------
\usepackage{microtype}  % Twierdzi, że poprawi rozmiar odstępów w tekście.
% \usepackage{graphicx}  % Wprowadza bardzo potrzebne komendy do wstawiania
% grafiki.
\usepackage{vmargin}  % Pozwala na prostą kontrolę rozmiaru marginesów,
% za pomocą komend poniżej. Rozmiar odstępów jest mierzony w calach.
% ------------------------------
% MARGINS
% ------------------------------
\setmarginsrb
{ 0.7in}  % left margin
{ 0.6in}  % top margin
{ 0.7in}  % right margin
{ 0.8in}  % bottom margin
{  20pt}  % head height
{0.25in}  % head sep
{   9pt}  % foot height
{ 0.3in}  % foot sep



% ------------------------------
% Często używane pakiety
% ------------------------------
% \usepackage{csquotes}  % Pozwala w prosty sposób wstawiać cytaty do tekstu.
\usepackage{xcolor}  % Pozwala używać kolorowych czcionek (zapewne dużo
% więcej, ale ja nie potrafię nic o tym powiedzieć).



% ------------------------------
% Pakiety do tekstów z nauk przyrodniczych
% ------------------------------
\let\lll\undefined  % Amsmath gryzie się z pakietami do języka
% polskiego, bo oba definiują komendę \lll. Aby rozwiązać ten problem
% oddefiniowuję tę komendę, ale może tym samym pozbywam się dużego Ł.
\usepackage[intlimits]{amsmath}  % Podstawowe wsparcie od American
% Mathematical Society (w skrócie AMS)
\usepackage{amsfonts, amssymb, amscd, amsthm}  % Dalsze wsparcie od AMS
\usepackage{upgreek}  % Ładniejsze greckie litery
% Przykładowa składnia: pi = \uppi
\usepackage{calrsfs}  % Zmienia czcionkę kaligraficzną w \mathcal
% na ładniejszą. Może w innych miejscach robi to samo, ale o tym nic
% nie wiem.



% ------------------------------
% Wspaniały pakiet PGF/TikZ
% ------------------------------
% \usepackage{tikz}

% \usetikzlibrary{decorations.markings}  % Włączenie konkretnych bibliotek
% % pakietu TikZ



% ---------------
% Tworzenie otoczeń "Twierdzenie", "Definicja", "Lemat", etc.
% ---------------
\newtheorem{theorem}{Twierdzenie}  % Komenda wprowadzająca otoczenie
% „theorem” do pisania twierdzeń matematycznych
\newtheorem{definition}{Definicja}  % Analogicznie jak powyżej
\newtheorem{corollary}{Wniosek}



% ------------------------------
% Pakiety których pliki *.sty mają być w tym samym katalogu co ten plik
% ------------------------------
\usepackage{latexgeneralcommands}
\usepackage{mathcommands}
% \usepackage{calculuscommands}




% ---------------------------------------------------------------------
% Dodatkowe ustawienia dla języka polskiego
% ---------------------------------------------------------------------
\renewcommand{\thesection}{\arabic{section}.}
% Kropki po numerach rozdziału (polski zwyczaj topograficzny)
\renewcommand{\thesubsection}{\thesection\arabic{subsection}}
% Brak kropki po numerach podrozdziału



% ------------------------------
% Ustawienia różnych parametrów tekstu
% ------------------------------
\renewcommand{\arraystretch}{1.2}  % Ustawienie szerokości odstępów między
% wierszami w tabelach



% ------------------------------
% Pakiet „hyperref”
% Polecano by umieszczać go na końcu preambuły
% ------------------------------
\usepackage{hyperref}  % Pozwala tworzyć hiperlinki i zamienia odwołania
% do bibliografii na hiperlinki










% ---------------------------------------------------------------------
% Tytuł i autor tekstu
\title{Rachunek wariacyjny \\
  Błędy i~uwagi}

\author{Kamil Ziemian}


% \date{}
% ---------------------------------------------------------------------










% ####################################################################
\begin{document}
% ####################################################################





% ######################################
\maketitle  % Tytuł całego tekstu
% ######################################





% % ######################################
% \section{XIX szkoła analizy matematycznej}

% \vspace{\spaceTwo}
% % ######################################





% ############################
\Work{ % Autorzy i tytuł dzieła
  I.M.~Gelfand, S.W.~Fomin \\
  „Rachunek wariacyjny”, \cite{GelfandFominRachunekWariacyjny1972}}


% ##################
\CenterBoldFont{Uwagi do konkretnych stron}


\start \Str{10} W~tym miejscu pierwszy raz autorzy książki używają pewnego
skrótu myślowego. Zapisują oni bowiem funkcjonał długości
krzywej\footnote{Dokładniej, chodzi o~klasę krzywych które da się zapisać za
  pomocą zależności typu $x \mapsto ( x, y( x ) )$, gdzie $y( x )$ jest normalną
  funkcją zmiennej~$x$. Krzywą która nie jest tego typu, jest choćby okrąg
  o~środku w~punkcie $( 0, 0 )$ i~promieniu jeden. Dla punktu tego okręgu
  o~współrzędnych $( x, y )$ spełnione jest równanie
  $x^{ 2 } + y^{ 2 } = 1$. Przy czym nie zakładamy, że~$y$ jest funkcją $x$,
  ani odwrotnie.} pewnej klasy krzywych na płaszczyźnie jako
\begin{equation}
  \label{eq:Gelfand-Fomin-Year1972-01}
  \int_{ a }^{ b } \sqrt{ 1 + y'^{ \, 2 } } \,\: dx,
\end{equation}
pomijając jawną zależność funkcji $y'( x )$ od zmiennej niezależnej~$x$.
Bardziej precyzyjny zapis tego funkcjonału miałby postać
\begin{equation}
  \label{eq:Gelfand-Fomin-Year1972-02}
  \int_{ a }^{ b } \sqrt{ 1 + \big( y'( x ) \big)^{ 2 } } \,\: dx.
\end{equation}

Innym przykładem tego skrótu myślowego, jest pojawiający~się też na tej
stronie zapis innego funkcjonału
\begin{equation}
  \label{eq:Gelfand-Fomin-Year1972-03}
  \int_{ a }^{ b } F\left( x, y, y' \right) dx,
\end{equation}
który w~bardziej precyzyjnej notacji zapisalibyśmy jako
\begin{equation}
  \label{eq:Gelfand-Fomin-Year1972-04}
  \int_{ a }^{ b } F\left( x, y( x ), y'( x ) \right) dx.
\end{equation}

Dla większej przejrzystości prowadzonych rozważań, w~dalszym ciągu tych
notatek będziemy~się starali możliwie jawnie zapisywać zależność wszystkich
funkcji od~ich argumentów, w~sytuacjach gdy wykonujemy operacje takie
jak całkowanie po~tych zmiennych.

\vspace{\spaceFour}





\start \Str{15} W~definicji liniowego funkcjonału nad liniową przestrzenią
unormowaną~$R$ nie pojawia się warunek, by był on jednorodny, czyli
\begin{equation}
  \label{eq:Gelfand-Fomin-Year1972-05}
  \varphi( \alpha \, h ) = \alpha \, \varphi( h ), \quad \alpha \in \Cbb, \, h \in R.
\end{equation}
Nie wiem, czy jest to przeczenie ze~strony autorów, czy też addytywność
i~ciągłość takiego funkcjonału od razu gwarantuje jego jednorodność.

\vspace{\spaceFour}





\start \Str{15} W~tym miejscu autorzy wprowadzają jedno z~kluczowych dla
rachunku wariacyjnego zagadnień. Mianowicie, przyjmijmy, iż~dana jest pewna
przestrzeń funkcji~$H$. Jeśli dla funkcji $f( x )$ zachodzi
\begin{equation}
  \label{eq:Gelfand-Fomin-Year1972-06}
  \int_{ x_{ 1 } }^{ x_{ 2 } } f( x ) h( x ) \, dx = 0,
\end{equation}
dla każdego $h( x ) \in H$, to co możemy o~niej na podstawie tego powiedzieć?

????
Z braku czasu mogę tylko zaznaczyć kilka problemów jakie mogą się w~związku
z~tym pojawić. Jaka całka Riemanna/Lebesgue’a/jakaś inna występuje w
powyższym wzorze? Co musimy założyć o funkcji $f( x )$ by odpowiednie całki
istniały? Dla jakich przestrzeni $H$ chcielibyśmy udowodnić twierdzenia,
która dowodzą, np. że funkcja $f( x )$ jest tożsamościowo równa 0?

????
Te i inne pytanie musimy na razie odłożyć na bok, licząc, iż przyjdą kiedyś
lepsze czasy, gdy będziemy się mogli nimi zająć.

\vspace{\spaceFour}





\start \Str{17} W~tym miejscu potrzebny nam jest rozkład funkcji ciągłej
$f : [ a, b ] \to \Rbb$ (lub $f : [ a, b ] \to \Cbb$), postaci
\begin{equation}
  \label{eq:Gelfand-Fomin-Year1972-07}
  f( x ) = \lambda( x ) + \alpha,
\end{equation}
gdzie $\alpha$ jest stałą, zaś $\lambda( x )$ jest funkcją posiadającą własność
\begin{equation}
  \label{eq:Gelfand-Fomin-Year1972-08}
  \int_{ a }^{ b } \lambda( x ) \, dx = 0.
\end{equation}
Taki rozkład da się bardzo łatwo podać. Weźmy mianowicie
\begin{subequations}
  \begin{align}
    \label{eq:Gelfand-Fomin-Year1972-09-A}
    &\alpha = \frac{ 1 }{ \abs{ b - a } } \, \int_{ a }^{ b } f( x ) dx, \\
    \label{eq:Gelfand-Fomin-Year1972-09-B}
    &\lambda( x ) = f( x ) - \alpha.
  \end{align}
\end{subequations}
Widzimy więc, że~wprowadzenie funkcji $\lambda( x )$ sprowadza~się do sprytnej
zmiany oznaczeń. Co nie zmienia faktu, że~jest to bardzo użyteczny trik.

\vspace{\spaceFour}





% ##################
\newpage

\CenterBoldFont{Błędy}


\begin{center}

  \begin{tabular}{|c|c|c|c|c|}
    \hline
    & \multicolumn{2}{c|}{} & & \\
    Strona & \multicolumn{2}{c|}{Wiersz} & Jest
                              & Powinno być \\ \cline{2-3}
    & Od góry & Od dołu & & \\
    \hline
    % & & & & \\
    % & & & & \\
    % & & & & \\
    % & & & & \\
    % & & & & \\
    % & & & & \\
    % & & & & \\
    \hline
  \end{tabular}

\end{center}

\vspace{\spaceTwo}


% ############################










% ############################
\Work{ % Autorzy i tytuł dzieła
  I.M.~Gelfand, S.W.~Fomin \\
  „Rachunek wariacyjny”, \cite{GelfandFominRachunekWariacyjny1979}}


% ##################
\CenterBoldFont{Błędy}


\begin{center}

  \begin{tabular}{|c|c|c|c|c|}
    \hline
    & \multicolumn{2}{c|}{} & & \\
    Strona & \multicolumn{2}{c|}{Wiersz} & Jest
                              & Powinno być \\ \cline{2-3}
    & Od góry & Od dołu & & \\
    \hline
    25  & &  1 & $\lim\limits_{ \Delta x \to 0 }
                 \frac{ \Delta y' }{ \Delta x } = \tilde{ F }_{ y' y' }$
           & $\lim\limits_{ \Delta x \to 0 }
             \frac{ \Delta y' }{ \Delta x } \tilde{ F }_{ y' y' }$ \\
    27  &  7 & & $\frac{ d }{ dx }( F - y' F_{ y' } )$
           & $\frac{ d }{ dx }( F - y' F_{ y' } ) = 0$ \\
           % & & & & \\
    \hline
  \end{tabular}

\end{center}

\vspace{\spaceTwo}



% ############################







% % ##################
% \CenterBoldFont{Błędy}


% \begin{center}

%   \begin{tabular}{|c|c|c|c|c|}
%     \hline
%     & \multicolumn{2}{c|}{} & & \\
%     Strona & \multicolumn{2}{c|}{Wiersz} & Jest
%                               & Powinno być \\ \cline{2-3}
%     & Od góry & Od dołu & & \\
%     \hline
%     % & & & & \\
%     % & & & & \\
%     \hline
%   \end{tabular}





%   % \begin{tabular}{|c|c|c|c|c|}
%   %   \hline
%   %   & \multicolumn{2}{c|}{} & & \\
%   %   Strona & \multicolumn{2}{c|}{Wiersz} & Jest
%   %                             & Powinno być \\ \cline{2-3}
%   %   & Od góry & Od dołu & & \\
%   %   \hline
%   %   & & & & \\
%   %   \hline
%   % \end{tabular}

% \end{center}


% \noindent
% \StrWd{}{} \\
% \Jest   \\
% \Powin  \\


% \vspace{\spaceTwo}
% % ############################






















% ####################################################################
% ####################################################################
% Bibliografia
\bibliographystyle{plalpha}

\bibliography{MathComScienceBooks}{}





% ############################

% Koniec dokumentu
\end{document}
