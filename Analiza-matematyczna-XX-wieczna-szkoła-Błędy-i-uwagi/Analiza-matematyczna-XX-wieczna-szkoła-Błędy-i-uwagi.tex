% ------------------------------------------------------------------------------------------------------------------
% Basic configuration and packages
% ------------------------------------------------------------------------------------------------------------------
% Package for discovering wrong and outdated usage of LaTeX.
% More information to be found in l2tabu English version.
\RequirePackage[l2tabu, orthodox]{nag}
% Class of LaTeX document: {size of paper, size of font}[document class]
\documentclass[a4paper,11pt]{article}



% ------------------------------------------------------
% Packages not tied to particular normal language
% ------------------------------------------------------
% This package should improved spaces in the text
\usepackage{microtype}
% Add few important symbols, like text Celcius degree
\usepackage{textcomp}



% ------------------------------------------------------
% Polonization of LaTeX document
% ------------------------------------------------------
% Basic polonization of the text
\usepackage[MeX]{polski}
% Switching on UTF-8 encoding
\usepackage[utf8]{inputenc}
% Adding font Latin Modern
\usepackage{lmodern}
% Package is need for fonts Latin Modern
\usepackage[T1]{fontenc}



% ------------------------------------------------------
% Setting margins
% ------------------------------------------------------
\usepackage[a4paper, total={14cm, 25cm}]{geometry}



% ------------------------------------------------------
% Setting vertical spaces in the text
% ------------------------------------------------------
% Setting space between lines
\renewcommand{\baselinestretch}{1.1}

% Setting space between lines in tables
\renewcommand{\arraystretch}{1.4}



% ------------------------------------------------------
% Packages for scientific papers
% ------------------------------------------------------
% Switching off \lll symbol, that I guess is representing letter "Ł"
% It collide with `amsmath' package's command with the same name
\let\lll\undefined
% Basic package from American Mathematical Society (AMS)
\usepackage[intlimits]{amsmath}
% Equations are numbered separately in every section
\numberwithin{equation}{section}

% Other very useful packages from AMS
\usepackage{amsfonts}
\usepackage{amssymb}
\usepackage{amscd}
\usepackage{amsthm}

% Package with better looking calligraphy fonts
\usepackage{calrsfs}

% Package with better looking greek letters
% Example of use: pi -> \uppi
\usepackage{upgreek}
% Improving look of lambda letter
\let\oldlambda\Lambda
\renewcommand{\lambda}{\uplambda}




% ------------------------------------------------------
% BibLaTeX
% ------------------------------------------------------
% Package biblatex, with biber as its backend, allow us to handle
% bibliography entries that use Unicode symbols outside ASCII
\usepackage[
language=polish,
backend=biber,
style=alphabetic,
url=false,
eprint=true,
]{biblatex}

\addbibresource{Analiza-matematyczna-wspolczesne-podejscie-Bibliography.bib}





% ------------------------------------------------------
% Defining new environments (?)
% ------------------------------------------------------
% Defining enviroment "Wniosek"
\newtheorem{corollary}{Wniosek}
\newtheorem{definition}{Definicja}
\newtheorem{theorem}{Twierdzenie}





% ------------------------------------------------------
% Private packages
% You need to put them in the same directory as .tex file
% ------------------------------------------------------
% Package containing various command useful for working with a text
\usepackage{general-commands}
% Package containing commands and other code useful for working with
% mathematical text
\usepackage{math-commands}

% Package containing commands created for writing about Bratteli, Robinson
% book "Operator algebras and quantum statistical mechanics"
% \usepackage{bratteli-robinson-operator-algebras-ETC}





% ------------------------------------------------------
% Package "hyperref"
% They advised to put it on the end of preambule
% ------------------------------------------------------
% It allows you to use hyperlinks in the text
\usepackage{hyperref}










% ------------------------------------------------------------------------------------------------------------------
% Title and author of the text
\title{Analiza matematyczna, współczesne podejście \\
  {\Large Błędy i~uwagi}}

\author{Kamil Ziemian}


% \date{}
% ------------------------------------------------------------------------------------------------------------------










% ####################################################################
\begin{document}
% ####################################################################





% ######################################
\maketitle
% ######################################






% ######################################
\section{Krzysztof Maurin \textit{Analiza. Część~I:
    Elementy}, \parencite{Maurin-Analiza-Vol-I-Pub-1974}}

\vspace{0em}


% ##################
\CenterBoldFont{Uwagi}

\vspace{0em}


\noindent
Tyle błędów, że szybciej będzie napisać tę książkę od nowa.

\VerSpaceFour





\noindent
Paragraf poświęcony całce Riemanna, jest napisany nie
najlepiej. Nie udowodniono wszystkich ważnych twierdzeń, nawet tych
z~których korzysta. Ponadto dowody są w dużej mierze przeprowadzone
bardzo pobieżnie i niewiele mówią czytelnikowi.

\VerSpaceFour





\noindent
Kwestia analityczności funkcji została tu~potraktowana
pobieżnie. Dla funkcji takich jak $\cos$, $\sin$ etc. pokazano ich
analityczność w~0, nic nie powiedziano o analityczności w innych
punktach.

\VerSpaceFour





\noindent
W~ogóle nie przedyskutowano problemu mnożenia szeregów.

\VerSpaceFour





\noindent
Większość z definicji i twierdzeń Rozdziału VII da się uogólnić
dla przypadku przestrzeni unormowanych.

\VerSpaceFour






% ##################
\CenterBoldFont{Uwagi do~konkretnych stron}


\noindent
\Str{10} Wszystkie dowody „nie wprost”, są przypisane
zasadzie \\
kontrapozycji.

\VerSpaceFour





\noindent
\Str{21} Użyte jest pojęcie subtelniejszego podziału, pomimo że~nie zostało
ono zdefiniowane.

\VerSpaceFour





\noindent
\Str{40} Twierdzenie tu zapowiedziane nie zostało nigdy udowodnione.

\VerSpaceFour





\noindent
\Str{41} Uwaga napisana jest fatalnie. Nawet jeżeli to twierdzenie jest
prawdziwe, co nie jest oczywiste, to sformułowanie, zaciemnia cały problem.

\VerSpaceFour





\noindent
\Str{43} W drugiej części dowodu Wniosku II.5 zupełnie niepotrzebnie
wprowadzono kule otwarte zawarte w $Z_{ 1 }$ i~$Z_{ 2 }$.

\VerSpaceFour





\noindent
\Str{48} Dowód pierwszej części Twierdzenia II. 18 jest zupełnie
niezrozumiały.

\VerSpaceFour





\noindent
\Str{53} Czy w twierdzeniu o pochodnej funkcji odwrotnej jest ważna
ciągłość tej funkcji?

\VerSpaceFour





\noindent
\Str{56} Dyskusja skierowania w rodzinie zbiorów $\Pi$,
bardziej zaciemnia niż wyjaśnia. Po przeanalizowaniu okazuje się, że w
definicji $\limsup$ powinno być: \\
$A_{ i } \prec A_{ j } \Leftrightarrow A_{ j } \subset A_{ i }$, choć z
dyskusji wynika wręcz przeciwnie.

\VerSpaceFour





\noindent
\Str{57} Nie zdefiniowano na jakie odcinki dzielimy przedział $[ a, b ]$.
Okazuje się, że nie ma to znaczenia jeśli przyjmiemy miarę dowolnego
odcinka $\mu ( \{ a, b \} ) = b - a$, jednak ta lekkomyślność jest rażąca.

\VerSpaceFour





\noindent
\Str{63} W~dowodzie Twierdzenia.III.13 „potrzeba i nie potrzeba” założenia
$\varphi' \neq 0$, zależnie od rozpatrywanego przypadku.

\VerSpaceFour





\noindent
\Str{66} Skorzystano tu z twierdzenia dla liczb rzeczywistych, które
zostało udowodnione tylko dla liczb całkowitych.

\VerSpaceFour





\noindent
\Str{68} Dowód korzysta z pojęcia bazy otoczeń domkniętych, które nie
zostało nigdzie wcześniej przedstawione.

\VerSpaceFour





\noindent
\Str{90} Podane są warunki na to by ciąg $T_{ n }$ był ciągiem Cauchy’ego,
nie zaś by było on zbieżny do $T_{ 0 }$.

\VerSpaceFour





\noindent
\Str{90} W Lemat V.5. zawiera poważne pomieszanie pojęć. Prawidłowe
sformułowanie powinno brzmieć: ????

\VerSpaceFour





\noindent
\Str{92} W dowodzie Lematu V.6 użyte są kule otwarte a powinny być
domknięte.

\VerSpaceFour





\noindent
\Str{93} Dowód testu Weierstrassa dla funkcji jest bez sensu. Dowodzi tylko
punktowej zbieżności szeregu funkcji. Zbieżności jednostajnej należy
dowodzić inną metodą.

\VerSpaceFour





\noindent
\Str{97} Dowód punktu (b) uwagi łatwo przeprowadzić przez kontrapozycje,
nie widzę jednak sposobu udowodnienia go w sposób analogiczny do punktu
(a). W dowodzie tym kluczowe jest oszacowanie
\mbox{$| a_{ n } ( z_{ 1 } - z_{ 0 } )^{ n } | < M$.} Jednak dla szeregu
o~niezerowym promieniu zbieżności nie być możliwe oszacowanie szeregu od
góry albo od dołu. Za przykład może posłużyć zachowanie szeregów
$\sum \frac{ 1 }{ n } z^{ n }$ i~$\sum n z^{ n }$ w~$z = 1$. Oba są tam
rozbieżne

\VerSpaceFour





\noindent
\Str{98} Dowód twierdzenia Cauchy’ego-Hadamarda zawiera w sobie dowód
kryterium Cauchy’ego zbieżności szeregów. Czyni to dowód bardzo zagmatwanym
i~trudnym do zrozumienia.





% ##################
\newpage

\CenterBoldFont{Błędy}


\begin{center}

  \begin{tabular}{|c|c|c|c|c|}
    \hline
    Strona & \multicolumn{2}{c|}{Wiersz} & Jest
                              & Powinno być \\ \cline{2-3}
    & Od góry & Od dołu & & \\
    \hline
    & & & & \\
    \hline
  \end{tabular}

\end{center}

\VerSpaceTwo


\noindent
Str. 23. \ldots 2$^{ \circ }$ identytywna\ldots \\
Str. 28. \textit{Ad} 2$^{ \circ }$: $| ( a'_{ n } + b'_{ n } )
- ( a_{ n } + b_{ n } ) |  \leq | a'_{ n } - a_{ n } | + | b'_{ n } - b_{ n } |
< 2 \, \varepsilon$, bo \\
$\big( ( a'_{ n } ) \sim ( a_{ n } ) \wedge( b'_{ n } ) \sim ( b_{ n } ) \big)
\Rightarrow \big( ( | a'_{ n } - a_{ n } | < \epsilon )
\wedge | b'_{ n } - b_{ n } | < \epsilon, \textrm{ dla } n > N ) \big)$. \\
Str. 32. \\
Str. 43. \ldots wynika z definicji odwzorowania ciągłego. \\
Str. 54. Niech $g'( x ) h + r_{ 1 }( h ) \neq 0$\ldots \\
Str. 97. $t := \frac{ | z - z_{ 0 } | }{ | z_{ 1 } - z_{ 0 } | }
\textrm{\ldots}$ \\
Str. 112.
$\big( 0 \leq f \leq C \frac{ 1 }{ x^{ \mu } } \textrm{ ; } \mu \geq 1 \big)
\Rightarrow \big( \int \limits^{ \infty }_{ a } f < \infty \big) \textrm{;}$ \\
Str. 112.
$\big( 0 \leq f \leq C \frac{ 1 }{ x ( \log x )^{ \mu } } \textrm{ ; } \mu \geq 1 \big)
\Rightarrow \big( \int \limits^{ \infty }_{ a } f < \infty \big) \textrm{;}$ \\
Str. 125. \ldots \\
Str. 143. \ldots korzystając ze wzorów (4) i (5)\ldots



% ############################










% ############################
\newpage

\section{Krzysztof Maurin \textit{Analiza. Część II: Ogólne
    struktury, funkcje algebraiczne, całkowanie, analiza
    tensorowa}, \cite{}}

\vspace{0em}


% ##################
\CenterBoldFont{Uwagi do konkretnych stron}

\vspace{0em}


\noindent
\Str{90} W~równaniu
\begin{equation}
  \label{eq:MaurinAnalizaOgolneStrukturyVolII-01}
  \rho( v )^{ 2 } = Q( v ) \cdot 1,
\end{equation}
Symbol „1” po prawej stronie oznacza jedynkę w algebrze $\Ccal( V )$.
W~tekście nie zostało to w ogóle zaznaczone.





% ##################
\newpage

\CenterBoldFont{Błędy}

\begin{center}

  \begin{tabular}{|c|c|c|c|c|}
    \hline
    Strona & \multicolumn{2}{c|}{Wiersz} & Jest
                              & Powinno być \\ \cline{2-3}
    & Od góry & Od dołu & & \\
    \hline
    24  & 17 & & $\exists_{ \substack{ W \in \Bcal( x ) } }$
           & $\exists_{ \substack{ W \in \Bcal( y ) } }$ \\
    26  & 16 & & $\Leftarrow:$ & $\Leftarrow$ (a.a.). \\
    90  & & 17 & nieskończenie & niekoniecznie \\
    90  & & \hphantom{0}8 & $V = ( V, B )$ & $V = ( V, Q )$ \\
    90  & & \hphantom{0}8 & algebra & algebra z jedynką \\
    92  & & 17 & $2Q( e_{ i }, e_{ j } )$ & $B( e_{ i }, e_{ j } )$ \\
    99  & \hphantom{0}3 & & $K( a )$ & $\mathbf{K}( a )$ \\
    % & & & & \\
    % & & & & \\
    % & & & & \\
    % & & & & \\
    \hline
  \end{tabular}

\end{center}

\VerSpaceTwo



% ############################










% ############################
\newpage

\section{Walter Rudin \textit{Analiza rzeczywista
    i~zespolona},
  \cite{RudinAnalizaRzeczywistaIZespolona1998}}

\vspace{0em}


% ##################
\CenterBoldFont{Uwagi do konkretnych stron}

\vspace{0em}


\noindent
\Str{132} Należy zauważyć, że wspomniana tu „miara rzeczywista” ma oznaczać
miarę zespoloną (czyli miarę której dziedzina zawiera się w ciele liczb
zespolonych, bez nieskończoności) która przyjmuje tylko wartości
rzeczywiste.





% ##################
\newpage

\CenterBoldFont{Błędy}


\begin{center}

  \begin{tabular}{|c|c|c|c|c|}
    \hline
    Strona & \multicolumn{2}{c|}{Wiersz} & Jest
                              & Powinno być \\ \cline{2-3}
    & Od góry & Od dołu & & \\
    \hline
    27  & & \hphantom{0}3 & fnkcji & funkcji \\
    50  & \hphantom{0}4 &  & Warunek (d) & Warunek (e) \\
    50  & 13 &  & że$K \prec f  \prec V$ & że $K \prec f  \prec V$ \\
    54  & & 16 & II i IV & II i VI \\
    54  & & 15 & że więc & więc \\
    64  & & \hphantom{0}2 & $c_{ i } \chi_{ E_{ i } }$ & $c_{ i } \chi_{ V_{ i } }$ \\
    95  & \hphantom{0}5 & & $| \varphi - x_{ n } |^{ 2 } \leq | \varphi |^{ 2 }$
           & $| \varphi - \hat{ x }_{ n } |^{ 2 } \leq | \varphi |^{ 2 }$ \\
    95  & \hphantom{0}8 & & $\Vert \hat{ x }_{ n } - x_{ m } \Vert_{ 2 }$
           & $\Vert \hat{ x }_{ n } - \hat{ x }_{ m } \Vert_{ 2 }$ \\
    99  & & \hphantom{0}7 & $\Vert f - P \Vert_{ 2 } < \infty$
           & $\Vert f - P \Vert_{ 2 } < \varepsilon$ \\
    132 & & 15 & miary rzeczywistej & miary zespolonej rzeczywistej \\
    140 & & 11 & Ponieważ $\lambda$ & Ponieważ $\Phi$ \\
    168 & & \hphantom{0}5 & $\{ E_{ i } )$ & $\{ E_{ i } \}$ \\
    177 & & \hphantom{0}5 & $= \int\limits^{ \infty }_{ -\infty } g( t ) \, dt \ldots$
           & $= \int\limits^{ x }_{ -\infty } g( t ) \, dt \ldots$ \\
           % & & & & \\
    \hline
  \end{tabular}

\end{center}

\VerSpaceTwo


\noindent
Str. 51. $\mu( K ) = \inf\{ \Lambda f : K \prec f \}$. \\
Str. 52. $\mu( K ) = \Lambda f$. \\
Str. 52. \ldots co w zestawieniu z nierównością (9) daje (8). \\
Str. 165. \ldots dla \textit{każdego} ciągu $\{ E_{ i } \}$\ldots


% ############################










% #############################
\newpage

\section{ % Autorzy i tytuł dzieła
  W. Żakowski, W. Lesiński \\
  \textit{Matematyka. Część IV},
  \cite{ZakowskiLeksinskiLMatematykaVolIV1978}}

\vspace{0em}


% ##################
\CenterBoldFont{Uwagi}

\vspace{0em}


\noindent
\textbf{Część~I.}

% \vspace{\spaceFour}


\noindent
Często w~tej części książki pojawia~się następująca sytuacja.
W~wyniku rachunków otrzymaliśmy całkę ogólną pewnego równania
różniczkowego $\varphi( x, C ) = 0$, przy czym
$C \in ( a, b ) \sum ( b, c )$. W~każdym rozważanym przypadku
okazywało~się, że~funkcję $\varphi( x, C )$ można w~naturalny sposób
przedłużyć do~$C = b$ i~$\varphi( x, b ) = $ również jest rozwiązaniem
tego równania. Czy to jest zbieg okoliczności, czy~też przy pewnych
warunkach musi to zachodzić? ???

\VerSpaceThree





% ##################
\CenterBoldFont{Uwagi do konkretnych stron}

\vspace{0em}


\noindent
\Str{23} ???

\VerSpaceFour





\noindent
\Str{26} Rachunki prowadzące do równania (I.62) dowodzą,
że~przedstawia ono rozwiązanie wyjściowego równania różniczkowego
dla~każdej stałej $C_{ 2 }$ różnej od~zera. Aby~zasadnie twierdzić,
że~jest to rozwiązanie tego równania dla~każdej stałej rzeczywistej
$C_{ 2 }$ należy podstawić\footnote{W~tym momencie nie znam prostszego
  rozwiązania tego problemu, ale~w~uwagach do części~I tej książki,
  jest zawarta sugestia innego podejścia.} $C_{ 2 } = 0$, co prowadzi
do~$u = 1 \pm \sqrt{2}$, i~sprawdzić, że~otrzymana w~ten sposób
funkcja jest rozwiązaniem zadanego równania. Jak~się okazuje, tak
w~istocie jest.





% ##################
\newpage

\CenterBoldFont{Błędy}


\begin{center}

  \begin{tabular}{|c|c|c|c|c|}
    \hline
    Strona & \multicolumn{2}{c|}{Wiersz} & Jest
                              & Powinno być \\ \cline{2-3}
    & Od góry & Od dołu & & \\
    \hline
    %     & & & & \\
    %     & & & & \\
    %     & & & & \\
    \hline
  \end{tabular}

\end{center}

\VerSpaceTwo



% ############################





















% ######################################
\newpage

\section{Teoria funkcji rzeczywistych}

\VerSpaceTwo
% ######################################



% ############################
\section{ % Autor i tytuł dzieła
  Stanisław Łojasiewicz \\
  \textit{Wstęp do~teorii funkcji rzeczywistych},
  \cite{LojasiewiczWstepDoTeoriiFunkcjiRzeczywistych1976}}

\vspace{0em}


% ##################
\CenterBoldFont{Uwagi do konkretnych stron}

\vspace{0em}


\noindent
\Str{7} Dla pełniejszego wykładu, warto tu przytoczyć pozostałe
operacje z~udziałem $\pm \infty$.
\begin{equation}
  \label{eq:Lojasiewicz-Wstep-do-teorii-ETC-01}
  \begin{split}
    &a + ( \pm \infty ) = ( \pm \infty ) + a = \pm \infty, \\
    &a \cdot ( \pm \infty ) = ( \pm \infty ) \cdot a = \pm \infty,
    \quad a > 0, \\
    &a \cdot ( \pm \infty ) = ( \pm \infty ) \cdot a = \mp \infty,
    \quad a < 0, \\
    &( +\infty ) \cdot ( +\infty ) = ( -\infty ) \cdot ( -\infty )
    = +\infty, \\
    &( +\infty ) \cdot ( -\infty ) = -\infty.
  \end{split}
\end{equation}

\VerSpaceFour





\noindent
\Str{7} Iloczyn w~$\overline{ \Rbb }$ nie jest ciągły
dla~$x = \pm \infty$ i~$y = 0$ (odpowiednio $x = 0$
i~$y = \pm \infty$), bo jeśli weźmiemy $x_{ n } = n$, $y_{ n } = 1 / n$,
to
\begin{equation}
  \label{eq:LojasiewiczWDTFRz-02}
  \lim\limits_{ n \to \infty } x_{ n } y_{ n }
  = \lim\limits_{ n \to \infty } 1 = 1 \neq 0 = ( +\infty ) \cdot 0.
\end{equation}
Suma nie jest ciągła dla~$x = +\infty$ i~$y = -\infty$ (odpowiednio
$x = -\infty$ i~$y = +\infty$), bo~jeśli weźmiemy $x_{ n } = 1 + n$
i~$y_{ n } = -n$, to
\begin{equation}
  \label{eq:LojasiewiczWDTFRz-03}
  \lim\limits_{ n \to \infty } ( x_{ n } + y_{ n } )
  = \lim\limits_{ n \to \infty } 1 = 1 \neq 0 = +\infty - \infty.
\end{equation}
Mnożenie nie jest łączne, bo
\begin{equation}
  \label{eq:LojasiewiczWDTFRz-04}
  \begin{split}
    &( a + \infty ) - \infty = \infty - \infty = 0, \\
    &a + ( \infty - \infty ) = a + 0 = a.
  \end{split}
\end{equation}
Mnożenie nie jest rozłączne względem dodawania, bo
\begin{equation}
  \label{eq:LojasiewiczWDTFRz-05}
  \begin{split}
    &( +\infty ) \cdot ( a - \infty ) = ( +\infty \cdot a ) - \infty
    = +\infty - \infty = 0, \\
    &( +\infty ) \cdot ( a - \infty ) = ( +\infty ) \cdot ( -\infty )
    = -\infty.
  \end{split}
\end{equation}

\VerSpaceFour





\noindent
\Str{7} Stwierdzenie, że~przez brak łączności dodawania nie
możemy przenosić wyrazów z~jednej strony równania na drugą, może
wydawać~się nieoczywiste, dlatego tutaj wyjaśnimy to na przykładzie.
Rozpatrzmy równość
\begin{equation}
  \label{eq:LojasiewiczWDTFRz-06}
  1 + \infty = \infty.
\end{equation}
Teraz chcielibyśmy odjąć od~obu stron $\infty$ i~wykonać obliczenia
w~następujący sposób
\begin{equation}
  \label{eq:LojasiewiczWDTFRz-07}
  \begin{split}
    &\textrm{L} = 1 + \infty - \infty = 1 + 0 = 1, \\
    &\textrm{P} = \infty - \infty = 0.
  \end{split}
\end{equation}
Błąd wynika z~tego, że~tak naprawdę wyrażenie po~lewej stronie ma
następującą postać
\begin{equation}
  \label{eq:LojasiewiczWDTFRz-08}
  \textrm{L} = ( 1 + \infty ) - \infty.
\end{equation}
Chcielibyśmy przekształcić ten wzór w~następujący sposób
\begin{equation}
  \label{eq:LojasiewiczWDTFRz-09}
  1 + ( \infty - \infty ) = 1 + 0 = 1.
\end{equation}
Tego jednak zrobić nie możemy ze~względu na~brak łączności dodawania.
Zignorowanie tego prowadzi do~błędnych wyników powyżej.

\VerSpaceFour





% ##################
\newpage

\CenterBoldFont{Błędy}


\begin{center}

  \begin{tabular}{|c|c|c|c|c|}
    \hline
    Strona & \multicolumn{2}{c|}{Wiersz} & Jest
                              & Powinno być \\ \cline{2-3}
    & Od góry & Od dołu & & \\
    \hline
    \hphantom{0}9 & 15 & & $\absOne{ \alpha }$ $\absOne{ \beta }$
           & $\absOne{ \alpha }$, $\absOne{ \beta }$ \\
    11 & & \hphantom{0}5 & $\{ \alpha_{ n } \}$ & $\{ x_{ n } \}$ \\
    % & & & & \\
    % & & & & \\
    % & & & & \\
    % & & & & \\
    \hline
  \end{tabular}

\end{center}

\VerSpaceTwo






% ############################










% ######################################
\newpage

\section{Książki o~rachunkach i~obliczeniach}

\VerSpaceTwo
% ######################################



% ############################
\section{P.J. Nahin \textit{Inside Interesting Integrals},
  \cite{NahinInterestingIntegrals2015}}

\vspace{0em}


% ##################
\CenterBoldFont{Błędy}


\begin{center}

  \begin{tabular}{|c|c|c|c|c|}
    \hline
    Strona & \multicolumn{2}{c|}{Wiersz} & Jest
                              & Powinno być \\ \cline{2-3}
    & Od góry & Od dołu & & \\
    \hline
    %     & & & & \\
    %     & & & & \\
    \hline
  \end{tabular}

\end{center}

\VerSpaceTwo


\noindent
\StrWierszGora{vi}{1} \\
\Jest \textbf{\textit{This book is~dedicated to all who, when they read
    the~following line form John le~Carr\'{e}'s 1989 Cold War spy
    novel}} \\
\PowinnoByc \textbf{This book is~dedicated to all who, when they read
  the~following line form John le~Carr\'{e}'s 1989 Cold War spy
  novel} \\
\StrWierszGora{vi}{8} \\
\Jest \textbf{\textit{as well as to all who understand how frustrating is
    the~lament in Anthony Zee's books}} \\
\PowinnoByc \textbf{as well as to all who understand how frustrating is
  the~lament in Anthony Zee's books} \\


% ############################










% ####################################################################
% ####################################################################
% Bibliography

\printbibliography





% ############################
% End of the document

\end{document}
