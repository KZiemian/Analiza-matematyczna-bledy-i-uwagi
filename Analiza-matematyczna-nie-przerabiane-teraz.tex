% ######################################
% Zakomentowane tylko po to by kompilator się tym nie przejmował.
% ######################################





% ##################
\Work{ % Autor i tytuł dzieła
  Stanisław Łojasiewicz \\
  ,,Wstęp do~teorii funkcji rzeczywistych'',
  \cite{LojasiewiczWstepDoFunkcjiRzeczywistych1976} }

\CenterTB{Uwagi}

% \textbf{Konkretne strony.}

% \vspace{\spaceFour}

\start \Str{7} Dla pełniejszego wykładu, warto tu przytoczyć pozostałe
operacje z~udziałem $\pm \infty$.
\begin{equation}
  \label{eq:Lojasiewicz-01}
  \begin{split}
    &a + ( \pm \infty ) = ( \pm \infty ) + a = \pm \infty, \\
    &a \cdot ( \pm \infty ) = ( \pm \infty ) \cdot a = \pm \infty,
    \quad a > 0, \\
    &a \cdot ( \pm \infty ) = ( \pm \infty ) \cdot a = \mp \infty,
    \quad a < 0, \\
    &( +\infty ) \cdot ( +\infty ) = ( -\infty ) \cdot ( -\infty )
    = +\infty, \\
    &( +\infty ) \cdot ( -\infty ) = -\infty.
  \end{split}
\end{equation}

\vspace{\spaceFour}


\start \Str{7} Iloczyn w~$\overline{ \R }$ nie jest ciągły
dla~$x = \pm \infty$ i~$y = 0$ (odpowiednio $x = 0$
i~$y = \pm \infty$), bo jeśli weźmiemy $x_{ n } = n$, $y_{ n } = 1/n$,
to
\begin{equation}
  \label{eq:Lojasiewicz-02}
  \Lim_{ \nToInfty } x_{ n } y_{ n } = \Lim_{ \nToInfty } 1 = 1 \neq 0
  = ( +\infty ) \cdot 0.
\end{equation}
Suma nie jest ciągła dla~$x = +\infty$ i~$y = -\infty$ (odpowiednio
$x = -\infty$ i~$y = +\infty$), bo~jeśli weźmiemy $x_{ n } = 1 + n$
i~$y_{ n } = -n$, to
\begin{equation}
  \label{eq:Lojasiewicz-03}
  \begin{split}
    \Lim_{ \nToInfty } ( x_{ n } + y_{ n } ) = \Lim_{ \nToInfty } 1 =
    1 \neq 0 = +\infty - \infty.
  \end{split}
\end{equation}
Mnożenie nie jest łączne, bo
\begin{equation}
  \label{eq:Lojasiewicz-04}
  \begin{split}
    &( a + \infty ) - \infty = \infty - \infty = 0, \\
    &a + ( \infty - \infty ) = a + 0 = a.
  \end{split}
\end{equation}
Mnożenie nie jest rozłączne względem dodawania, bo
\begin{equation}
  \label{eq:Lojasiewicz-05}
  \begin{split}
    &( +\infty ) \cdot ( a - \infty ) = ( +\infty \cdot a ) - \infty
    = +\infty - \infty = 0, \\
    &( +\infty ) \cdot ( a - \infty ) = ( +\infty ) \cdot ( -\infty )
    = -\infty.
  \end{split}
\end{equation}

\vspace{\spaceFour}


\start \Str{7} Stwierdzenie, że~przez brak łączności dodawania nie
możemy przenosić wyrazów z~jednej strony równania na drugą, może
wydawać~się nieoczywiste, dlatego tutaj wyjaśnimy to na przykładzie.
Rozpatrzmy równość
\begin{equation}
  \label{eq:Lojasiewicz-06}
  1 + \infty = \infty.
\end{equation}
Teraz chcielibyśmy odjąć od~obu stron $\infty$ i~wykonać obliczenia
w~następujący sposób
\begin{equation}
  \label{eq:Lojasiewicz-07}
  \begin{split}
    &\tr{L} = 1 + \infty - \infty = 1 + 0 = 1, \\
    &\tr{P} = \infty - \infty = 0.
  \end{split}
\end{equation}
Błąd wynika z~tego, że~tak naprawdę wyrażenie po~lewej stronie ma
następującą postać
\begin{equation}
  \label{eq:Lojasiewicz-08}
  \tr{L} = ( 1 + \infty ) - \infty.
\end{equation}
Chcielibyśmy przekształcić ten wzór w~następujący sposób
\begin{equation}
  \label{eq:Lojasiewicz-09}
  1 + ( \infty - \infty ) = 1 + 0 = 1.
\end{equation}
Tego jednak zrobić nie możemy ze~względu na~brak łączności dodawania.
Zignorowanie tego prowadzi do~błędnych wyników powyżej.

\vspace{\spaceFour}



\CenterTB{Błędy}
\begin{center}
  \begin{tabular}{|c|c|c|c|c|}
    \hline
    & \multicolumn{2}{c|}{} & & \\
    Strona & \multicolumn{2}{c|}{Wiersz} & Jest
                              & Powinno być \\ \cline{2-3}
    & Od góry & Od dołu & & \\
    \hline
    9   & 15 & & $\abso{ \al }$ $\abso{ \be }$
           & $\abso{ \al }$, $\abso{ \be }$ \\
    11  & &  5 & $\set{ \alpha_{ n } }$ & $\set{ x_{ n } }$ \\
    % & & & & \\
    % & & & & \\
    % & & & & \\
    % & & & & \\
    \hline
  \end{tabular}
\end{center}

\vspace{\spaceTwo}

% \vspace{\spaceFour}





% % ####################
% \Work{ % Autor i tytuł dzieła
%   Grigorij Michajłowicz Fichtenholz \\
%   ,,Rachunek różniczkowy i~całkowy. Tom~I'',
%   \cite{FichtenholzRachunekRiCTomI2005} }


% % Uwagi:
% % \begin{itemize}
% % \item
% % \item
% % \item
% % \item
% % \end{itemize}


% Powinno być:
% \begin{itemize}
% \item[--] Str. 303.
%   \[\sum_{ i = 1 }^{ n } a_{ i }^{2} \sum_{ i = 1 }^{ n } b_{ i }^{ 2
%     } - \{ \sum_{ i = 1 }^{ n } a_{ i } b_{ i } \}^{ 2 } \geq 0
%     \textrm{,}\]
% \item[--] Str. 371.
%   \[z'_{ x } = \frac{ x }{ p } \textrm{,} \quad z'_{ y } = \frac{ y }{
%       q } \textrm{.}\]
% \item[--] Str. 376. \ldots ale równa jest 0, gdy\ldots
% \item[--] Str.
% \end{itemize}

% \vspace{\spaceTwo}










% % ##################
% \Work{
%   Włodzimierz Krysicki, L. Włodarski \\
%   ,,Analiza matematyczna w zadaniach. Tom~I'',
%   \cite{KrysickiWlodarskiAnalizaWZadaniachTomI2005}}


% \CenterTB{Uwagi}

% \start \Str{89} Problemy takie jak policzenie granicy
% $\Lim_{ \xToZero } \fr{ \sin 3 x }{ x }$, sugerują następujące
% postępowanie. Wiemy, że~$\Lim_{ \xToZero } 3x = 0$ oraz
% że~$\Lim_{ \xToZero } \fr{ \sin x }{ x } = 1$, chcielibyśmy
% przekształcić wyrażenie do postaci $3 \fr{ \sin 3x }{ 3x }$
% i~skorzystać z~tego, że~$\Lim_{ \xToZero } \fr{ \sin 3x }{ 3x } = 0$,
% pytanie jedna czy ta ostania równość zachodzi? Pozytywną odpowiedź
% na~to~pytanie daje poniższe twierdzenie.

% % Funkcje tu użyte wciąż nie są zdefiniowane \start \Str{96} Można
% % zauważyć, że~funkcje $\artanh$ i~$\arcoth$ mają taką samą pochodną,
% % więc powinny różnią~się o~stałą, co~jest nonsensem. Wyjaśnieniem
% % problemu jest fakt, że~te dwie funkcje~są określone na rozłącznych
% % dziedzinach.

% \vspace{\spaceFour}


% \start \Str{125} Warto przedyskutować, dlaczego tak ważnej jest
% założenie o~tym, że~$\dd{}{ x }{ t } \neq 0$. Jeżeli mamy dane dwie
% funkcje $y( t )$ i~$x( t )$, to~nie musi istnieć funkcja $y( x )$.
% Jeżeli jednak dla jakiegoś $t_{ 0 }$ mamy
% $\dd{}{ x }{ t }( t_{ 0 } ) \neq 0$ to~na mocy twierdzenie
% o~odwracaniu funkcji klasy $\Cj$ istnieje\footnote{Nie wiem czy
%   założenie o~klasie różniczkowalności jest konieczne, bowiem
%   w~przypadku funkcji rzeczywistej jednej zmiennej istnieje wiele
%   wariantów twierdzenia o~funkcji uwikłanej i~funkcji odwrotnej, więc
%   może~się okazać, iż~jeden z~nich pozwala osłabić ten warunek.
%   Szeroką dyskusję tych twierdzeń można znaleźć w~książce Fichtenholza
%   \cite{FichtenholzRachunekRiCTomI2005}.} funkcja $t( x )$ w~pewnym
% otoczeniu $t_{ 0 }$ i~w~tym otoczeniu $y( x ) = y( t( x ) )$.

% Co~jednak dzieje~się, gdy~$\dd{}{ x }{ t }( t_{ 0 } ) = 0$, czy
% funkcja $y( x )$ wówczas nie istnieje? Następujący przypadek pokazuje,
% że~tak nie musi być. Rozpatrzmy funkcje $x( t ) = t^{ 3 }$,
% $y( t ) = t$. Pomimo, że~$\dd{}{ x }{ t }( 0 ) = 0$~to, można to
% zauważyć np. rysując wykresy obu funkcji, można odwikłać $t( x )$
% i~wówczas
% \begin{equation*}
%   y( x ) =
%   \begin{cases}
%     \sqrt[1 / 3]{ x } & x \geq 0, \\
%     -\sqrt[1 / 3]{ x } & x < 0.
%   \end{cases}
% \end{equation*}
% Funkcja ta jednak nie jest różniczkowalna dla $x = 0$. Nie potrafię
% stwierdzić, czy znikanie pochodnej $\dd{}{ x }{ t }( t_{ 0 } )$ musi
% pociągać za sobą, że~jeśli funkcja $y( x )$ istnieje to jest
% nieróżniczkowalna po $x$ w~punkcie $x( t_{ 0 } )$. Wątpię jednak,
% aby~tak było.

% \vspace{\spaceFour}


% \start \Str{130} Autorzy popełnili tu błąd przyjmując, że~moduł
% pochodnej równy jest pochodnej modułu:
% \begin{equation*}
%   \left| \dd{}{ f }{ t } \right| = \dd{}{ \abso{ f } }{ t }.
% \end{equation*}
% Że~jest inaczej można~się przekonać rozważając znany przykłady ruchu
% jednostajnego po okręgu, gdzie prędkość ma stałą długość, a~jednak
% moduł przyśpieszenia nie jest równy 0, lecz $\fr{ v^{ 2 } }{ \rho }$,
% gdzie $\rho$ to promień krzywizny.

% Przeprowadzając proste obliczenia dostajemy poprawne wzory na składowe
% i~moduł przyśpieszenia:
% \begin{align*}
%   a_{ x } &= 50 ( \cos 5t^{ 2 } - 100 t^{ 2 } \sin 5t^{ 2 } ), \\
%   a_{ y } &= -50 ( \sin 5t^{ 2 } + 100 t^{ 2 } \cos 5t^{ 2 } ), \\
%   a &= 50 \sqrt{ 1 + 100 t^{ 4 } }.
% \end{align*}
% Widzimy więc, że~moduł rośnie w~czasie. Należało~się tego spodziewać,
% bowiem przyśpieszenie normalne do toru dane jest wzorem
% $\fr{ v^{ 2 } }{ \rho }$, więc jeśli prędkość rośnie liniowo, to ta
% składowa przyśpieszenia również musi rosnąć.

% % Uwagi:
% % \begin{itemize}
% % \item
% % \item
% % \item
% % \item
% % \end{itemize}

% \CenterTB{Błędy}
% \begin{center}
%   \begin{tabular}{|c|c|c|c|c|}
%     \hline
%     & \multicolumn{2}{c|}{} & & \\
%     Strona & \multicolumn{2}{c|}{Wiersz} & Jest
%                               & Powinno być \\ \cline{2-3}
%     & Od góry & Od dołu & & \\
%     \hline
%     46  &  8 & & $\sqrt[n]{ u_{ n } } < p$ & $\sqrt[n]{ u_{ n } } \leq p$ \\
%     79  & &  8 & $-b \, a$ & $-b / a$ \\
%     80  & 10 & & $= 2$ & $x = 2$ \\
%     88  & &  2 & $\fr{ x^{ 2 } - 1 }{ x - 2 }$ & $\fr{ x^{ 2 } - 4 }
%                                                  { x - 2 }$ \\
%     89  &  5 & & $\fr{ ( x - 3 )( -1 )^{ [ x ] } }{ x^{ 2 } - 9 }$
%            & $\fr{ ( x + 3 )( -1 )^{ [ x ] } }{ x^{ 2 } - 9 }$ \\
%     96  & &  3 & $\fr{ -1 }{ 1 - x^{ 2 } }$
%            & $\fr{ 1 }{ 1 - x^{ 2 } }$ \\
%     116 &  2 & & $\left[ 2 \tan \fr{ x }{ 3 } + 1 \right)$
%            & $ \left( 2 \tan \fr{ x }{ 3 } + 1 \right)$ \\
%     129 & &  5 & $a = 796$ & $a = 7.96$ \\
%     231 & 11 & & $\bigg| \Sum_{ k = 0 }^{ n } u( x ) - S( x ) < \bigg|
%                  \veps$
%            & $\bigg| \Sum_{ k = 0 }^{ n } u( x )
%              - S( x ) \bigg| < \veps$ \\
%     234 & & 11 & $\sin{ 1 \atop n }$ & $\sin \fr{ 1 }{ n }$ \\
%     241 & &  6 & $f( 0 )$ & $f'( 0 )$ \\
%     241 & &  4 & $2^{ 3 }$ & $2^{ 2 }$ \\
%     256 & &  4 & $\fr{ f'( x ) }{ {}'( x ) }$ & $\fr{ f'( x ) }
%                                                 { g'( x ) }$ \\
%     258 & &  4 & ${ 1 \atop t^{ 2 } }$ & $\fr{ 1 }{ t^{ 2 } }$ \\
%     259 &  8 & & oraz \emph{istnieje} & \emph{ale istnieje} \\
%     441 & &  7 & $\arctan^{ 3 }x$ & $\arctan x^{ 3 }$ \\
%     442 & &  6 & $\fr{ 1 }{ ( 1 - x ) \sqrt{ x } }$
%            & $\fr{ 1 }{ x \ln x \ln( \ln x ) }$ \\
%     436 & &  6 & $-4$ & $-2$ \\
%     % & & & & \\
%     % & & & & \\
%     % & & & & \\
%     \hline
%   \end{tabular}
% \end{center}

% \begin{itemize}

% \item[--] Str. 325. 16.32.
%   $\int \frac{ 6 x - 13 }{ x^{ 2 } - \frac{ 7 }{ 2 } x + \frac{ 3 }{ 2
%     } } dx$.

% \item[--] Str. 378. 19.15.
%   $\int \limits_{ 0 }^{ a } 3x \sqrt{ x^{ 2 } + 4 a^{ 2 } } dx, a > 0
%   \textrm{.}$

% \item[--] Str. 436. 5.38. $\frac{ 2 }{ 3 }$.

% \item[--] Str. 438. 6.50.
%   $y' = 7 x^{ 4 / 3 } - 13 x^{ 9 / 4 } - \frac{ 2 }{ 7 } x^{ -3 / 2 }
%   \textrm{.}$

% \item[--] Str. 438. 6.53.
%   $y' = x^{ -2 / 3 } - 3 x^{ 2 } + \frac{ 1 }{ 2 } \frac{ 1 }{ \sqrt[
%     4 ]{ x } }$.

% \item[--] Str. 438. 6.56. \ldots
%   $y' = \frac{ -5 }{ 7 \sqrt[ 7 ]{ x^{ 8 } } } - 14 x^{ 6 } - \frac{ 3
%   }{ 4 \sqrt{ x^{ 3 } } }$.

% \item[--] Str. 438. 6.89. \ldots
%   $z' = \frac{ -2 a x }{ ( a^{ 2 } + x^{ 2 } ) \sqrt{ a^{ 4 } - x^{ 4
%       } } }$.

% \item[--] Str. 440. 6.113. $\cos x \neq 0$,
%   $y' = \frac{ 7 \sin^{ 3 } x }{ \cos^{ 8 } x }$.

% \item[--] Str. 441. 6.129. $x > 1$,
%   $y' = \frac{ x \ln x }{ \sqrt{ ( x^{ 2 } - 1 )^{ 3 } } }$.

% \item[--] Str. 441. 6.131.
%   $y' = x^{ 4 } \arctan x + \frac{ x^{ 5 } - x }{ 5 ( 1 + x^{ 2 } ) }
%   - \frac{ 1 }{ 5 } x^{ 3 } + \frac{ 1 }{ 5 } x$.
% \item[--] Str. 478. \ldots
%   $I = \frac{ 1 }{ 3 } \ln | a^{ 3 } + x^{ 3 } |$.

% \item[--] Str. 480. 16.26. $I = \frac{ 1 }{ 8 } ( 2 x + 1 )^{ 4 }$.

% \item[--] Str. 480. 16.27. $x \neq \frac{ 2 }{ 3 }$;
%   $I = \frac{ -1 }{ 9 ( 3 x - 2 )^{ 3 } }$.

% \item[--] Str. 480. 16.37. $x \neq \frac{ 2 }{ 3 }$,
%   $x \neq \frac{ 3 }{ 2 }$;
%   $I = \frac{ 1 }{ 5 } \ln | \frac{ 2 x - 3 }{ 3 x - 2 } |$.

% \end{itemize}

% \vspace{\spaceTwo}





% % ##################
% \Work{ % Autorzy i tytuł dzieła
%   Włodzimierz Krysicki, L. Włodarski \\
%   ,,Analiza matematyczna w~zadaniach. Tom~II'',
%   \cite{KrysickiWlodarskiAnalizaWZadaniachTomII2004} }


% \CenterTB{Uwagi}

% \start \Str{199} \tb{Zadanie 7.2.} Rozwiązanie tego zadania jest
% niepełne, w~następujący sensie. Równanie numer (2) w~tym zadaniu,
% po~spierwiastkowaniu sprowadza~się do~równania:
% \begin{equation*}
%   \left| \dd{}{ y }{ x } \right| = \abso{ \cos x }.
% \end{equation*}
% Równania tego nie da~się przedstawić w~postaci Newtona, zaś poza
% równaniami (3) z~tego zadania, można uzyskać z~niego nieskończoną
% ilość innych, np. ograniczając~się do przedziału
% $( -\fr{ \pi }{ 2 }, \frac{ 3 \pi }{ 2 } )$ możemy rozpatrzyć:
% \begin{equation*}
%   \dd{}{ y }{ x } =
%   \begin{cases}
%     -\cos x, & x \in ( -\fr{ \pi }{ 2 }, \fr{ \pi }{ 2 } ), \\
%     \cos x, & x \in ( \fr{ \pi }{ 2 }, \fr{ 3 \pi }{ 2 } ).
%   \end{cases}
% \end{equation*}
% Dla dowolnych warunków początkowych w~rozpatrywanym przedziale, można
% znaleźć rozwiązania tego równania różniczkowalne w~całym przedziale.
% Na przykład dla $x_{ 0 } = \fr{ \pi }{ 2 }, y_{ 0 } = 0$, takim
% rozwiązaniem jest:
% \begin{equation*}
%   y( x )
%   \begin{cases}
%     -\sin x + 1, & x \in ( -\fr{ \pi }{ 2 }, \fr{ \pi }{ 2 } ), \\
%     \sin x - 1, & x \in ( \fr{ \pi }{ 2 }, \fr{ 3 \pi }{ 2 } ).
%   \end{cases}
% \end{equation*}

% \vspace{\spaceFour}


% \start \Str{200} \tb{Zadanie 7.3.} Również tutaj problem znalezienia
% wszystkich rozwiązań równania różniczkowego nie został w~pełni
% rozwiązany. Ograniczając~się do przedziału $( 0, 2 \pi )$ możemy
% zauważyć, że~funkcja:
% \begin{equation*}
%   y( x ) =
%   \begin{cases}
%     C_{ 1 } \sin x,& x \in ( 0, \pi ), \\
%     0, & x = \pi, \\
%     C_{ 2 } \sin x,& x \in ( \pi, 2 \pi ),
%   \end{cases}
% \end{equation*}
% jest rozwiązaniem równania (1) z~tego zadania w~całym badanym
% przedziale, choć nie jest różniczkowalne w~0.


% \CenterTB{Błędy}
% \begin{center}
%   \begin{tabular}{|c|c|c|c|c|}
%     \hline
%     & \multicolumn{2}{c|}{} & & \\
%     Strona & \multicolumn{2}{c|}{Wiersz} & Jest
%                               & Powinno być \\ \cline{2-3}
%     & Od góry & Od dołu & & \\
%     \hline
%     25  & &  7 & $( \sin z )^{ \tan z }$ & $( \sin x )^{ \tan z }$ \\
%     26  & 10 & & $3 \cdot 3$ & $9$ \\
%     26  & &  4 & $|\, \mr{n}$ & $\ln$ \\
%     26  &  2 & & $e^{ x^{ 2 } \sin(x - y^{ 2 }) }$
%            & $\sin^{ x^{ 2 } }(x - y^{ 2 })$ \\
%     26  &  1 & & $e^{ x^{ 2 } \sin(x - y^{ 2 }) }$
%            & $\sin^{ x^{ 2 } }(x - y^{ 2 })$ \\
%            % & & & & \\
%            % & & & & \\
%     201 & &  1 & $+$ & $=$ \\
%     203 &  7 & & $\fr{ a y }{ d x }$ & $\dd{}{ y }{ x }$ \\
%     203 &  7 & & sprawdzają & spełniają \\
%     207 &  2 & & $1 \fr{ 1 + \tfr{ 1 }{ C_{ 2 }^{ 2 } } }
%                  { x^{ 2 } + 1 }$
%            & $1 - \fr{ 1 + \fr{ 1 }{ C_{ 2 }^{ 2 } } }
%              { x^{ 2 } + 1 }$ \\
%              % & & & & \\
%     431 & &  3 & $x^{ \fr{ 1 }{ y - 1 } }$ & $x^{ \fr{ 1 }{ y } - 1 }$ \\
%     432 & &  9 & $\fr{\uppi R^{ 3 } }{ 3 }$ & $\fr{\uppi R^{ 2 } }{ 3 }$ \\
%     449 & &  9 & $C - e^{ \fr{ 1 }{ x } }$ & $C e^{ -\fr{ 1 }{ x } }$ \\
%     449 & &  9 & $c$ & $C$ \\
%     % & & & & \\
%     \hline
%   \end{tabular}
% \end{center}

% \begin{itemize}
% \item[--] Str. 432.
%   $\fr{ \pr^{ 2 } u }{ \pr x \pr y } = \fr{ \pr^{ 2 } u }{ \pr y \pr x
%   } = 6 x^{ 2 } - 30 x y^{ 2 } - 2 \sin 2y$.

% \item[--] Str. 454. 9.5. $y = C \frac{ x - 2 }{ x + 2 }$.

% \item[--] Str. 454. 9.5. $y = C x$.

%   % \item[--] Str.
%   % \item[--] Str.
%   % \item[--] Str.
%   % \item[--] Str.
%   % \item[--] Str.
% \end{itemize}

% \vspace{\spaceTwo}















% % ####################
% \Work{
%   W. Żakowski, W. Lesiński \\
%   ,,Matematyka. Część IV'' % \cite{ZL78}
% }


% \CenterTB{Uwagi}


% \noi \tb{Część~I.}

% \vspace{\spaceFour}


% \start Często w~tej części książki pojawia~się następująca sytuacja.
% W~wyniku rachunków otrzymaliśmy całkę ogólną pewnego równania
% różniczkowego $\vp( x, C ) = 0$, przy czym
% $C \in ( a, b ) \sum ( b, c )$. W~każdym rozważanym przypadku
% okazywało~się, że~funkcję $\vp( x, C )$ można w~naturalny sposób
% przedłużyć do~$C = b$ i~$\vp( x, b ) = $ również jest rozwiązaniem
% tego równania. Czy to jest zbieg okoliczności, czy~też przy pewnych
% warunkach musi to zachodzić? \Prze

% \vspace{\spaceThree}



% \noi \tb{Konkretne strony.}

% \vspace{\spaceFour}


% \start \Str{23} \Dok

% \start \Str{26} Rachunki prowadzące do równania (I.62) dowodzą,
% że~przedstawia ono rozwiązanie wyjściowego równania różniczkowego
% dla~każdej stałej $C_{ 2 }$ różnej od~zera. Aby~zasadnie twierdzić,
% że~jest to rozwiązanie tego równania dla~każdej stałej rzeczywistej
% $C_{ 2 }$ należy podstawić\footnote{W~tym momencie nie znam prostszego
%   rozwiązania tego problemu, ale~w~uwagach do części~I tej książki,
%   jest zawarta sugestia innego podejścia.} $C_{ 2 } = 0$, co prowadzi
% do~$u = 1 \pm \sqrt{2}$, i~sprawdzić, że~otrzymana w~ten sposób
% funkcja jest rozwiązaniem zadanego równania. Jak~się okazuje, tak
% w~istocie jest.

% % \CenterTB{Błędy}
% % \begin{center}
% %   \begin{tabular}{|c|c|c|c|c|}
% %     \hline
% %     & \multicolumn{2}{c|}{} & & \\
% %     Strona & \multicolumn{2}{c|}{Wiersz} & Jest
% %                               & Powinno być \\ \cline{2-3}
% %     & Od góry & Od dołu & & \\
% %     \hline
% %     %     & & & & \\
% %     %     & & & & \\
% %     %     & & & & \\
% %     \hline
% %   \end{tabular}
% % \end{center}



% % ####################
% \Work{ % Autor i tytuł dzieła
% B. W. Szabat \\
% ,,Wstęp do analizy zespolonej'',
% \cite{SzabatWstepDoAnalizyZespolonej1974} }


% \CenterTB{Błędy}
% \begin{center}
%   \begin{tabular}{|c|c|c|c|c|}
%     \hline
%     & \multicolumn{2}{c|}{} & & \\
%     Strona & \multicolumn{2}{c|}{Wiersz} & Jest
%                               & Powinno być \\ \cline{2-3}
%     & Od góry & Od dołu & & \\
%     \hline
%     17  &  5 & & dal & dal\dywiz \\
%     25  & & 11 & $z_{ 2 }( t )$ & $z_{ 2 }( \tau( t ) )$ \\
%     %     & & & & \\
%     \hline
%   \end{tabular}
% \end{center}

% \vspace{\spaceTwo}





% ##################
\Work{ % Autor i tytuł dzieła
  Krzysztof Maurin \\
  ,,Analiza. Część~I: Elementy'', \cite{} }


Uwagi:
\begin{itemize}
\item Tyle błędów, że szybciej będzie napisać tę książkę od nowa.
\item Paragraf poświęcony całce Riemanna, jest napisany nie najlepiej.
  Nie udowodniono wszystkich ważnych twierdzeń, nawet tych z których
  korzysta. Ponadto dowody są w dużej mierze przeprowadzone bardzo
  pobieżnie i niewiele mówią czytelnikowi.
\item Kwestia analityczności funkcji potraktowna pobieżnie. Dla
  funkcji takich jak $\cos$, $\sin$ etc. pokazano ich analityczność w
  0, nic nie powiedziano o analityczności w innych punktach.
\item W ogóle nie przedyskutowano problemu mnożenia szeregów.
\item Większość z definicji i twierdzeń Rozdziału VII da się uogólnić
  dla przypadku przetrzeni unormowanych.
\item Str. 10. Wszystkie dowody ,,nie wprost'', są przypisane zasadzie
  \\ kontrapozycji.
\item Str. 21. Użyte jest pojęcie subtelniejszego podziału, pomimo że
  nie zostało ono zdefiniowane.
\item Str. 40. Twierdzenie tu zapowiedziane nie zostało nigdy
  udowodnione.
\item Str. 41. Uwaga napisana jest fatalnie. Nawet jeżeli to
  twierdzenie jest prawdziwe, co nie jest oczywiste, to sforumłowanie,
  zaciemnia cały problem.
\item Str. 43. W drugiej części dowodu Wniosku II.5 zupełnie
  niepotrzebnie wprowadzono kule otwarte zawarte w $Z_{ 1 }$ i
  $Z_{ 2 }$.
\item Str. 48. Dowód pierwszej części Twierdzenia II. 18. jest
  zupełnie niezrozumiały.
\item Str. 53. Czy w twierdzeniu o pochodnej funkcji odwrotnej jest
  ważna ciągłość tej funkcji?
\item Str. 56. Dyskusja skierowania w rodzinie zbiorów $\Pi$, bardziej
  zaciemnia niż wyjasnia. Po przeanalizowaniu okazuje się, że w
  definicji $\limsup$ powinno być: \\$A_{ i } \prec A_{ j }
  \Leftrightarrow A_{ j } \subset A_{ i
  }$, choć z dyskusji wynika wręcz przeciwnie.
\item Str. 57. Nie zdefiniowano na jakie odcinki dzielimy przedział $[
  a, b
  ]$. Okazuje się, że nie ma to znaczenia jeśli przyjmiemy miarę
  dowolnego odcinka $\mu ( \{ a, b \} ) = b -
  a$, jednak ta lekkomyślność jest rażąca.
\item Str. 63. W dowodzie Twierdzenia.III.13 ,,potrzeba i nie
  potrzeba'' założenia $\varphi' \neq
  0$, zależnie od rozpatrywanego przypadku.
\item Str. 66. Skorzystano tu z twierdzenia dla liczb rzeczywistych,
  które zostało udowodnione tylko dla liczb całkowitych.
\item Str. 68. Dowód korzysta z pojęcia bazy otoczeń domkniętych,
  które niezostało nigdzie wcześniej przedstawione.
\item Str. 90. Podane są warunki na to by ciąg $T_{ n
  }$ był ciągiem Cauchy`ego, nie zaś by było on zbieżny do $T_{ 0 }$.
\item Str. 90. W Lemat V.5. zawiera poważne pomieszanie pojęć.
  Prawidłowe sformułowanie powinno brzmieć:
\item Str. 92. W dowodzie Lematu V.6. użyte są kule otwarte a powinny
  być domknięte.
\item Str. 93. Dowód testu Weierstrassa dla funkcji jest bez sensu.
  Dowodzi tylko punktowej zbieżności szeregu funkcji. Zbieżności
  jednostajnej należy dowodzić inną metodą.
\item Str. 97. Dowód punktu (b) uwagii łatwo przeprowadzić przez
  kontrapozycje, nie widzę jednak sposobu udowodnienia go w sposób
  analogiczny do punktu (a). W dowodzie tym kluczowe jest oszacowanie
  \\$|a_{n}(z_{1}-z_{0})^{n}|<M$. Jednak dla szeregu o niezerowym
  promieniu zbieżności nie być możliwe oszacowanie szeregu od góry
  albo od dołu. Za przykłąd może posłużyć zachowanie szeregów
  $\sum \frac{1}{n}z^{n}$ i $\sum nz^{n}$ w $z=1$. Oba są tam
  rozbieżne
\item Str. 98. Dowód twierdzenia Cauchy'ego-Hadamarda zawiera w sobie
  dowód kryterium Cauchy`ego zbieżności szeregów. Czyni to dowód
  bardzo zagmatwanym i trudnym do zrozumienia.
\end{itemize}


% Błędy:\\
% \begin{tabular}{|c|c|c|c|c|}
%   \hline
%   & \multicolumn{2}{c|}{} & & \\
%   Strona & \multicolumn{2}{c|}{Wiersz} & Jest
%                             & Powinno być \\ \cline{2-3}
%   & Od góry & Od dołu & & \\
%   \hline
%   & & & & \\
%   \hline
% \end{tabular}

\begin{itemize}
\item[--] Str. 23. \ldots 2$^{ \circ }$ identytywna\ldots
\item[--] Str. 28. \emph{Ad} 2$^{ \circ }$: $| ( a'_{ n } + b'_{ n } ) - ( a_{ n } + b_{ n } ) |  \leq | a'_{ n } - a_{ n } | + | b'_{ n } - b_{ n } | <2 \, \varepsilon$, bo\\
  $\big( ( a'_{ n } ) \sim ( a_{ n } ) \wedge( b'_{ n } ) \sim ( b_{ n
  } ) \big) \Rightarrow \big( ( | a'_{ n } - a_{ n } | < \varepsilon )
  \wedge | b'_{ n } - b_{ n } | <\varepsilon, \textrm{ dla } n > N )
  \big)$.
\item[--] Str. 32.
\item[--] Str. 43. \ldots wynika z definicji odwzorowania ciągłego.
\item[--] Str. 54. Niech $g'( x ) h + r_{ 1 }( h ) \neq 0$\ldots
\item[--] Str.
  97.$$t := \frac{ | z - z_{ 0 } | }{ | z_{ 1 } - z_{ 0 } | }
  \textrm{\ldots}$$
\item[--] Str. 112.
  $$\big( 0 \leq f \leq C \frac{ 1 }{ x^{ \mu } } \textrm{ ; } \mu
  \geq 1 \big) \Rightarrow \big( \int \limits^{ \infty }_{ a } f <
  \infty \big) \textrm{;}$$
\item[--] Str. 112.
  $$\big( 0 \leq f \leq C \frac{ 1 }{ x ( \log x )^{ \mu } } \textrm{
    ; } \mu \geq 1 \big) \Rightarrow \big( \int \limits^{ \infty }_{ a
  } f < \infty \big) \textrm{;}$$
\item[--] Str. 125. \ldots
\item[--] Str. 143. \ldots korzystając ze wzorów (4) i (5)\ldots
\end{itemize}





% ##################
\Work{ % Autor i tytuł dzieła
  Krzysztof Maurin \\
  ,,Analiza. Część II: Ogólne struktury, funkcje algebraiczne,\\
  całkowanie, analiza tensorowa'', \cite{} }


Uwagi:
\begin{itemize}
\item Str 90. W równaniu
  \begin{displaymath}
    \rho( v )^{ 2 } = Q( v ) \cdot 1,
  \end{displaymath}
  1 po prawej stronie oznacza jedynkę w algebrze $\mathcal{C}( V )$. W
  tekście nie zostało to w ogóle zaznaczone.
\end{itemize}

\begin{center}
  \begin{tabular}{|c|c|c|c|c|}
    \hline
    & \multicolumn{2}{c|}{} & & \\
    Strona & \multicolumn{2}{c|}{Wiersz} & Jest
                              & Powinno być \\ \cline{2-3}
    & Od góry & Od dołu & & \\
    \hline
    24  & 17 & & $\exists_{ \substack{ W \in \mathcal{B}( x ) } }$
           & $\exists_{ \substack{ W \in \mathcal{B}( y ) } }$ \\
    26  & 16 & & $\Leftarrow:$ & $\Leftarrow$ (a.a.). \\
    90  & & 17 & nieskończenie & niekoniecznie \\
    90  & &  8 & $V = (V, B)$ & $V = (V, Q)$ \\
    90  & &  8 & algebra & algebra z jedynką \\
    92  & & 17 & $2Q( e_{ i }, e_{ j } )$ & $B( e_{ i }, e_{ j } )$ \\
    99  &  3 & & $K( a )$ & $\mathbf{K}( a )$ \\
    % & & & & \\
    % & & & & \\
    % & & & & \\
    % & & & & \\
    \hline
  \end{tabular}
\end{center}
