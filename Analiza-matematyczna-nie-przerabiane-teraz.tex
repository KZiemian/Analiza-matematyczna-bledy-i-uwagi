% ######################################
% Zakomentowane tylko po to by kompilator się tym nie przejmował.
% ######################################







% ####################
\Work{ % Autor i tytuł dzieła
  P. J. Nahin \\
  ,,Inside Interesting Integrals'',
  \cite{NahinInterestingIntegrals2015} }


\CenterTB{Błędy}
% \begin{center}
%   \begin{tabular}{|c|c|c|c|c|}
%     \hline
%     & \multicolumn{2}{c|}{} & & \\
%     Strona & \multicolumn{2}{c|}{Wiersz} & Jest
%                               & Powinno być \\ \cline{2-3}
%     & Od góry & Od dołu & & \\
%     \hline
%     %     & & & & \\
%     %     & & & & \\
%     \hline
%   \end{tabular}
% \end{center}
\noi
\StrWg{vi}{1} \\
\Jest \textbf{\emph{This book is~dedicated to all who, when they read
    the~following line form John le~Carr\'{e}'s 1989 Cold War spy
    novel}} \\
\Powin \textbf{This book is~dedicated to all who, when they read
  the~following line form John le~Carr\'{e}'s 1989 Cold War spy
  novel} \\
\StrWg{vi}{8} \\
\Jest \emph{\textbf{as well as to all who understand how frustrating is
    the~lament in Anthony Zee's books}} \\
\Powin \textbf{as well as to all who understand how frustrating is
  the~lament in Anthony Zee's books} \\

\vspace{\spaceTwo}









% % ######################################
% \newpage
% \section{Rachunek wariacyjny}

% \vspace{\spaceTwo}
% % ######################################




% \vspace{\spaceFour}















% % ##################
% \Work{
%   Włodzimierz Krysicki, L. Włodarski \\
%   ,,Analiza matematyczna w zadaniach. Tom~I'',
%   \cite{KrysickiWlodarskiAnalizaWZadaniachTomI2005}}


% \CenterTB{Uwagi}

% \start \Str{89} Problemy takie jak policzenie granicy
% $\Lim_{ \xToZero } \fr{ \sin 3 x }{ x }$, sugerują następujące
% postępowanie. Wiemy, że~$\Lim_{ \xToZero } 3x = 0$ oraz
% że~$\Lim_{ \xToZero } \fr{ \sin x }{ x } = 1$, chcielibyśmy
% przekształcić wyrażenie do postaci $3 \fr{ \sin 3x }{ 3x }$
% i~skorzystać z~tego, że~$\Lim_{ \xToZero } \fr{ \sin 3x }{ 3x } = 0$,
% pytanie jedna czy ta ostania równość zachodzi? Pozytywną odpowiedź
% na~to~pytanie daje poniższe twierdzenie.

% % Funkcje tu użyte wciąż nie są zdefiniowane \start \Str{96} Można
% % zauważyć, że~funkcje $\artanh$ i~$\arcoth$ mają taką samą pochodną,
% % więc powinny różnią~się o~stałą, co~jest nonsensem. Wyjaśnieniem
% % problemu jest fakt, że~te dwie funkcje~są określone na rozłącznych
% % dziedzinach.

% \vspace{\spaceFour}


% \start \Str{125} Warto przedyskutować, dlaczego tak ważnej jest
% założenie o~tym, że~$\dd{}{ x }{ t } \neq 0$. Jeżeli mamy dane dwie
% funkcje $y( t )$ i~$x( t )$, to~nie musi istnieć funkcja $y( x )$.
% Jeżeli jednak dla jakiegoś $t_{ 0 }$ mamy
% $\dd{}{ x }{ t }( t_{ 0 } ) \neq 0$ to~na mocy twierdzenie
% o~odwracaniu funkcji klasy $\Cj$ istnieje\footnote{Nie wiem czy
%   założenie o~klasie różniczkowalności jest konieczne, bowiem
%   w~przypadku funkcji rzeczywistej jednej zmiennej istnieje wiele
%   wariantów twierdzenia o~funkcji uwikłanej i~funkcji odwrotnej, więc
%   może~się okazać, iż~jeden z~nich pozwala osłabić ten warunek.
%   Szeroką dyskusję tych twierdzeń można znaleźć w~książce Fichtenholza
%   \cite{FichtenholzRachunekRiCTomI2005}.} funkcja $t( x )$ w~pewnym
% otoczeniu $t_{ 0 }$ i~w~tym otoczeniu $y( x ) = y( t( x ) )$.

% Co~jednak dzieje~się, gdy~$\dd{}{ x }{ t }( t_{ 0 } ) = 0$, czy
% funkcja $y( x )$ wówczas nie istnieje? Następujący przypadek pokazuje,
% że~tak nie musi być. Rozpatrzmy funkcje $x( t ) = t^{ 3 }$,
% $y( t ) = t$. Pomimo, że~$\dd{}{ x }{ t }( 0 ) = 0$~to, można to
% zauważyć np. rysując wykresy obu funkcji, można odwikłać $t( x )$
% i~wówczas
% \begin{equation*}
%   y( x ) =
%   \begin{cases}
%     \sqrt[1 / 3]{ x } & x \geq 0, \\
%     -\sqrt[1 / 3]{ x } & x < 0.
%   \end{cases}
% \end{equation*}
% Funkcja ta jednak nie jest różniczkowalna dla $x = 0$. Nie potrafię
% stwierdzić, czy znikanie pochodnej $\dd{}{ x }{ t }( t_{ 0 } )$ musi
% pociągać za sobą, że~jeśli funkcja $y( x )$ istnieje to jest
% nieróżniczkowalna po $x$ w~punkcie $x( t_{ 0 } )$. Wątpię jednak,
% aby~tak było.

% \vspace{\spaceFour}


% \start \Str{130} Autorzy popełnili tu błąd przyjmując, że~moduł
% pochodnej równy jest pochodnej modułu:
% \begin{equation*}
%   \left| \dd{}{ f }{ t } \right| = \dd{}{ \abso{ f } }{ t }.
% \end{equation*}
% Że~jest inaczej można~się przekonać rozważając znany przykłady ruchu
% jednostajnego po okręgu, gdzie prędkość ma stałą długość, a~jednak
% moduł przyśpieszenia nie jest równy 0, lecz $\fr{ v^{ 2 } }{ \rho }$,
% gdzie $\rho$ to promień krzywizny.

% Przeprowadzając proste obliczenia dostajemy poprawne wzory na składowe
% i~moduł przyśpieszenia:
% \begin{align*}
%   a_{ x } &= 50 ( \cos 5t^{ 2 } - 100 t^{ 2 } \sin 5t^{ 2 } ), \\
%   a_{ y } &= -50 ( \sin 5t^{ 2 } + 100 t^{ 2 } \cos 5t^{ 2 } ), \\
%   a &= 50 \sqrt{ 1 + 100 t^{ 4 } }.
% \end{align*}
% Widzimy więc, że~moduł rośnie w~czasie. Należało~się tego spodziewać,
% bowiem przyśpieszenie normalne do toru dane jest wzorem
% $\fr{ v^{ 2 } }{ \rho }$, więc jeśli prędkość rośnie liniowo, to ta
% składowa przyśpieszenia również musi rosnąć.

% % Uwagi:
% % \begin{itemize}
% % \item
% % \item
% % \item
% % \item
% % \end{itemize}

% \CenterTB{Błędy}
% \begin{center}
%   \begin{tabular}{|c|c|c|c|c|}
%     \hline
%     & \multicolumn{2}{c|}{} & & \\
%     Strona & \multicolumn{2}{c|}{Wiersz} & Jest
%                               & Powinno być \\ \cline{2-3}
%     & Od góry & Od dołu & & \\
%     \hline
%     46  &  8 & & $\sqrt[n]{ u_{ n } } < p$ & $\sqrt[n]{ u_{ n } } \leq p$ \\
%     79  & &  8 & $-b \, a$ & $-b / a$ \\
%     80  & 10 & & $= 2$ & $x = 2$ \\
%     88  & &  2 & $\fr{ x^{ 2 } - 1 }{ x - 2 }$ & $\fr{ x^{ 2 } - 4 }
%                                                  { x - 2 }$ \\
%     89  &  5 & & $\fr{ ( x - 3 )( -1 )^{ [ x ] } }{ x^{ 2 } - 9 }$
%            & $\fr{ ( x + 3 )( -1 )^{ [ x ] } }{ x^{ 2 } - 9 }$ \\
%     96  & &  3 & $\fr{ -1 }{ 1 - x^{ 2 } }$
%            & $\fr{ 1 }{ 1 - x^{ 2 } }$ \\
%     116 &  2 & & $\left[ 2 \tan \fr{ x }{ 3 } + 1 \right)$
%            & $ \left( 2 \tan \fr{ x }{ 3 } + 1 \right)$ \\
%     129 & &  5 & $a = 796$ & $a = 7.96$ \\
%     231 & 11 & & $\bigg| \Sum_{ k = 0 }^{ n } u( x ) - S( x ) < \bigg|
%                  \veps$
%            & $\bigg| \Sum_{ k = 0 }^{ n } u( x )
%              - S( x ) \bigg| < \veps$ \\
%     234 & & 11 & $\sin{ 1 \atop n }$ & $\sin \fr{ 1 }{ n }$ \\
%     241 & &  6 & $f( 0 )$ & $f'( 0 )$ \\
%     241 & &  4 & $2^{ 3 }$ & $2^{ 2 }$ \\
%     256 & &  4 & $\fr{ f'( x ) }{ {}'( x ) }$ & $\fr{ f'( x ) }
%                                                 { g'( x ) }$ \\
%     258 & &  4 & ${ 1 \atop t^{ 2 } }$ & $\fr{ 1 }{ t^{ 2 } }$ \\
%     259 &  8 & & oraz \emph{istnieje} & \emph{ale istnieje} \\
%     441 & &  7 & $\arctan^{ 3 }x$ & $\arctan x^{ 3 }$ \\
%     442 & &  6 & $\fr{ 1 }{ ( 1 - x ) \sqrt{ x } }$
%            & $\fr{ 1 }{ x \ln x \ln( \ln x ) }$ \\
%     436 & &  6 & $-4$ & $-2$ \\
%     % & & & & \\
%     % & & & & \\
%     % & & & & \\
%     \hline
%   \end{tabular}
% \end{center}

% \begin{itemize}

% \item[--] Str. 325. 16.32.
%   $\int \frac{ 6 x - 13 }{ x^{ 2 } - \frac{ 7 }{ 2 } x + \frac{ 3 }{ 2
%     } } dx$.

% \item[--] Str. 378. 19.15.
%   $\int \limits_{ 0 }^{ a } 3x \sqrt{ x^{ 2 } + 4 a^{ 2 } } dx, a > 0
%   \textrm{.}$

% \item[--] Str. 436. 5.38. $\frac{ 2 }{ 3 }$.

% \item[--] Str. 438. 6.50.
%   $y' = 7 x^{ 4 / 3 } - 13 x^{ 9 / 4 } - \frac{ 2 }{ 7 } x^{ -3 / 2 }
%   \textrm{.}$

% \item[--] Str. 438. 6.53.
%   $y' = x^{ -2 / 3 } - 3 x^{ 2 } + \frac{ 1 }{ 2 } \frac{ 1 }{ \sqrt[
%     4 ]{ x } }$.

% \item[--] Str. 438. 6.56. \ldots
%   $y' = \frac{ -5 }{ 7 \sqrt[ 7 ]{ x^{ 8 } } } - 14 x^{ 6 } - \frac{ 3
%   }{ 4 \sqrt{ x^{ 3 } } }$.

% \item[--] Str. 438. 6.89. \ldots
%   $z' = \frac{ -2 a x }{ ( a^{ 2 } + x^{ 2 } ) \sqrt{ a^{ 4 } - x^{ 4
%       } } }$.

% \item[--] Str. 440. 6.113. $\cos x \neq 0$,
%   $y' = \frac{ 7 \sin^{ 3 } x }{ \cos^{ 8 } x }$.

% \item[--] Str. 441. 6.129. $x > 1$,
%   $y' = \frac{ x \ln x }{ \sqrt{ ( x^{ 2 } - 1 )^{ 3 } } }$.

% \item[--] Str. 441. 6.131.
%   $y' = x^{ 4 } \arctan x + \frac{ x^{ 5 } - x }{ 5 ( 1 + x^{ 2 } ) }
%   - \frac{ 1 }{ 5 } x^{ 3 } + \frac{ 1 }{ 5 } x$.
% \item[--] Str. 478. \ldots
%   $I = \frac{ 1 }{ 3 } \ln | a^{ 3 } + x^{ 3 } |$.

% \item[--] Str. 480. 16.26. $I = \frac{ 1 }{ 8 } ( 2 x + 1 )^{ 4 }$.

% \item[--] Str. 480. 16.27. $x \neq \frac{ 2 }{ 3 }$;
%   $I = \frac{ -1 }{ 9 ( 3 x - 2 )^{ 3 } }$.

% \item[--] Str. 480. 16.37. $x \neq \frac{ 2 }{ 3 }$,
%   $x \neq \frac{ 3 }{ 2 }$;
%   $I = \frac{ 1 }{ 5 } \ln | \frac{ 2 x - 3 }{ 3 x - 2 } |$.

% \end{itemize}

% \vspace{\spaceTwo}





% % ##################
% \Work{ % Autorzy i tytuł dzieła
%   Włodzimierz Krysicki, L. Włodarski \\
%   ,,Analiza matematyczna w~zadaniach. Tom~II'',
%   \cite{KrysickiWlodarskiAnalizaWZadaniachTomII2004} }


% \CenterTB{Uwagi}

% \start \Str{199} \tb{Zadanie 7.2.} Rozwiązanie tego zadania jest
% niepełne, w~następujący sensie. Równanie numer (2) w~tym zadaniu,
% po~spierwiastkowaniu sprowadza~się do~równania:
% \begin{equation*}
%   \left| \dd{}{ y }{ x } \right| = \abso{ \cos x }.
% \end{equation*}
% Równania tego nie da~się przedstawić w~postaci Newtona, zaś poza
% równaniami (3) z~tego zadania, można uzyskać z~niego nieskończoną
% ilość innych, np. ograniczając~się do przedziału
% $( -\fr{ \pi }{ 2 }, \frac{ 3 \pi }{ 2 } )$ możemy rozpatrzyć:
% \begin{equation*}
%   \dd{}{ y }{ x } =
%   \begin{cases}
%     -\cos x, & x \in ( -\fr{ \pi }{ 2 }, \fr{ \pi }{ 2 } ), \\
%     \cos x, & x \in ( \fr{ \pi }{ 2 }, \fr{ 3 \pi }{ 2 } ).
%   \end{cases}
% \end{equation*}
% Dla dowolnych warunków początkowych w~rozpatrywanym przedziale, można
% znaleźć rozwiązania tego równania różniczkowalne w~całym przedziale.
% Na przykład dla $x_{ 0 } = \fr{ \pi }{ 2 }, y_{ 0 } = 0$, takim
% rozwiązaniem jest:
% \begin{equation*}
%   y( x )
%   \begin{cases}
%     -\sin x + 1, & x \in ( -\fr{ \pi }{ 2 }, \fr{ \pi }{ 2 } ), \\
%     \sin x - 1, & x \in ( \fr{ \pi }{ 2 }, \fr{ 3 \pi }{ 2 } ).
%   \end{cases}
% \end{equation*}

% \vspace{\spaceFour}


% \start \Str{200} \tb{Zadanie 7.3.} Również tutaj problem znalezienia
% wszystkich rozwiązań równania różniczkowego nie został w~pełni
% rozwiązany. Ograniczając~się do przedziału $( 0, 2 \pi )$ możemy
% zauważyć, że~funkcja:
% \begin{equation*}
%   y( x ) =
%   \begin{cases}
%     C_{ 1 } \sin x,& x \in ( 0, \pi ), \\
%     0, & x = \pi, \\
%     C_{ 2 } \sin x,& x \in ( \pi, 2 \pi ),
%   \end{cases}
% \end{equation*}
% jest rozwiązaniem równania (1) z~tego zadania w~całym badanym
% przedziale, choć nie jest różniczkowalne w~0.


% \CenterTB{Błędy}
% \begin{center}
%   \begin{tabular}{|c|c|c|c|c|}
%     \hline
%     & \multicolumn{2}{c|}{} & & \\
%     Strona & \multicolumn{2}{c|}{Wiersz} & Jest
%                               & Powinno być \\ \cline{2-3}
%     & Od góry & Od dołu & & \\
%     \hline
%     25  & &  7 & $( \sin z )^{ \tan z }$ & $( \sin x )^{ \tan z }$ \\
%     26  & 10 & & $3 \cdot 3$ & $9$ \\
%     26  & &  4 & $|\, \mr{n}$ & $\ln$ \\
%     26  &  2 & & $e^{ x^{ 2 } \sin(x - y^{ 2 }) }$
%            & $\sin^{ x^{ 2 } }(x - y^{ 2 })$ \\
%     26  &  1 & & $e^{ x^{ 2 } \sin(x - y^{ 2 }) }$
%            & $\sin^{ x^{ 2 } }(x - y^{ 2 })$ \\
%            % & & & & \\
%            % & & & & \\
%     201 & &  1 & $+$ & $=$ \\
%     203 &  7 & & $\fr{ a y }{ d x }$ & $\dd{}{ y }{ x }$ \\
%     203 &  7 & & sprawdzają & spełniają \\
%     207 &  2 & & $1 \fr{ 1 + \tfr{ 1 }{ C_{ 2 }^{ 2 } } }
%                  { x^{ 2 } + 1 }$
%            & $1 - \fr{ 1 + \fr{ 1 }{ C_{ 2 }^{ 2 } } }
%              { x^{ 2 } + 1 }$ \\
%              % & & & & \\
%     431 & &  3 & $x^{ \fr{ 1 }{ y - 1 } }$ & $x^{ \fr{ 1 }{ y } - 1 }$ \\
%     432 & &  9 & $\fr{\uppi R^{ 3 } }{ 3 }$ & $\fr{\uppi R^{ 2 } }{ 3 }$ \\
%     449 & &  9 & $C - e^{ \fr{ 1 }{ x } }$ & $C e^{ -\fr{ 1 }{ x } }$ \\
%     449 & &  9 & $c$ & $C$ \\
%     % & & & & \\
%     \hline
%   \end{tabular}
% \end{center}

% \begin{itemize}
% \item[--] Str. 432.
%   $\fr{ \pr^{ 2 } u }{ \pr x \pr y } = \fr{ \pr^{ 2 } u }{ \pr y \pr x
%   } = 6 x^{ 2 } - 30 x y^{ 2 } - 2 \sin 2y$.

% \item[--] Str. 454. 9.5. $y = C \frac{ x - 2 }{ x + 2 }$.

% \item[--] Str. 454. 9.5. $y = C x$.

%   % \item[--] Str.
%   % \item[--] Str.
%   % \item[--] Str.
%   % \item[--] Str.
%   % \item[--] Str.
% \end{itemize}

% \vspace{\spaceTwo}















% % ####################
% \Work{
%   W. Żakowski, W. Lesiński \\
%   ,,Matematyka. Część IV'' % \cite{ZL78}
% }


% \CenterTB{Uwagi}


% \noi \tb{Część~I.}

% \vspace{\spaceFour}


% \start Często w~tej części książki pojawia~się następująca sytuacja.
% W~wyniku rachunków otrzymaliśmy całkę ogólną pewnego równania
% różniczkowego $\vp( x, C ) = 0$, przy czym
% $C \in ( a, b ) \sum ( b, c )$. W~każdym rozważanym przypadku
% okazywało~się, że~funkcję $\vp( x, C )$ można w~naturalny sposób
% przedłużyć do~$C = b$ i~$\vp( x, b ) = $ również jest rozwiązaniem
% tego równania. Czy to jest zbieg okoliczności, czy~też przy pewnych
% warunkach musi to zachodzić? \Prze

% \vspace{\spaceThree}



% \noi \tb{Konkretne strony.}

% \vspace{\spaceFour}


% \start \Str{23} \Dok

% \start \Str{26} Rachunki prowadzące do równania (I.62) dowodzą,
% że~przedstawia ono rozwiązanie wyjściowego równania różniczkowego
% dla~każdej stałej $C_{ 2 }$ różnej od~zera. Aby~zasadnie twierdzić,
% że~jest to rozwiązanie tego równania dla~każdej stałej rzeczywistej
% $C_{ 2 }$ należy podstawić\footnote{W~tym momencie nie znam prostszego
%   rozwiązania tego problemu, ale~w~uwagach do części~I tej książki,
%   jest zawarta sugestia innego podejścia.} $C_{ 2 } = 0$, co prowadzi
% do~$u = 1 \pm \sqrt{2}$, i~sprawdzić, że~otrzymana w~ten sposób
% funkcja jest rozwiązaniem zadanego równania. Jak~się okazuje, tak
% w~istocie jest.

% % \CenterTB{Błędy}
% % \begin{center}
% %   \begin{tabular}{|c|c|c|c|c|}
% %     \hline
% %     & \multicolumn{2}{c|}{} & & \\
% %     Strona & \multicolumn{2}{c|}{Wiersz} & Jest
% %                               & Powinno być \\ \cline{2-3}
% %     & Od góry & Od dołu & & \\
% %     \hline
% %     %     & & & & \\
% %     %     & & & & \\
% %     %     & & & & \\
% %     \hline
% %   \end{tabular}
% % \end{center}


















% ####################
\Work{ % Autor i tytuł dzieła
  Izrail Moisjejewicz Gelfand, S.~W.~Fomin \\
  ,,Rachunek wariacyjny'', \cite{ } }



\CenterTB{Błędy}
\begin{center}
  \begin{tabular}{|c|c|c|c|c|}
    \hline
    & \multicolumn{2}{c|}{} & & \\
    Strona & \multicolumn{2}{c|}{Wiersz} & Jest
                              & Powinno być \\ \cline{2-3}
    & Od góry & Od dołu & & \\
    \hline
    25  & &  1 & $\lim\limits_{ \Delta x \to 0 }
                 \frac{ \Delta y' }{ \Delta x } = \tilde{ F }_{ y' y' }$
           & $\lim\limits_{ \Delta x \to 0 }
             \frac{ \Delta y' }{ \Delta x } \tilde{ F }_{ y' y' }$ \\
    27  &  7 & & $\frac{ d }{ dx }( F - y' F_{ y' } )$
           & $\frac{ d }{ dx }( F - y' F_{ y' } ) = 0$ \\
           % & & & & \\
    \hline
  \end{tabular}
\end{center}

\vspace{\spaceTwo}





% ####################
\Work{ % Autor i tytuł dzieła
  J. Jost, X. Li-Jost \\
  ,,Calculus of Variations'', \cite{} }


\CenterTB{Uwagi}

\start \Str{6} Dowód podstawowego lematu rachunku wariacyjnego jest
poprawny, acz mało elegancko zrobiony.


% \CenterTB{Błędy}
% \begin{tabular}{|c|c|c|c|c|}
%   \hline
%   & \multicolumn{2}{c|}{} & & \\
%   Strona & \multicolumn{2}{c|}{Wiersz} & Jest
%                             & Powinno być \\ \cline{2-3}
%   & Od góry & Od dołu & & \\
%   \hline
%   & & & & \\
%   & & & & \\
%   \hline
% \end{tabular}

\vspace{\spaceTwo}



\begin{center}
  Walter Rudin \\
  ,,Analiza rzeczywista i zespolona'', \cite{WRARZ}.
\end{center}


Uwagi:

\start \Str{132} Należy zauważyć, że wspomniana tu ,,miara
rzeczywista'' ma oznaczać miarę zespoloną (czyli miarę której
dziedzina zawiera się w ciele liczb zespolonych, bez nieskończoności)
która przyjmuje tylko wartości rzeczywiste.



Błędy:\\
\begin{center}
  \begin{tabular}{|c|c|c|c|c|}
    \hline
    & \multicolumn{2}{c|}{} & & \\
    Strona & \multicolumn{2}{c|}{Wiersz} & Jest
                              & Powinno być \\ \cline{2-3}
    & Od góry & Od dołu & & \\
    \hline
    % & & & & \\
    27  & &  3 & fnkcji & funkcji \\
    50  &  4 &  & Warunek (d) & Warunek (e) \\
    50  & 13 &  & że$K \prec f  \prec V$ & że $K \prec f  \prec V$ \\
    54  & & 16 & II i IV & II i VI \\
    54  & & 15 & że więc & więc \\
    64  & &  2 & $c_{ i } \chi_{ E_{ i } }$ & $c_{ i } \chi_{ V_{ i } }$ \\
    95  &  5 & & $| \varphi - x_{ n } |^{ 2 } \leq | \varphi |^{ 2 }$
           & $| \varphi - \hat{ x }_{ n } |^{ 2 } \leq | \varphi |^{ 2 }$ \\
    95  &  8 & & $|| \hat{ x }_{ n } - x_{ m }||_{ 2 }$
           & $|| \hat{ x }_{ n } - \hat{ x }_{ m }||_{ 2 }$ \\
    99  & &  7 & $|| f - P ||_{ 2 } < \infty$
           & $|| f - P ||_{ 2 } < \veps$ \\
    132 &  & 15 & miary rzeczywistej & miary zespolonej rzeczywistej \\
    140 & & 11 & Ponieważ $\lambda$ & Ponieważ $\Phi$ \\
    168 & &  5 & $\{ E_{ i } )$ & $\{ E_{ i } \}$ \\
    177 & &  5 & $= \int\limits^{ \infty }_{ -\infty } g( t ) dt \ldots$
           & $= \int\limits^{ x }_{ -\infty } g( t ) dt \ldots$ \\
           % & & & & \\
    \hline
  \end{tabular}
\end{center}


\begin{itemize}
\item[--] Str. 51.
  \begin{equation}
    \mu( K ) = \inf\{ \Lambda f : K \prec f \} \, .
    \tag{8}
  \end{equation}
\item[--] Str. 52.
  \begin{equation}
    \mu( K ) = \Lambda f .
    \tag{9}
  \end{equation}
\item[--] Str. 52. \ldots co w zestawieniu z nierównością (9) daje
  (8).
\item[--] Str. 165. \ldots dla \emph{każdego} ciągu
  $\{ E_{ i } \}$\ldots
\end{itemize}
