% Autor: Kamil Ziemian

% ---------------------------------------------------------------------
% Podstawowe ustawienia i pakiety
% ---------------------------------------------------------------------
\RequirePackage[l2tabu, orthodox]{nag}  % Wykrywa przestarzałe i niewłaściwe
% sposoby używania LaTeXa. Więcej jest w l2tabu English version.
\documentclass[a4paper,11pt]{article}
% {rozmiar papieru, rozmiar fontu}[klasa dokumentu]
\usepackage[MeX]{polski}  % Polonizacja LaTeXa, bez niej będzie pracował
% w języku angielskim.
\usepackage[utf8]{inputenc} % Włączenie kodowania UTF-8, co daje dostęp
% do polskich znaków.
\usepackage{lmodern}  % Wprowadza fonty Latin Modern.
\usepackage[T1]{fontenc}  % Potrzebne do używania fontów Latin Modern.



% ------------------------------
% Podstawowe pakiety (niezwiązane z ustawieniami języka)
% ------------------------------
\usepackage{microtype}  % Twierdzi, że poprawi rozmiar odstępów w tekście.
% \usepackage{graphicx}  % Wprowadza bardzo potrzebne komendy do wstawiania
% grafiki.
\usepackage{vmargin}  % Pozwala na prostą kontrolę rozmiaru marginesów,
% za pomocą komend poniżej. Rozmiar odstępów jest mierzony w calach.
% ------------------------------
% MARGINS
% ------------------------------
\setmarginsrb
{ 0.7in} % left margin
{ 0.6in} % top margin
{ 0.7in} % right margin
{ 0.8in} % bottom margin
{  20pt} % head height
{0.25in} % head sep
{   9pt} % foot height
{ 0.3in} % foot sep



% ------------------------------
% Często używane pakiety
% ------------------------------
% \usepackage{csquotes}  % Pozwala w prosty sposób wstawiać cytaty do tekstu.
% \usepackage{xcolor}  % Pozwala używać kolorowych czcionek (zapewne dużo
% więcej, ale ja nie potrafię nic o tym powiedzieć).



% ------------------------------
% Pakiety do tekstów z nauk przyrodniczych
% ------------------------------
\let\lll\undefined  % Amsmath gryzie się z pakietami do języka
% polskiego, bo oba definiują komendę \lll. Aby rozwiązać ten problem
% oddefiniowuję tę komendę, ale może tym samym pozbywam się dużego Ł.
\usepackage[intlimits]{amsmath}  % Podstawowe wsparcie od American
% Mathematical Society (w skrócie AMS)
\usepackage{amsfonts, amssymb, amscd, amsthm}  % Dalsze wsparcie od AMS
\usepackage{upgreek}  % Ładniejsze greckie litery
% Przykładowa składnia: pi = \uppi
\usepackage{calrsfs}  % Zmienia czcionkę kaligraficzną w \mathcal
% na ładniejszą. Może w innych miejscach robi to samo, ale o tym nic
% nie wiem.



% ---------------
% Wspaniały pakiet PGF/TikZ
% ---------------
\usepackage{tikz}

\usetikzlibrary{decorations.markings}  % Włączenie konkretnych bibliotek
% pakietu TikZ



% ---------------
% Tworzenie otoczeń "Twierdzenie", "Definicja", "Lemat", etc.
% ---------------
\newtheorem{theorem}{Twierdzenie}  % Komenda wprowadzająca otoczenie
% „theorem” do pisania twierdzeń matematycznych
\newtheorem{definition}{Definicja}  % Analogicznie jak powyżej
\newtheorem{corollary}{Wniosek}



% ------------------------------
% Pakiety których pliki *.sty mają być w tym samym katalogu co ten plik
% ------------------------------
\usepackage{latexgeneralcommands}
\usepackage{mathcommands}
\usepackage{calculuscommands}




% ---------------------------------------------------------------------
% Dodatkowe ustawienia dla języka polskiego
% ---------------------------------------------------------------------
\renewcommand{\thesection}{\arabic{section}.}
% Kropki po numerach rozdziału (polski zwyczaj topograficzny)
\renewcommand{\thesubsection}{\thesection\arabic{subsection}}
% Brak kropki po numerach podrozdziału



% ------------------------------
% Ustawienia różnych parametrów tekstu
% ------------------------------
\renewcommand{\arraystretch}{1.2}  % Ustawienie szerokości odstępów między
% wierszami w tabelach



% ------------------------------
% Pakiet „hyperref”
% Polecano by umieszczać go na końcu preambuły
% ------------------------------
\usepackage{hyperref}  % Pozwala tworzyć hiperlinki i zamienia odwołania
% do bibliografii na hiperlinki










% ---------------------------------------------------------------------
% Tytuł, autor, data
\title{Analiza matematyczna --~błędy i~uwagi}

% \author{}
% \date{}
% ---------------------------------------------------------------------










% ####################################################################
\begin{document}
% ####################################################################





% ######################################
\maketitle  % Tytuł całego tekstu
% ######################################





% ######################################
\section{XIX szkoła analizy matematycznej}

\vspace{\spaceTwo}
% ######################################



% ############################
\Work{ % Autor i tytuł dzieła
  Grigorij Michajłowicz Fichtenholz \\
  „Rachunek różniczkowy i~całkowy. Tom~I”,
  \cite{FichtenholzRachunekRozniczkowyETCVolI2005} }




Str. 303.
$\displaystyle \sum_{ i = 1 }^{ n } a_{ i }^{2} \sum_{ i = 1 }^{ n } b_{ i }^{
  2 }
- \{ \sum_{ i = 1 }^{ n } a_{ i } b_{ i } \}^{ 2 } \geq 0$, \\
Str. 371. $\displaystyle z'_{ x } = \frac{ x }{ p }\textrm{,} \quad
z'_{ y } = \frac{ y }{ q }\textrm{.}$ \\
Str. 376. \ldots ale równa jest 0, gdy\ldots


\vspace{\spaceTwo}
% ############################










% ############################
\Work{ % Autor i tytuł dzieła
  Grigorij Michajłowicz Fichtenholz \\
  „Rachunek różniczkowy i całkowy. Tom~II”,
  \cite{FichtenholzRachunekRozniczkowyETCVolII2004} }


% ##################
\CenterBoldFont{Uwagi do konkretnych stron}

\vspace{\spaceFour}


\start \Str{6} Pada tu~często przytaczanie stwierdzenie, że~dwie
funkcje pierwotne zadanej funkcji~$f( x )$ różnią~się o~stałą, jednak
rozumiane ściśle, nie jest ono prawdziwe. Aby~wyjaśnić ten problem,
rozpatrzmy funkcje funkcję rzeczywistą $f( x ) = 1 / x$, o~dziedzinie
$\Dcal( f ) = \Rbb \setminus \{ 0 \}$ oraz
\begin{subequations}
  \begin{align}
    \label{eq:FichtenholzVolII-01-A}
    F_{ 1 }( x ) = \ln\absOne{ x }, \quad
    & \Dcal( F_{ 1 } ) = \Rbb \setminus \{ 0 \}, \\
    \label{eq:FichtenholzVolII-01-B}
    F_{ 2 }( x ) = \ln\absOne{ x } + \frac{ x }{ \absOne{ x } }, \quad
    & \Dcal( F_{ 2 } ) = \Rbb \setminus \{ 0 \}.
  \end{align}
\end{subequations}
Ponieważ $\Dcal( x / \absOne{ x } ) = \Rbb \setminus \{ 0 \}$ i
\begin{equation}
  \label{eq:FichtenholzVolII-02}
  \frac{ x }{ \absOne{ x } } =
  \begin{cases}
    \hphantom{-}1, & x > 0, \\
    -1, & x < 0.
  \end{cases}
\end{equation}
$x / \absOne{ x } = 1$ dla~$x > 0$ i~$x / \absOne{ x } = -1$
dla~$x < 0$, więc zachodzi równość
\begin{equation}
  \label{eq:FichtenholzVolII-03}
  \frac{ d F_{ 1 }( x ) }{ dx } = \frac{ d F_{ 2 }( x ) }{ dx }
  = \frac{ 1 }{ x } = f( x ).
\end{equation}
Pochodne $F'_{ i }( x )$, $i = 1, 2$ również mają dziedzinę równą
$\Dcal( f ) = \Rbb \setminus \{ 0 \}$. Jednak funkcje $F_{ i }( x )$
nie różnią~się o~funkcje stałą
\begin{subequations}
  \begin{align}
    \label{eq:FichtenholzVolII-04-A}
    g( x ) &:= F_{ 2 }( x ) - F_{ 1 }( x ) = \frac{ x }{ \absOne{ x } }, \\
    \label{eq:FichtenholzVolII-04-B}
    g( -1 ) &\hphantom{:}= -1, \quad g( 1 ) = 1.
  \end{align}
\end{subequations}
Widzimy więc, iż~nie jest prawdą, że~dwie funkcje pierwotne różnią~się
o~stałą. W~tym twierdzeniu kryje~się jednak ziarno prawdy, co~zostanie
wyjaśnione poniżej.

Na~początku ustalany oznaczenia i~definicje. W~dalszym ciągu $I$
i~$I_{ \iota }$ oznaczać będą jeden z~następujących zbiorów
$( a, b )$, $[ a, b ]$, $[ a, b )$, $( a, b ]$, $( -\infty, a )$,
$( a, +\infty )$, $( -\infty, a ]$, $[ a, +\infty )$,
$( -\infty, +\infty )$, $\absOne{ a }, \absOne{ b } < +\infty$.
Funkcja $f( x ): I \to \Rbb$ jest różniczkowalna, jeśli jest
różniczkowalna we~wnętrzu zbioru~$I$ (czyli jednym z~odcinków
otwartych $( a, b )$, $( -\infty, a )$, $( a, +\infty )$), jeśli zaś
$I$ jest z~którejś strony domknięty, to~istnieje tam pochodna
jednostronna. $F( x ): I \to \Rbb$ jest funkcją pierwotną $f( x )$,
jeśli
\begin{equation}
  \label{eq:FichtenholzVolII-05}
  \frac{ d F( x ) }{ dx } = f( x ), \quad \forall x \in I.
\end{equation}

Udowadniam teraz ważne twierdzenie.





% #############
\begin{theorem}
  \label{thm:FichtenholzVolII-01}

  Każde dwie funkcje pierwotne funkcji $f: I \to \Rbb$ różnią~się
  o~funkcje stałą.

\end{theorem}



\begin{proof}

  Oznaczmy dowolne dwie funkcje pierwotne $f$ przez $F_{ 1 }$
  oraz~$F_{ 2 }$ i~oznaczmy $g( x ) := F_{ 1 }( x ) - F_{ 2 }( x )$.
  Zachodzi wówczas
  \begin{equation}
    \label{eq:FichtenholzVolII-06}
    \frac{ d g( x ) }{ dx } = \frac{ d }{ dx } ( F_{ 1 }( x ) - F_{ 2 }( x ) )
    = 0, \quad \forall x \in I.
  \end{equation}
  Rozważmy dwa dowolne punkty $x_{ 1 }, x_{ 2 } \in I$. Wraz z~nimi
  w~$I$ ~się cały odcinek $[ x_{ 1 }, x_{ 2 } ]$. Spełnione~są tym
  samym założenia twierdzenia o~wartości średniej, na~mocy którego
  \begin{equation}
    \label{eq:FichtenholzVolII-07}
    g( x_{ 2 } ) = g( x_{ 1 } ) + g'( c ) ( x_{ 2 } - x_{ 1 } )
    = g( x_{ 1 } ), \quad c \in ( x_{ 1 }, x_{ 2 } ).
  \end{equation}

\end{proof}
% #############





Przypomnijmy, że~dowolny zbiór w~$\Rbb$ można rozłożyć na~składowe
spójne (jest to szczególny przypadek ogólnego twierdzenia dla
przestrzeni topologicznych, zob.~str.~79,
\cite{SchwartzKursAnalizyMatematycznejVolI1979}). Wynika stąd
następujący wniosek.





% #############
\begin{corollary}
  \label{cor:FichtenholzVolII-01}

  Niech $A = \bigcup_{ \iota \in \Ical } I_{ \iota }$, przy czym każdy z~przedziałów
  $I_{ \iota }$ jest składową spójną $A$ i~niech $f: A \to \Rbb$
  będzie różniczkowalna na~każdym $I_{ \iota }$. Wówczas dwie funkcje
  pierwotne~$f$ różnią~się o~funkcje schodkową $g( x )$, która jest
  stała na~każdym $I_{ \iota }$.

\end{corollary}



\begin{proof}

  Ponieważ suma dwóch różnych $I_{ \iota }$ nie przedziałem zawartym
  w~$A$, pochodna jest dobrze określona w~każdym punkcie tego
  zbioru\footnote{W~przeciwnym razie mogłoby dojść do~sytuacji,
    że~$I_{ 1 } = (0, 1)$, $I_{ 2 } = [ 1, 2 )$, więc
    $( 0, 2 ) \subset A$, lecz funkcja nie jest wszędzie
    różniczkowalna, bo~pochodna w~$1$ nie istnieje.}, więc dla każdego
  $I_{ \iota }$ można zastosować twierdzenie
  \ref{thm:FichtenholzVolII-01}, skąd wynika, że~funkcja $G( x )$ jest
  stała na~tym zbiorze.

\end{proof}
% #############





Podany powyżej kontrprzykład pokazuje, że~silniejszy wynik niż dla
funkcji określonej na zbiorze $A$ nie może zachodzić. Widzimy teraz,
że~problem jest bardziej złożony, niż~się powszechnie twierdzi. Błąd
wynika zapewne stąd, iż~większość osób nie interesuje badanie tak
„subtelnego” przypadku, jakim jest dziedzina funkcji bardziej
skomplikowana niż~zbiór~$I$, choć przykład funkcji $1 / x$ pokazuje,
że~sytuacja ta jest spotykana w~konkretnych, prostych rachunkach.

Otwartym pozostaje pytanie, czy sam problem można postawić w~bardziej
ogólnej postaci? Po pierwsze twierdzenie o~wartości średniej wymaga by
funkcja była ciągła na przedziale $[ a, b ]$, a~różniczkowalna tylko
w~$( a, b )$ (zob.~punkt~112, str.~196,
\cite{FichtenholzRachunekRozniczkowyETCVolI2005}), co~potencjalnie
może prowadzić do~uogólnienia pojęcia funkcji pierwotnej. Po~drugie,
czy~można zdefiniować w~sposób sensowny pochodną funkcji na~zbiorze,
który nie jest postaci $\bigcup_{ \iota \in \Ical } I_{ \iota }$ (przy
zadanym ograniczeniu na~sumę dwóch $I_{ \iota }$)? Jeśli by tak było,
to pytanie o~postać funkcji pierwotnych jest otwarte, a~wynik ciężki
do~przewidzenia.

\vspace{\spaceFour}



\start \Str{34} Nie jestem w~stanie zrozumieć, dlaczego największym
wspólnym dzielnikiem wielomianów $Q$ i~$Q'$ jest $Q_{ 1 }$. Może to
jakiś znany fakt z~algebry? \Dok

\vspace{\spaceFour}



\start \textbf{Str.~223, punkt [363, 5)].} Problem z~tym przykładem
jest taki, że
\begin{equation}
  \label{eq:FichtenholzVolII-08}
  \sum_{ n = 1 }^{ \infty } \frac{ x^{ 2n - 1 } }{ 1 - x^{ 2n } }
  \neq \sum_{ n = 1 }^{ \infty } \left( \frac{ 1 }{ 1 - x^{ 2n - 1 } }
    - \frac{ 1 }{ 1 - x^{ 2n } } \right).
\end{equation}
Gdyby tak równość zachodziła, to po pierwsze
\begin{equation}
  \label{eq:FichtenholzVolII-09}
  \begin{split}
    \sum_{ i = 1 }^{ n } \frac{ x^{ 2i - 1 } }{ 1 - x^{ 2n } }
    &=
      \frac{ 1 }{ 1 - x } - \frac{ 1 }{ 1 - x^{ 2 } }
      + \frac{ 1 }{ 1 - x^{ 3 } } - \frac{ 1 }{ 1 - x^{ 4 } }
      + \ldots + \frac{ 1 }{ 1 - x^{ 2n - 1 } } - \frac{ 1 }{ 1 - x^{ 2n } } \\
    &\neq \frac{ 1 }{ 1 - x } - \frac{ 1 }{ 1 - x^{ 2n } }.
  \end{split}
\end{equation}
Wynika to z~tego, że~z~wyjątkiem $n = 1$
\begin{equation}
  \label{eq:FichtenholzVolII-10}
  \frac{ x^{ 2n - 1 } }{ 1 - x^{ 2n } }
  \neq \frac{ 1 }{ 1 - x^{ 2n - 1 } } - \frac{ 1 }{ 1 - x^{ 2n } }.
\end{equation}
Prawidłowa równość ma~postać
\begin{equation}
  \label{eq:FichtenholzVolII-11}
  \sum_{ n = 1 }^{ \infty } \frac{ x^{ 2n - 1 } ( 1 - x ) }{ ( 1 - x^{ 2n - 1 } )
    ( 1 - x^{ 2n } ) }
  \neq \sum_{ n = 1 }^{ \infty } \left( \frac{ 1 }{ 1 - x^{ 2n - 1 } }
    - \frac{ 1 }{ 1 - x^{ 2n } } \right),
\end{equation}
jednak postać wyrazu ogólnego szeregu po~lewej stronie tej równości
jest znacznie mniej elegancka i~zwięzła, niż~tego
w~\eqref{eq:FichtenholzVolII-08}, wątpliwe więc, że~Fichtenholz chciał
podać tu~ten przykład.

Wydaje mi~się, że~nie jest to kwestia drobnej literówki we~wzorze
\eqref{eq:FichtenholzVolII-08}, lecz~został tu~popełniony poważniejszy
błąd, którego nie jestem w~stanie poprawić.

\vspace{\spaceFour}



\start \Str{228} Twierdzenie~2 jest podane w~błędnej formie
i~udowodnione w~nie najlepszy sposób. Poniżej podana jest poprawna
wersja twierdzenie i~jaśniejsza, mam nadzieję, wersja dowodu.





% #############
\begin{theorem}
  \label{thm:FichtenholzVolII-02}

  Jeśli istnieje granica\footnote{Zakładamy przy tym, że~$b_{ n }$.}
  \begin{equation}
    \label{eq:FichtenholzVolII-12}
    \lim\limits_{ n \to \infty } \frac{ a_{ n } }{ b_{ n } } = K, \quad
    ( 0 \leq K \leq +\infty ),
  \end{equation}
  to, gdy $K < +\infty$, ze~zbieżności szeregu~(B) wynika zbieżność
  szeregu~(A), lub równoważnie, z~rozbieżności szeregu (A) wynika
  rozbieżność szeregu (B). Gdy natomiast $K > 0$, ze zbieżności
  szeregu (A) wynika zbieżność szeregu (B), lub~równoważnie,
  z~rozbieżności szeregu~(B) wynika rozbieżność szeregu~(A).

\end{theorem}



\begin{proof}
  Niech $K < +\infty$, wtedy istnieje takie~$N$, że~dla $n > N$
  \begin{equation}
    \label{eq:FichtenholzVolII-13}
    \frac{ a_{ n } }{ b_{ n } } < K + \epsilon, \quad
    K + \epsilon > 0,
  \end{equation}
  skąd
  \begin{equation}
    \label{eq:FichtenholzVolII-14}
    a_{ n } < ( K + \epsilon ) b_{ n }.
  \end{equation}
  Z~tej równości wynika, że~zbieżność szeregu~(B) pociąga za~sobą
  zbieżność~(A), jak również, iż~rozbieżność szeregu~(A) implikuje
  rozbieżność szeregu~(B). Oba twierdzenia~są jednak równoważne
  na~mocy zasady kontrapozycji.

  Analogicznie jeśli~$K > 0$, wówczas istnieje takie~$N$, że~dla
  $n > N$
  \begin{equation}
    \label{eq:FichtenholzVolII-15}
    a_{ n } > ( K - \epsilon ) b_{ n }, \quad
    K - \epsilon > 0.
  \end{equation}
  Z~tej nierówności wynika, że~zbieżność szeregu~(A) implikuje
  zbieżność szeregu~(B), zaś rozbieżność szeregu~(B) pociąga za~sobą
  rozbieżność szeregu~(A). Jak poprzednio, oba te~twierdzenia~są
  równoważne na~mocy zasady kontrapozycji.

\end{proof}
% #############


\vspace{\spaceFour}



\start \StrWg{237}{5} Aby~zachodził wzór
$\Dcal_{ n } = \tfrac{ x }{ n + 1 }$, należy numerować wyrazy tego
szeregu od~0.

\vspace{\spaceFour}





% ##################
\CenterBoldFont{Błędy}


\begin{center}

  \begin{tabular}{|c|c|c|c|c|}
    \hline
    & \multicolumn{2}{c|}{} & & \\
    Strona & \multicolumn{2}{c|}{Wiersz} & Jest
                              & Powinno być \\ \cline{2-3}
    & Od góry & Od dołu & & \\
    \hline
    8   & & 11 & $< | \Delta P | <$ & $\leq | \Delta P | \leq$ \\[0.2em]
    8   & &  7 & $< \frac{ | \Delta P | }{ \Delta x } <$
           & $\leq \frac{ | \Delta P | }{ \Delta x } \leq$ \\[0.2em]
    12  & &  5 & $a^{ n }$ & $a_{ n }$ \\
    13  &  7 & & $( x - a )^{ k } \, dx$ & $\int ( x - a )^{ -k } \, dx$ \\
    18  & &  4 & $\frac{ 1 }{ 3 }$ & $\frac{ 1 }{ 2 }$ \\
    18  & &  1 & więc$\cdot t$ & więc $t$ \\
    23  & 16 & & $n \_ 1$ & $n - 1$ \\
    23  & 17 & & $n \_ 2$ & $n - 2$ \\
    23  & & 16 & otrzvmujemy & otrzymujemy \\
    25  &  1 & & $e^{ \cdot ( k + 1 ) t}$ & $e^{ ( k + 1 ) t}$ \\
    28  & &  5 & z$\cdot$algebry & z~algebry \\[0.2em]
    29  & & 11 & $\frac{ P( x ) }{ ( x - a )^{ k - 1 } Q_{ 1 }( x ) }$
           & $\frac{ P_{ 1 }( x ) }{ ( x - a )^{ k - 1 } Q_{ 1 }( x ) }$
    \\[0.3em]
    34  &  3 & & [lub (6) & [lub (6)] \\
    34  &  4 & & lub (6)] & [lub (6)] \\[0.2em]
    35  & & 17 & $\left[ \frac{ a x^{ 2 } + b x + c }
                 { x^{ 3 } + x^{ 2 } + x + 1 } \right]$
           & $\left[ \frac{ a x^{ 2 } + b x + c }
             { x^{ 3 } + x^{ 2 } + x + 1 } \right]'$ \\[0.4em]
    35  & & 12 & $x^{ 2 } + x^{ 2 } + x + 1$
           & $x^{ 3 } + x^{ 2 } + x + 1$ \\
    36  & 14 & & $x^{ \dot{ 2 } }$ & $x^{ 2 }$ \\[0.2em]
    39  &  3 & & $\sqrt[ m ]{ \frac{ \alpha x + \beta }{ \gamma x + \delta } } \, dx$
           & $\sqrt[ m ]{ \frac{ \alpha x + \beta }{ \gamma x + \delta } }$ \\[0.2em]
    40  &  2 & & $\frac{ 2t - 1 }{ \sqrt{ 3 } }$
           & $\frac{ 2t + 1 }{ \sqrt{ 3 } }$ \\
    \hline
  \end{tabular}





  \begin{tabular}{|c|c|c|c|c|}
    \hline
    & \multicolumn{2}{c|}{} & & \\
    Strona & \multicolumn{2}{c|}{Wiersz} & Jest
                              & Powinno być \\ \cline{2-3}
    & Od góry & Od dołu & & \\
    \hline
    40  & & 12 & ${ m + 1 \atop n }$ & $\frac{ m + 1 }{ n }$ \\
    44  & &  4 & $\sqrt{ a x }$ & $\sqrt{ a } x$ \\
    44  & &  1 & $\sqrt{ a x }$ & $\sqrt{ a } x$ \\
    45  & & 13 & $2 \sqrt{ a } t - b$ & $2 \sqrt{ c } t - b$ \\
    46  & &  8 & $( 2 a x + b^{ 2 } )$ & $( 2 a x + b )^{ 2 }$ \\
    222 & &  5 & [9] jest rzeczywiście & jest oczywiście \\
    224 & & 17 & $A'_{ n + m }$ & $A'_{ n - m }$ \\
    226 & &  7 & $H_{ 2k }$ & $H_{ 2^{ k } }$ \\
    227 & 11 & & $2^{ k + 1 } + 1$ & $2^{ k - 1 } + 1$ \\
    229 & &  9 & $< \frac{ 1 }{ 2^{ n } }$ & $\leq \frac{ 1 }{ 2^{ n } }$ \\
    231 &  8 & & $1\; ;$ & $1;$ \\
    233 & &  5 & $\epsilon = \Ecal - 1$ & $\epsilon < \Ecal - 1$ \\
    233 & &  5 & $\Ecal - \epsilon = 1$ & $\Ecal - \epsilon > 1$ \\
    234 & 11 & & $\Dcal_{ n } > 1$ & $\Dcal_{ n } \geq 1$ \\
    238 & &  3 & $( 1 + x )^{ 1 / x } + \ln( 1 + x )$
           & $( 1 + x )^{ 1 / x } \ln( 1 + x )$ \\
           % & & & & \\
           % & & & & \\
           % & & & & \\
           % & & & & \\
           % & & & & \\
           % & & & & \\
           479 & 2 & & $\int\limits_{ 0 }^{ +\infty }$
           & $\int\limits_{ a }^{ +\infty }$ \\[0.8em]
           479 & 4 & & $\int\limits_{ a }^{ A } \frac{ dx }{ x }$
           & $\int\limits_{ a }^{ A } \frac{ dx }{ x^{ \lambda } }$ \\[0.5em]
           483 & 4 & & $I^{ 3 }$ & $I^{ 2 }$ \\
           % & & & & \\
    \hline
  \end{tabular}

\end{center}


\noindent
\StrWd{34}{3} \\
\Jest  Zróżniczkujmy (10) obustronnie \\
\Powin Wykonując jawnie różniczkowanie w~(10) \\
\StrWd{71}{16} \\
\Jest  $R\left( x, \sqrt{ a x^{ 4 } + b x^{ 3 } + c x^{ 2 } + d x + e }
\right.$ \\
\Powin $R\left( x, \sqrt{ a x^{ 4 } + b x^{ 3 } + c x^{ 2 } + d x + e }
\right)$ \\

\vspace{\spaceTwo}
% ############################










% ######################################
\section{Współczesnej podejście do analizy matematycznej}

\vspace{\spaceTwo}
% ######################################



% ############################
\Work{ % Autor i tytuł dzieła
  Krzysztof Maurin \\
  „Analiza. Część~I: Elementy”, \cite{MaurinAnalizaVolI1974} }


% ##################
\CenterBoldFont{Uwagi}


\start Tyle błędów, że szybciej będzie napisać tę
książkę od nowa.

\vspace{\spaceFour}



\start Paragraf poświęcony całce Riemanna, jest napisany nie
najlepiej. Nie udowodniono wszystkich ważnych twierdzeń, nawet tych z
których korzysta. Ponadto dowody są w dużej mierze przeprowadzone
bardzo pobieżnie i niewiele mówią czytelnikowi.

\vspace{\spaceFour}



\start Kwestia analityczności funkcji została tu~potraktowana
pobieżnie. Dla funkcji takich jak $\cos$, $\sin$ etc. pokazano ich
analityczność w~0, nic nie powiedziano o analityczności w innych
punktach.

\vspace{\spaceFour}



\start W ogóle nie przedyskutowano problemu mnożenia szeregów.

\vspace{\spaceFour}



\start Większość z definicji i twierdzeń Rozdziału VII da się uogólnić
dla przypadku przestrzeni unormowanych.

\vspace{\spaceFour}




% ##################
\CenterBoldFont{Uwagi do konkretnych stron}


\start \Str{10} Wszystkie dowody „nie wprost”, są przypisane
zasadzie \\
kontrapozycji.

\vspace{\spaceFour}



\start \Str{21} Użyte jest pojęcie subtelniejszego podziału, pomimo że
nie zostało ono zdefiniowane.

\vspace{\spaceFour}



\start \Str{40} Twierdzenie tu zapowiedziane nie zostało nigdy
udowodnione.

\vspace{\spaceFour}



\start \Str{41} Uwaga napisana jest fatalnie. Nawet jeżeli to
twierdzenie jest prawdziwe, co nie jest oczywiste, to sformułowanie,
zaciemnia cały problem.

\vspace{\spaceFour}



\start \Str{43} W drugiej części dowodu Wniosku II.5 zupełnie
niepotrzebnie wprowadzono kule otwarte zawarte w $Z_{ 1 }$ i~$Z_{ 2 }$.

\vspace{\spaceFour}



\start \Str{48} Dowód pierwszej części Twierdzenia II. 18 jest
zupełnie niezrozumiały.

\vspace{\spaceFour}



\start \Str{53} Czy w twierdzeniu o pochodnej funkcji odwrotnej jest
ważna ciągłość tej funkcji?

\vspace{\spaceFour}



\start \Str{56} Dyskusja skierowania w rodzinie zbiorów $\Pi$,
bardziej zaciemnia niż wyjasnia. Po przeanalizowaniu okazuje się, że w
definicji $\limsup$ powinno być: \\
$A_{ i } \prec A_{ j }\Leftrightarrow A_{ j } \subset A_{ i }$, choć z
dyskusji wynika wręcz przeciwnie.

\vspace{\spaceFour}



\start \Str{57} Nie zdefiniowano na jakie odcinki dzielimy przedział
$[ a, b ]$. Okazuje się, że nie ma to znaczenia jeśli przyjmiemy miarę
dowolnego odcinka $\mu ( \{ a, b \} ) = b - a$, jednak ta
lekkomyślność jest rażąca.

\vspace{\spaceFour}



\start \Str{63} W~dowodzie Twierdzenia.III.13 „potrzeba i nie
potrzeba” założenia $\varphi' \neq 0$, zależnie od rozpatrywanego
przypadku.

\vspace{\spaceFour}



\start \Str{66} Skorzystano tu z twierdzenia dla liczb rzeczywistych,
które zostało udowodnione tylko dla liczb całkowitych.

\vspace{\spaceFour}



\start \Str{68} Dowód korzysta z pojęcia bazy otoczeń domkniętych,
które nie zostało nigdzie wcześniej przedstawione.

\vspace{\spaceFour}



\start \Str{90} Podane są warunki na to by ciąg $T_{ n }$ był ciągiem
Cauchy’ego, nie zaś by było on zbieżny do $T_{ 0 }$.

\vspace{\spaceFour}



\start \Str{90} W Lemat V.5. zawiera poważne pomieszanie pojęć.
Prawidłowe sformułowanie powinno brzmieć: ????

\vspace{\spaceFour}



\start \Str{92} W dowodzie Lematu V.6 użyte są kule otwarte a powinny
być domknięte.

\vspace{\spaceFour}



\start \Str{93} Dowód testu Weierstrassa dla funkcji jest bez sensu.
Dowodzi tylko punktowej zbieżności szeregu funkcji. Zbieżności
jednostajnej należy dowodzić inną metodą.

\vspace{\spaceFour}



\start \Str{97} Dowód punktu (b) uwagi łatwo przeprowadzić przez
kontrapozycje, nie widzę jednak sposobu udowodnienia go w sposób
analogiczny do punktu (a). W dowodzie tym kluczowe jest oszacowanie \\
$| a_{ n } ( z_{ 1 } - z_{ 0 } )^{ n } | < M$. Jednak dla szeregu o niezerowym
promieniu zbieżności nie być możliwe oszacowanie szeregu od góry albo
od dołu. Za przykład może posłużyć zachowanie szeregów
$\sum \frac{ 1 }{ n } z^{ n }$ i~$\sum n z^{ n }$ w~$z = 1$. Oba są tam
rozbieżne

\vspace{\spaceFour}



\start \Str{98} Dowód twierdzenia Cauchy’ego-Hadamarda zawiera w sobie
dowód kryterium Cauchy’ego zbieżności szeregów. Czyni to dowód bardzo
zagmatwanym i~trudnym do zrozumienia.





% ##################
\CenterBoldFont{Błędy}


\begin{center}

  \begin{tabular}{|c|c|c|c|c|}
    \hline
    & \multicolumn{2}{c|}{} & & \\
    Strona & \multicolumn{2}{c|}{Wiersz} & Jest
                              & Powinno być \\ \cline{2-3}
    & Od góry & Od dołu & & \\
    \hline
    & & & & \\
    \hline
  \end{tabular}

\end{center}


\noindent
Str. 23. \ldots 2$^{ \circ }$ identytywna\ldots \\
Str. 28. \textit{Ad} 2$^{ \circ }$: $| ( a'_{ n } + b'_{ n } )
- ( a_{ n } + b_{ n } ) |  \leq | a'_{ n } - a_{ n } | + | b'_{ n } - b_{ n } |
< 2 \, \varepsilon$, bo \\
$\big( ( a'_{ n } ) \sim ( a_{ n } ) \wedge( b'_{ n } ) \sim ( b_{ n } ) \big)
\Rightarrow \big( ( | a'_{ n } - a_{ n } | < \epsilon )
\wedge | b'_{ n } - b_{ n } | < \epsilon, \textrm{ dla } n > N ) \big)$. \\
Str. 32. \\
Str. 43. \ldots wynika z definicji odwzorowania ciągłego. \\
Str. 54. Niech $g'( x ) h + r_{ 1 }( h ) \neq 0$\ldots \\
Str. 97. $t := \frac{ | z - z_{ 0 } | }{ | z_{ 1 } - z_{ 0 } | }
\textrm{\ldots}$ \\
Str. 112.
$\big( 0 \leq f \leq C \frac{ 1 }{ x^{ \mu } } \textrm{ ; } \mu \geq 1 \big)
\Rightarrow \big( \int \limits^{ \infty }_{ a } f < \infty \big) \textrm{;}$ \\
Str. 112.
$\big( 0 \leq f \leq C \frac{ 1 }{ x ( \log x )^{ \mu } } \textrm{ ; } \mu \geq 1 \big)
\Rightarrow \big( \int \limits^{ \infty }_{ a } f < \infty \big) \textrm{;}$ \\
Str. 125. \ldots \\
Str. 143. \ldots korzystając ze wzorów (4) i (5)\ldots


\vspace{\spaceTwo}
% ############################










% ############################
\Work{ % Autor i tytuł dzieła
  Krzysztof Maurin \\
  „Analiza. Część II: Ogólne struktury, funkcje algebraiczne, \\
  całkowanie, analiza tensorowa”, \cite{} }


% ##################
\CenterBoldFont{Uwagi do konkretnych stron}

\Str{90} W~równaniu
\begin{equation}
  \label{eq:MaurinAnalizaOgolneStrukturyVolII-01}
  \rho( v )^{ 2 } = Q( v ) \cdot 1,
\end{equation}
Symbol „1” po prawej stronie oznacza jedynkę w algebrze $\Ccal( V )$. W~tekście nie zostało to w ogóle zaznaczone.





% ##################
\CenterBoldFont{Błędy}

\begin{center}

  \begin{tabular}{|c|c|c|c|c|}
    \hline
    & \multicolumn{2}{c|}{} & & \\
    Strona & \multicolumn{2}{c|}{Wiersz} & Jest
                              & Powinno być \\ \cline{2-3}
    & Od góry & Od dołu & & \\
    \hline
    24  & 17 & & $\exists_{ \substack{ W \in \Bcal( x ) } }$
           & $\exists_{ \substack{ W \in \Bcal( y ) } }$ \\
    26  & 16 & & $\Leftarrow:$ & $\Leftarrow$ (a.a.). \\
    90  & & 17 & nieskończenie & niekoniecznie \\
    90  & &  8 & $V = ( V, B )$ & $V = ( V, Q )$ \\
    90  & &  8 & algebra & algebra z jedynką \\
    92  & & 17 & $2Q( e_{ i }, e_{ j } )$ & $B( e_{ i }, e_{ j } )$ \\
    99  &  3 & & $K( a )$ & $\mathbf{K}( a )$ \\
    % & & & & \\
    % & & & & \\
    % & & & & \\
    % & & & & \\
    \hline
  \end{tabular}

\end{center}

\vspace{\spaceTwo}
% ############################










% ############################
\Work{ % Autor i tytuł dzieła
  Walter Rudin \\
  „Analiza rzeczywista i zespolona”,
  \cite{RudinAnalizaRzeczywistaIZespolona1998} }


% ##################
\CenterBoldFont{Uwagi do konkretnych stron}


\start \Str{132} Należy zauważyć, że wspomniana tu „miara
rzeczywista” ma oznaczać miarę zespoloną (czyli miarę której
dziedzina zawiera się w ciele liczb zespolonych, bez nieskończoności)
która przyjmuje tylko wartości rzeczywiste.



% ##################
\CenterBoldFont{Błędy}


\begin{center}

  \begin{tabular}{|c|c|c|c|c|}
    \hline
    & \multicolumn{2}{c|}{} & & \\
    Strona & \multicolumn{2}{c|}{Wiersz} & Jest
                              & Powinno być \\ \cline{2-3}
    & Od góry & Od dołu & & \\
    \hline
    % & & & & \\
    27  & &  3 & fnkcji & funkcji \\
    50  &  4 &  & Warunek (d) & Warunek (e) \\
    50  & 13 &  & że$K \prec f  \prec V$ & że $K \prec f  \prec V$ \\
    54  & & 16 & II i IV & II i VI \\
    54  & & 15 & że więc & więc \\
    64  & &  2 & $c_{ i } \chi_{ E_{ i } }$ & $c_{ i } \chi_{ V_{ i } }$ \\
    95  &  5 & & $| \varphi - x_{ n } |^{ 2 } \leq | \varphi |^{ 2 }$
           & $| \varphi - \hat{ x }_{ n } |^{ 2 } \leq | \varphi |^{ 2 }$ \\
    95  &  8 & & $\Vert \hat{ x }_{ n } - x_{ m } \Vert_{ 2 }$
           & $\Vert \hat{ x }_{ n } - \hat{ x }_{ m } \Vert_{ 2 }$ \\
    99  & &  7 & $\Vert f - P \Vert_{ 2 } < \infty$
           & $\Vert f - P \Vert_{ 2 } < \varepsilon$ \\
    132 &  & 15 & miary rzeczywistej & miary zespolonej rzeczywistej \\
    140 & & 11 & Ponieważ $\lambda$ & Ponieważ $\Phi$ \\
    168 & &  5 & $\{ E_{ i } )$ & $\{ E_{ i } \}$ \\
    177 & &  5 & $= \int\limits^{ \infty }_{ -\infty } g( t ) \, dt \ldots$
           & $= \int\limits^{ x }_{ -\infty } g( t ) \, dt \ldots$ \\
           % & & & & \\
    \hline
  \end{tabular}

\end{center}


Str. 51. $\mu( K ) = \inf\{ \Lambda f : K \prec f \}$. \\
Str. 52. $\mu( K ) = \Lambda f$. \\
Str. 52. \ldots co w zestawieniu z nierównością (9) daje (8). \\
Str. 165. \ldots dla \textit{każdego} ciągu $\{ E_{ i } \}$\ldots

\vspace{\spaceTwo}
% ############################










% #############################
\Work{ % Autorzy i tytuł dzieła
  W. Żakowski, W. Lesiński \\
  „Matematyka. Część IV”, \cite{ZakowskiLeksinskiLMatematykaVolIV1978} }


% ##################
\CenterBoldFont{Uwagi}


\noindent
\textbf{Część~I.}

% \vspace{\spaceFour}


\start Często w~tej części książki pojawia~się następująca sytuacja.
W~wyniku rachunków otrzymaliśmy całkę ogólną pewnego równania
różniczkowego $\varphi( x, C ) = 0$, przy czym
$C \in ( a, b ) \sum ( b, c )$. W~każdym rozważanym przypadku
okazywało~się, że~funkcję $\varphi( x, C )$ można w~naturalny sposób
przedłużyć do~$C = b$ i~$\varphi( x, b ) = $ również jest rozwiązaniem
tego równania. Czy to jest zbieg okoliczności, czy~też przy pewnych
warunkach musi to zachodzić? \Prze

\vspace{\spaceThree}





% ##################
\CenterBoldFont{Uwagi do konkretnych stron}

\vspace{\spaceFour}


\start \Str{23} \Dok

\vspace{\spaceFour}



\start \Str{26} Rachunki prowadzące do równania (I.62) dowodzą,
że~przedstawia ono rozwiązanie wyjściowego równania różniczkowego
dla~każdej stałej $C_{ 2 }$ różnej od~zera. Aby~zasadnie twierdzić,
że~jest to rozwiązanie tego równania dla~każdej stałej rzeczywistej
$C_{ 2 }$ należy podstawić\footnote{W~tym momencie nie znam prostszego
  rozwiązania tego problemu, ale~w~uwagach do części~I tej książki,
  jest zawarta sugestia innego podejścia.} $C_{ 2 } = 0$, co prowadzi
do~$u = 1 \pm \sqrt{2}$, i~sprawdzić, że~otrzymana w~ten sposób
funkcja jest rozwiązaniem zadanego równania. Jak~się okazuje, tak
w~istocie jest.





% ##################
\CenterBoldFont{Błędy}


\begin{center}

  \begin{tabular}{|c|c|c|c|c|}
    \hline
    & \multicolumn{2}{c|}{} & & \\
    Strona & \multicolumn{2}{c|}{Wiersz} & Jest
                              & Powinno być \\ \cline{2-3}
    & Od góry & Od dołu & & \\
    \hline
    %     & & & & \\
    %     & & & & \\
    %     & & & & \\
    \hline
  \end{tabular}

\end{center}

% ############################










% ######################################
\newpage
\section{Rachunek wariacyjny}

\vspace{\spaceTwo}
% ######################################



% ############################
\Work{ % Autorzy i tytuł dzieła
  I.M.~Gelfand, S.W.~Fomin \\
  „Rachunek wariacyjny”, \cite{GelfandFominRachunekWariacyjny1979} }


% ##################
\CenterBoldFont{Błędy}


\begin{center}

  \begin{tabular}{|c|c|c|c|c|}
    \hline
    & \multicolumn{2}{c|}{} & & \\
    Strona & \multicolumn{2}{c|}{Wiersz} & Jest
                              & Powinno być \\ \cline{2-3}
    & Od góry & Od dołu & & \\
    \hline
    25  & &  1 & $\lim\limits_{ \Delta x \to 0 }
                 \frac{ \Delta y' }{ \Delta x } = \tilde{ F }_{ y' y' }$
           & $\lim\limits_{ \Delta x \to 0 }
             \frac{ \Delta y' }{ \Delta x } \tilde{ F }_{ y' y' }$ \\
    27  &  7 & & $\frac{ d }{ dx }( F - y' F_{ y' } )$
           & $\frac{ d }{ dx }( F - y' F_{ y' } ) = 0$ \\
           % & & & & \\
    \hline
  \end{tabular}

\end{center}

\vspace{\spaceTwo}
% ############################










% ############################
\Work{ % Autorzy i tytuł dzieła
  J. Jost, X. Li-Jost \\
  „Calculus of Variations”, \cite{JostLiJostCalculusOfVariations1998} }


% ##################
\CenterBoldFont{Uwagi}


\start \Str{6} Dowód podstawowego lematu rachunku wariacyjnego jest
poprawny, acz mało elegancko zrobiony.


% % ##################
% \CenterBoldFont{Błędy}


% \begin{center}

%   \begin{tabular}{|c|c|c|c|c|}
%     \hline
%     & \multicolumn{2}{c|}{} & & \\
%     Strona & \multicolumn{2}{c|}{Wiersz} & Jest
%                               & Powinno być \\ \cline{2-3}
%     & Od góry & Od dołu & & \\
%     \hline
%     & & & & \\
%     & & & & \\
%     \hline
%   \end{tabular}

% \end{center}


\vspace{\spaceTwo}
% ############################










% ######################################
\newpage
\section{Teoria funkcji rzeczywistych}

\vspace{\spaceTwo}
% ######################################



% ############################
\Work{ % Autor i tytuł dzieła
  Stanisław Łojasiewicz \\
  „Wstęp do~teorii funkcji rzeczywistych”,
  \cite{LojasiewiczWstepDoTeoriiFunkcjiRzeczywistych1976} }


% ##################
\CenterBoldFont{Uwagi do konkretnych stron}


\start \Str{7} Dla pełniejszego wykładu, warto tu przytoczyć pozostałe
operacje z~udziałem $\pm \infty$.
\begin{equation}
  \label{eq:LojasiewiczWDTFRz-01}
  \begin{split}
    &a + ( \pm \infty ) = ( \pm \infty ) + a = \pm \infty, \\
    &a \cdot ( \pm \infty ) = ( \pm \infty ) \cdot a = \pm \infty,
    \quad a > 0, \\
    &a \cdot ( \pm \infty ) = ( \pm \infty ) \cdot a = \mp \infty,
    \quad a < 0, \\
    &( +\infty ) \cdot ( +\infty ) = ( -\infty ) \cdot ( -\infty )
    = +\infty, \\
    &( +\infty ) \cdot ( -\infty ) = -\infty.
  \end{split}
\end{equation}

\vspace{\spaceFour}



\start \Str{7} Iloczyn w~$\overline{ \Rbb }$ nie jest ciągły
dla~$x = \pm \infty$ i~$y = 0$ (odpowiednio $x = 0$
i~$y = \pm \infty$), bo jeśli weźmiemy $x_{ n } = n$, $y_{ n } = 1 / n$,
to
\begin{equation}
  \label{eq:LojasiewiczWDTFRz-02}
  \lim\limits_{ n \to \infty } x_{ n } y_{ n }
  = \lim\limits_{ n \to \infty } 1 = 1 \neq 0 = ( +\infty ) \cdot 0.
\end{equation}
Suma nie jest ciągła dla~$x = +\infty$ i~$y = -\infty$ (odpowiednio
$x = -\infty$ i~$y = +\infty$), bo~jeśli weźmiemy $x_{ n } = 1 + n$
i~$y_{ n } = -n$, to
\begin{equation}
  \label{eq:LojasiewiczWDTFRz-03}
  \lim\limits_{ n \to \infty } ( x_{ n } + y_{ n } )
  = \lim\limits_{ n \to \infty } 1 = 1 \neq 0 = +\infty - \infty.
\end{equation}
Mnożenie nie jest łączne, bo
\begin{equation}
  \label{eq:LojasiewiczWDTFRz-04}
  \begin{split}
    &( a + \infty ) - \infty = \infty - \infty = 0, \\
    &a + ( \infty - \infty ) = a + 0 = a.
  \end{split}
\end{equation}
Mnożenie nie jest rozłączne względem dodawania, bo
\begin{equation}
  \label{eq:LojasiewiczWDTFRz-05}
  \begin{split}
    &( +\infty ) \cdot ( a - \infty ) = ( +\infty \cdot a ) - \infty
    = +\infty - \infty = 0, \\
    &( +\infty ) \cdot ( a - \infty ) = ( +\infty ) \cdot ( -\infty )
    = -\infty.
  \end{split}
\end{equation}

\vspace{\spaceFour}



\start \Str{7} Stwierdzenie, że~przez brak łączności dodawania nie
możemy przenosić wyrazów z~jednej strony równania na drugą, może
wydawać~się nieoczywiste, dlatego tutaj wyjaśnimy to na przykładzie.
Rozpatrzmy równość
\begin{equation}
  \label{eq:LojasiewiczWDTFRz-06}
  1 + \infty = \infty.
\end{equation}
Teraz chcielibyśmy odjąć od~obu stron $\infty$ i~wykonać obliczenia
w~następujący sposób
\begin{equation}
  \label{eq:LojasiewiczWDTFRz-07}
  \begin{split}
    &\textrm{L} = 1 + \infty - \infty = 1 + 0 = 1, \\
    &\textrm{P} = \infty - \infty = 0.
  \end{split}
\end{equation}
Błąd wynika z~tego, że~tak naprawdę wyrażenie po~lewej stronie ma
następującą postać
\begin{equation}
  \label{eq:LojasiewiczWDTFRz-08}
  \textrm{L} = ( 1 + \infty ) - \infty.
\end{equation}
Chcielibyśmy przekształcić ten wzór w~następujący sposób
\begin{equation}
  \label{eq:LojasiewiczWDTFRz-09}
  1 + ( \infty - \infty ) = 1 + 0 = 1.
\end{equation}
Tego jednak zrobić nie możemy ze~względu na~brak łączności dodawania.
Zignorowanie tego prowadzi do~błędnych wyników powyżej.

\vspace{\spaceFour}





% ##################
\CenterBoldFont{Błędy}


\begin{center}

  \begin{tabular}{|c|c|c|c|c|}
    \hline
    & \multicolumn{2}{c|}{} & & \\
    Strona & \multicolumn{2}{c|}{Wiersz} & Jest
                              & Powinno być \\ \cline{2-3}
    & Od góry & Od dołu & & \\
    \hline
    9   & 15 & & $\absOne{ \alpha }$ $\absOne{ \beta }$
           & $\absOne{ \alpha }$, $\absOne{ \beta }$ \\
    11  & &  5 & $\{ \alpha_{ n } \}$ & $\{ x_{ n } \}$ \\
    % & & & & \\
    % & & & & \\
    % & & & & \\
    % & & & & \\
    \hline
  \end{tabular}

\end{center}

\vspace{\spaceTwo}
% ############################










% ######################################
\section{Książki o~obliczeniach}

\vspace{\spaceTwo}
% ######################################



% ############################
\Work{ % Autor i tytuł dzieła
  P. J. Nahin \\
  „Inside Interesting Integrals”,
  \cite{NahinInterestingIntegrals2015} }


% ##################
\CenterBoldFont{Błędy}

\begin{center}

  \begin{tabular}{|c|c|c|c|c|}
    \hline
    & \multicolumn{2}{c|}{} & & \\
    Strona & \multicolumn{2}{c|}{Wiersz} & Jest
                              & Powinno być \\ \cline{2-3}
    & Od góry & Od dołu & & \\
    \hline
    %     & & & & \\
    %     & & & & \\
    \hline
  \end{tabular}

\end{center}


\noindent
\StrWg{vi}{1} \\
\Jest  \textbf{\textit{This book is~dedicated to all who, when they read
    the~following line form John le~Carr\'{e}'s 1989 Cold War spy
    novel}} \\
\Powin \textbf{This book is~dedicated to all who, when they read
  the~following line form John le~Carr\'{e}'s 1989 Cold War spy
  novel} \\
\StrWg{vi}{8} \\
\Jest  \textbf{\textit{as well as to all who understand how frustrating is
    the~lament in Anthony Zee's books}} \\
\Powin \textbf{as well as to all who understand how frustrating is
  the~lament in Anthony Zee's books} \\

\vspace{\spaceTwo}
% ############################










% ####################################################################
% ####################################################################
% Bibliografia
\bibliographystyle{plalpha}

\bibliography{DEUSPhilBooks,MathComScienceBooks}{}





% ############################

% Koniec dokumentu
\end{document}
