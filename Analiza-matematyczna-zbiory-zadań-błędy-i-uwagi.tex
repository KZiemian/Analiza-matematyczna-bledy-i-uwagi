% ---------------------------------------------------------------------
% Podstawowe ustawienia i pakiety
% ---------------------------------------------------------------------
\RequirePackage[l2tabu, orthodox]{nag}  % Wykrywa przestarzałe i niewłaściwe
% sposoby używania LaTeXa. Więcej jest w l2tabu English version.
\documentclass[a4paper,11pt]{article}
% {rozmiar papieru, rozmiar fontu}[klasa dokumentu]
\usepackage[MeX]{polski}  % Polonizacja LaTeXa, bez niej będzie pracował
% w języku angielskim.
\usepackage[utf8]{inputenc} % Włączenie kodowania UTF-8, co daje dostęp
% do polskich znaków.
\usepackage{lmodern}  % Wprowadza fonty Latin Modern.
\usepackage[T1]{fontenc}  % Potrzebne do używania fontów Latin Modern.



% ------------------------------
% Podstawowe pakiety (niezwiązane z ustawieniami języka)
% ------------------------------
\usepackage{microtype}  % Twierdzi, że poprawi rozmiar odstępów w tekście.
\usepackage{graphicx}  % Wprowadza bardzo potrzebne komendy do wstawiania
% grafiki.
% \usepackage{verbatim}  % Poprawia otoczenie VERBATIME.
% \usepackage{textcomp}  % Dodaje takie symbole jak stopnie Celsiusa,
% wprowadzane bezpośrednio w tekście.
\usepackage{vmargin}  % Pozwala na prostą kontrolę rozmiaru marginesów,
% za pomocą komend poniżej. Rozmiar odstępów jest mierzony w calach.
% ------------------------------
% MARGINS
% ------------------------------
\setmarginsrb
{ 0.7in}  % left margin
{ 0.6in}  % top margin
{ 0.7in}  % right margin
{ 0.8in}  % bottom margin
{  20pt}  % head height
{0.25in}  % head sep
{   9pt}  % foot height
{ 0.3in}  % foot sep



% ------------------------------
% Często przydatne pakiety
% ------------------------------
% \usepackage{csquotes}  % Pozwala w prosty sposób wstawiać cytaty do tekstu.
% \usepackage{xcolor}  % Pozwala używać kolorowych czcionek (zapewne dużo
% więcej, ale ja nie potrafię nic o tym powiedzieć).



% ------------------------------
% Pakiety do tekstów z nauk przyrodniczych
% ------------------------------
\let\lll\undefined  % Amsmath gryzie się z pakietami do języka
% polskiego, bo oba definiują komendę \lll. Aby rozwiązać ten problem
% oddefiniowuję tę komendę, ale może tym samym pozbywam się dużego Ł.
\usepackage[intlimits]{amsmath}  % Podstawowe wsparcie od American
% Mathematical Society (w skrócie AMS)
\usepackage{amsfonts, amssymb, amscd, amsthm}  % Dalsze wsparcie od AMS
\usepackage{bm}  % Daję komendę \bm do pogrubionej czcionki matematycznej
% Gryzie się z innymi czcionkami
% \usepackage{siunitx}  % Do prostszego pisania jednostek fizycznych
\usepackage{upgreek}  % Ładniejsze greckie litery
% Przykładowa składnia: pi = \uppi
% \usepackage{slashed}  % Pozwala w prosty sposób pisać slash Feynmana.
\usepackage{calrsfs}  % Zmienia czcionkę kaligraficzną w \mathcal
% na ładniejszą. Może w innych miejscach robi to samo, ale o tym nic
% nie wiem.
\usepackage{tikz}  % Potężny pakiet PGF/TikZ.
\usetikzlibrary{decorations.markings}  % Włączenie konkretnych bibliotek
% pakietu TikZ



% ---------------
% Tworzenie otoczeń "Twierdzenie", "Definicja", "Lemat", etc.
% ---------------
\newtheorem{theorem}{Twierdzenie}  % Komenda wprowadzająca otoczenie
% „theorem” do pisania twierdzeń matematycznych
\newtheorem{definition}{Definicja}  % Analogicznie jak powyżej
\newtheorem{corollary}{Wniosek}



% ------------------------------
% Pakiety których pliki *.sty mają być w tym samym katalogu co ten plik
% ------------------------------
\usepackage{latexgeneralcommands}
\usepackage{mathcommands}





% ---------------------------------------------------------------------
% Dodatkowe ustawienia dla języka polskiego
% ---------------------------------------------------------------------
\renewcommand{\thesection}{\arabic{section}.}
% Kropki po numerach rozdziału (polski zwyczaj topograficzny)
\renewcommand{\thesubsection}{\thesection\arabic{subsection}}
% Brak kropki po numerach podrozdziału



% ------------------------------
% Ustawienia różnych parametrów tekstu
% ------------------------------
\renewcommand{\arraystretch}{1.2}  % Ustawienie szerokości odstępów między
% wierszami w tabelach.



% ------------------------------
% Pakiet „hyperref”
% Polecano by umieszczać go na końcu preambuły
% ------------------------------
\usepackage{hyperref}  % Pozwala tworzyć hiperlinki i zamienia odwołania
% do bibliografii na hiperlinki










% ---------------------------------------------------------------------
% Tytuł, autor, data
\title{Zbiory zadań z~analizy matematycznej \\
  Błędy i~uwagi}

\author{Kamil Ziemian}


% \date{}
% ---------------------------------------------------------------------










% ####################################################################
\begin{document}
% ####################################################################





% ######################################
\maketitle  % Tytuł całego tekstu
% ######################################





% ######################################
\newpage
\section{Zbiory zadań z~analizy matematycznej}

\vspace{\spaceTwo}
% ######################################



% ############################
\Work{
  Włodzimierz Krysicki, L. Włodarski \\
  „Analiza matematyczna w zadaniach. Tom~I”,
  \cite{KrysickiWlodarskiAnalizaWZadaniachVolI2005}}


% ##################
\CenterBoldFont{Uwagi}


\start \Str{89} Problemy takie jak policzenie granicy
$\lim\limits_{ x \to 0 } \frac{ \sin 3 x }{ x }$, sugerują następujące
postępowanie. Wiemy, że~$\lim\limits_{ x \to 0 } 3x = 0$ oraz
że~$\lim\limits_{ x \to 0 } \frac{ \sin x }{ x } = 1$, chcielibyśmy
przekształcić wyrażenie do postaci $3 \frac{ \sin 3x }{ 3x }$
i~skorzystać z~tego, że~$\lim\limits_{ x \to 0 } \frac{ \sin 3x }{ 3x } = 0$,
pytanie jedna czy ta ostania równość zachodzi? Pozytywną odpowiedź
na~to~pytanie daje poniższe twierdzenie.

Funkcje tu użyte wciąż nie są zdefiniowane

\vspace{\spaceFour}



\start \Str{96} Można
zauważyć, że~funkcje $\artanh$ i~$\arcoth$ mają taką samą pochodną,
więc powinny różnią~się o~stałą, co~jest nonsensem. Wyjaśnieniem
problemu jest fakt, że~te dwie funkcje~są określone na rozłącznych
dziedzinach.

\vspace{\spaceFour}



\start \Str{125} Warto przedyskutować, dlaczego tak ważnej jest
założenie o~tym, że~$\frac{ dx }{ dt } \neq 0$. Jeżeli mamy dane dwie
funkcje $y( t )$ i~$x( t )$, to~nie musi istnieć funkcja $y( x )$.
Jeżeli jednak dla jakiegoś $t_{ 0 }$ mamy
$\frac{ dx }{ dt }( t_{ 0 } ) \neq 0$ to~na mocy twierdzenie
o~odwracaniu funkcji klasy $\Ccal^{ k }$ istnieje\footnote{Nie wiem czy
  założenie o~klasie różniczkowalności jest konieczne, bowiem
  w~przypadku funkcji rzeczywistej jednej zmiennej istnieje wiele
  wariantów twierdzenia o~funkcji uwikłanej i~funkcji odwrotnej, więc
  może~się okazać, iż~jeden z~nich pozwala osłabić ten warunek.
  Szeroką dyskusję tych twierdzeń można znaleźć w~książce Fichtenholza
  \cite{FichtenholzRachunekRozniczkowyETCVolI2005}.} funkcja $t( x )$ w~pewnym
otoczeniu $t_{ 0 }$ i~w~tym otoczeniu $y( x ) = y\big( t( x ) \big)$.

Co~jednak dzieje~się, gdy~$\frac{ dx }{ dt }( t_{ 0 } ) = 0$, czy
funkcja $y( x )$ wówczas nie istnieje? Następujący przypadek pokazuje,
że~tak nie musi być. Rozpatrzmy funkcje $x( t ) = t^{ 3 }$,
$y( t ) = t$. Pomimo, że~$\frac{ dx }{ dt }( 0 ) = 0$~to, można to
zauważyć np. rysując wykresy obu funkcji, można odwikłać $t( x )$
i~wówczas
\begin{equation}
  \label{eq:KrysickiWlodarskiAnalizaVolI-01}
  y( x ) =
  \begin{cases}
    \hphantom{-}\sqrt[ 1 / 3 ]{ x } & x \geq 0, \\
    -\sqrt[ 1 / 3 ]{ x } & x < 0.
  \end{cases}
\end{equation}
Funkcja ta jednak nie jest różniczkowalna dla $x = 0$. Nie potrafię
stwierdzić, czy znikanie pochodnej $\frac{ dx }{ dt }( t_{ 0 } )$ musi
pociągać za sobą, że~jeśli funkcja $y( x )$ istnieje to jest
nieróżniczkowalna po $x$ w~punkcie $x( t_{ 0 } )$. Wątpię jednak,
aby~tak było.

\vspace{\spaceFour}



\start \Str{130} Autorzy popełnili tu błąd przyjmując, że~moduł
pochodnej równy jest pochodnej modułu:
\begin{equation*}
  \label{eq:KrysickiWlodarskiAnalizaVolI-02}
  \left| \frac{ df }{ dt } \right| =
  \frac{ d \absOne{ f } }{ dt }.
\end{equation*}
Że~jest inaczej można~się przekonać rozważając znany przykłady ruchu
jednostajnego po okręgu, gdzie prędkość ma stałą długość, a~jednak
moduł przyśpieszenia nie jest równy 0, lecz $\frac{ v^{ 2 } }{ \rho }$,
gdzie $\rho$ to promień krzywizny.

Przeprowadzając proste obliczenia dostajemy poprawne wzory na składowe
i~moduł przyśpieszenia:
\begin{subequations}
  \begin{align}
    \label{eq:KrysickiWlodarskiAnalizaVolI-03-A}
    a_{ x } &= 50 ( \cos 5t^{ 2 } - 100 t^{ 2 } \sin 5t^{ 2 } ), \\
    \label{eq:KrysickiWlodarskiAnalizaVolI-03-B}
    a_{ y } &= -50 ( \sin 5t^{ 2 } + 100 t^{ 2 } \cos 5t^{ 2 } ), \\
    \label{eq:KrysickiWlodarskiAnalizaVolI-03-C}
    a &= 50 \sqrt{ 1 + 100 t^{ 4 } }.
  \end{align}
\end{subequations}

Widzimy więc, że~moduł rośnie w~czasie. Należało~się tego spodziewać,
bowiem przyśpieszenie normalne do toru dane jest wzorem
$\frac{ v^{ 2 } }{ \rho }$, więc jeśli prędkość rośnie liniowo, to ta
składowa przyśpieszenia również musi rosnąć.





% ##################
\CenterBoldFont{Błędy}


\begin{center}

  \begin{tabular}{|c|c|c|c|c|}
    \hline
    & \multicolumn{2}{c|}{} & & \\
    Strona & \multicolumn{2}{c|}{Wiersz} & Jest
                              & Powinno być \\ \cline{2-3}
    & Od góry & Od dołu & & \\
    \hline
    46  &  8 & & $\sqrt[n]{ u_{ n } } < p$ & $\sqrt[n]{ u_{ n } } \leq p$ \\
    79  & &  8 & $-b \, a$ & $-b / a$ \\
    80  & 10 & & $= 2$ & $x = 2$ \\
    88  & &  2 & $\frac{ x^{ 2 } - 1 }{ x - 2 }$
           & $\frac{ x^{ 2 } - 4 }{ x - 2 }$ \\
    89  &  5 & & $\frac{ ( x - 3 )( -1 )^{ [ x ] } }{ x^{ 2 } - 9 }$
           & $\frac{ ( x + 3 )( -1 )^{ [ x ] } }{ x^{ 2 } - 9 }$ \\
    96  & &  3 & $\frac{ -1 }{ 1 - x^{ 2 } }$
           & $\frac{ 1 }{ 1 - x^{ 2 } }$ \\
    116 &  2 & & $\left[ 2 \tan \frac{ x }{ 3 } + 1 \right)$
           & $ \left( 2 \tan \frac{ x }{ 3 } + 1 \right)$ \\
    129 & &  5 & $a = 796$ & $a = 7.96$ \\
    231 & 11 & & $\bigg| \sum_{ k = 0 }^{ n } u( x ) - S( x ) < \bigg|
                 \varepsilon$
           & $\bigg| \sum_{ k = 0 }^{ n } u( x ) - S( x ) \bigg|
             < \varepsilon$ \\
    234 & & 11 & $\sin{ 1 \atop n }$ & $\sin \frac{ 1 }{ n }$ \\
    241 & &  6 & $f( 0 )$ & $f'( 0 )$ \\
    241 & &  4 & $2^{ 3 }$ & $2^{ 2 }$ \\
    256 & &  4 & $\frac{ f'( x ) }{ {}'( x ) }$
           & $\frac{ f'( x ) }{ g'( x ) }$ \\[0.4em]
    258 & &  4 & ${ 1 \atop t^{ 2 } }$ & $\frac{ 1 }{ t^{ 2 } }$ \\
    259 &  8 & & oraz \textit{istnieje} & \textit{ale istnieje} \\
    441 & &  7 & $\arctan^{ 3 }x$ & $\arctan x^{ 3 }$ \\
    442 & &  6 & $\frac{ 1 }{ ( 1 - x ) \sqrt{ x } }$
           & $\frac{ 1 }{ x \ln x \ln( \ln x ) }$ \\
    436 & &  6 & $-4$ & $-2$ \\
    % & & & & \\
    % & & & & \\
    % & & & & \\
    \hline
  \end{tabular}

\end{center}



Str. 325. 16.32.
$\int \frac{ 6 x - 13 }{ x^{ 2 } - \frac{ 7 }{ 2 } x + \frac{ 3 }{ 2 } }
\, dx$.

Str. 378. 19.15.
$\int\limits_{ 0 }^{ a } 3x \sqrt{ x^{ 2 } + 4 a^{ 2 } } \, dx$, $a > 0$.

Str. 436. 5.38. $\frac{ 2 }{ 3 }$.

Str. 438. 6.50.
$y' = 7 x^{ 4 / 3 } - 13 x^{ 9 / 4 } - \frac{ 2 }{ 7 } x^{ -3 / 2 }$.

Str. 438. 6.53.
$y' = x^{ -2 / 3 } - 3 x^{ 2 } + \frac{ 1 }{ 2 } \frac{ 1 }{ \sqrt[ 4 ]{ x } }$.

Str. 438. 6.56. \ldots
$y' = \frac{ -5 }{ 7 \sqrt[ 7 ]{ x^{ 8 } } } - 14 x^{ 6 }
- \frac{ 3 }{ 4 \sqrt{ x^{ 3 } } }$.

Str. 438. 6.89. \ldots
$z' = \frac{ -2 a x }{ ( a^{ 2 } + x^{ 2 } ) \sqrt{ a^{ 4 } - x^{ 4 } } }$.

Str. 440. 6.113. $\cos x \neq 0$, $y' = \frac{ 7 \sin^{ 3 } x }{ \cos^{ 8 } x }$.

Str. 441. 6.129. $x > 1$,
$y' = \frac{ x \ln x }{ \sqrt{ ( x^{ 2 } - 1 )^{ 3 } } }$.

Str. 441. 6.131.
$y' = x^{ 4 } \arctan x + \frac{ x^{ 5 } - x }{ 5 ( 1 + x^{ 2 } ) }
- \frac{ 1 }{ 5 } x^{ 3 } + \frac{ 1 }{ 5 } x$.

Str. 478. \ldots $I = \frac{ 1 }{ 3 } \ln| a^{ 3 } + x^{ 3 } |$.

Str. 480. 16.26. $I = \frac{ 1 }{ 8 } ( 2 x + 1 )^{ 4 }$.

Str. 480. 16.27. $x \neq \frac{ 2 }{ 3 }$;
$I = \frac{ -1 }{ 9 ( 3 x - 2 )^{ 3 } }$.

Str. 480. 16.37. $x \neq \frac{ 2 }{ 3 }$, $x \neq \frac{ 3 }{ 2 }$;
$I = \frac{ 1 }{ 5 } \ln| \frac{ 2 x - 3 }{ 3 x - 2 } |$.


\vspace{\spaceTwo}
% ############################










% ##########################
\Work{ % Autorzy i tytuł dzieła
  Włodzimierz Krysicki, L. Włodarski \\
  „Analiza matematyczna w~zadaniach. Tom~II”,
  \cite{KrysickiWlodarskiAnalizaWZadaniachVolII2004} }


% ##################
\CenterBoldFont{Uwagi do konkretnych stron}


\start \Str{199} \textbf{Zadanie 7.2.} Rozwiązanie tego zadania jest
niepełne, w~następujący sensie. Równanie numer (2) w~tym zadaniu,
po~spierwiastkowaniu sprowadza~się do~równania:
\begin{equation}
  \label{eq:KrysickiWlodarskiAnalizaVolII-01}
  \left| \frac{ dy }{ dx } \right| = \absOne{ \cos x }.
\end{equation}
Równania tego nie da~się przedstawić w~postaci Newtona, zaś poza
równaniami (3) z~tego zadania, można uzyskać z~niego nieskończoną
ilość innych, np. ograniczając~się do przedziału
$( -\frac{ \pi }{ 2 }, \frac{ 3 \pi }{ 2 } )$ możemy rozpatrzyć:
\begin{equation}
  \label{eq:KrysickiWlodarskiAnalizaVolII-02}
  \frac{ dy }{ dx } =
  \begin{cases}
    -\cos x, & x \in ( -\frac{ \pi }{ 2 }, \frac{ \pi }{ 2 } ), \\
    \hphantom{-}\cos x,
             & x \in ( \frac{ \pi }{ 2 }, \frac{ 3 \pi }{ 2 } ).
  \end{cases}
\end{equation}
Dla dowolnych warunków początkowych w~rozpatrywanym przedziale, można
znaleźć rozwiązania tego równania różniczkowalne w~całym przedziale.
Na przykład dla $x_{ 0 } = \frac{ \pi }{ 2 }$, $y_{ 0 } = 0$, takim
rozwiązaniem jest:
\begin{equation}
  \label{eq:KrysickiWlodarskiAnalizaVolII-03}
  y( x ) =
  \begin{cases}
    -\sin x + 1, & x \in ( -\frac{ \pi }{ 2 }, \frac{ \pi }{ 2 } ), \\
    \hphantom{-}\sin x - 1,
                 & x \in ( \frac{ \pi }{ 2 }, \frac{ 3 \pi }{ 2 } ).
  \end{cases}
\end{equation}

\vspace{\spaceFour}



\start \Str{200} \textbf{Zadanie 7.3.} Również tutaj problem znalezienia
wszystkich rozwiązań równania różniczkowego nie został w~pełni
rozwiązany. Ograniczając~się do przedziału $( 0, 2 \pi )$ możemy
zauważyć, że~funkcja:
\begin{equation}
  \label{eq:KrysickiWlodarskiAnalizaVolII-04}
  y( x ) =
  \begin{cases}
    C_{ 1 } \sin x,& x \in ( 0, \pi ), \\
    0, & x = \pi, \\
    C_{ 2 } \sin x,& x \in ( \pi, 2 \pi ),
  \end{cases}
\end{equation}
jest rozwiązaniem równania (1) z~tego zadania w~całym badanym
przedziale, choć nie jest różniczkowalne w~0.





% ##################
\CenterBoldFont{Błędy}


\begin{center}

  \begin{tabular}{|c|c|c|c|c|}
    \hline
    & \multicolumn{2}{c|}{} & & \\
    Strona & \multicolumn{2}{c|}{Wiersz} & Jest
                              & Powinno być \\ \cline{2-3}
    & Od góry & Od dołu & & \\
    \hline
    25  & &  7 & $( \sin z )^{ \tan z }$ & $( \sin x )^{ \tan z }$ \\
    26  & 10 & & $3 \cdot 3$ & $9$ \\
    26  & &  4 & $|\, \mathrm{n}$ & $\ln$ \\
    26  &  2 & & $e^{ x^{ 2 } \sin( x - y^{ 2 } ) }$
           & $\sin^{ x^{ 2 } }( x - y^{ 2 } )$ \\
    26  &  1 & & $e^{ x^{ 2 } \sin( x - y^{ 2 } ) }$
           & $\sin^{ x^{ 2 } }( x - y^{ 2 } )$ \\
           % & & & & \\
           % & & & & \\
    201 & &  1 & $+$ & $=$ \\
    203 &  7 & & $\frac{ a y }{ d x }$ & $\frac{ dy }{ dx }$ \\
    203 &  7 & & sprawdzają & spełniają \\
    207 &  2 & & $1 \frac{ 1 +
                 \tfrac{ 1 }{ C_{ 2 }^{ 2 } } } { x^{ 2 } + 1 }$
           & $1 - \frac{ 1 +
             \frac{ 1 }{ C_{ 2 }^{ 2 } } }{ x^{ 2 } + 1 }$ \\
             % & & & & \\
    431 & &  3 & $x^{ \frac{ 1 }{ y - 1 } }$ & $x^{ \frac{ 1 }{ y } - 1 }$ \\
    432 & &  9 & $\frac{ \pi R^{ 3 } }{ 3 }$ & $\frac{ \pi R^{ 2 } }{ 3 }$ \\
    449 & &  9 & $C - e^{ \frac{ 1 }{ x } }$ & $C e^{ -\frac{ 1 }{ x } }$ \\
    449 & &  9 & $c$ & $C$ \\
    % & & & & \\
    \hline
  \end{tabular}

\end{center}

Str. 432.
$\frac{ \partial^{ 2 } u }{ \partial x \partial y }
= \frac{ \partial^{ 2 } u }{ \partial y \partial x }
= 6 x^{ 2 } - 30 x y^{ 2 } - 2 \sin 2y$. \\
Str. 454. 9.5. $y = C \frac{ x - 2 }{ x + 2 }$. \\
Str. 454. 9.5. $y = C x$.

\vspace{\spaceTwo}
% ############################










% ######################################
\newpage
\section{Zbiory zadań z~analizy zespolonej}

\vspace{\spaceTwo}
% ######################################



% ####################
\Work{ % Autor i tytuł dzieła
  Jak Krzyż \\
  „Zbiór zadań z~funkcji analitycznych”,
  \cite{KrzyzZbiorZadanZFunkcjiAnalitycznych2005} }


% ##################
\CenterBoldFont{Uwagi}


\start \StrWg{28}{18} Sens tego niezbyt zgrabnie zbudowanego zdanie
jest prosty. Mając dany dowolny niezerowy kąt\footnote{Czy to
  założenie jest potrzebne?} o~wierzchołku $z = 0$, możemy dobrać
takie $\alpha$, że~funkcja $z^{ \alpha }$ przekształca ten kąt na~kąt półpełny.

\vspace{\spaceFour}



\start \StrWg{31}{14} Stwierdzeni, że~całka
\begin{equation}
  \label{eq:KrzyzZZzFA-01}
  \int_{ \gamma } f( z ) \, dz
\end{equation}
niezależny od parametryzacji, choć zrozumiałe jest niezręczne. Już
następna część zdania przypomina, że~parametryzacji zmieniające
orientację zmienia jej znak na przeciwny. Należałoby więc napisać,
iż~całka ta nie zależy od~wyboru parametryzacji zachowującej
orientację.

\vspace{\spaceFour}


% \start

% \vspace{\spaceFour}


% \start

% \vspace{\spaceFour}


% \start

% \vspace{\spaceFour}


% \start

% \vspace{\spaceFour}


% \start

% \vspace{\spaceFour}


% \start

% \vspace{\spaceFour}


% ##################
\CenterBoldFont{Błędy}


\begin{center}

  \begin{tabular}{|c|c|c|c|c|}
    \hline
    & \multicolumn{2}{c|}{} & & \\
    Strona & \multicolumn{2}{c|}{Wiersz} & Jest
                              & Powinno być \\ \cline{2-3}
    & Od góry & Od dołu & & \\
    \hline
    28  & 18 & & kąt dowolny & dowolny kąt \\
    66  & 20 & & dwu & dwóch \\
    73  & &  2 & w~skończoności & skończonego \\
    % & & & & \\
    % & & & & \\
    % & & & & \\
    % & & & & \\
    % & & & & \\
    % & & & & \\
    % & & & & \\
    \hline
  \end{tabular}





  % \begin{tabular}{|c|c|c|c|c|}
  %   \hline
  %   & \multicolumn{2}{c|}{} & & \\
  %   Strona & \multicolumn{2}{c|}{Wiersz} & Jest
  %   & Powinno być \\ \cline{2-3}
  %   & Od góry & Od dołu & & \\
  %   \hline
  %   %   & & & & \\
  %   %   & & & & \\
  %   %   & & & & \\
  %   %   & & & & \\
  %   %   & & & & \\
  %   \hline
  % \end{tabular}

\end{center}

% \noi \\
% \StrWg{}{} \\
% \Jest \Powin
% \StrWg{}{} \\
% \Jest \Powin
% \StrWg{}{} \\
% \Jest \Powin
% \StrWg{}{} \\
% \Jest \Powin
% \StrWg{}{} \\
% \Jest \Powin
% \StrWg{}{} \\
% \Jest \Powin
% \StrWg{}{} \\
% \Jest \Powin

\vspace{\spaceTwo}
% ############################










% ####################################################################
% ####################################################################
% Bibliografia
\bibliographystyle{plalpha}

\bibliography{MathComScienceBooks}{}





% ############################

% Koniec dokumentu
\end{document}
