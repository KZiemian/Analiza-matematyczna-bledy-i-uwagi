% Autor: Kamil Ziemian

% --------------------------------------------------------------------
% Podstawowe ustawienia i pakiety
% --------------------------------------------------------------------
\RequirePackage[l2tabu, orthodox]{nag}  % Wykrywa przestarzałe i niewłaściwe
% sposoby używania LaTeXa. Więcej jest w l2tabu English version.
\documentclass[a4paper,11pt]{article}
% {rozmiar papieru, rozmiar fontu}[klasa dokumentu]
\usepackage[MeX]{polski}  % Polonizacja LaTeXa, bez niej będzie pracował
% w języku angielskim.
\usepackage[utf8]{inputenc} % Włączenie kodowania UTF-8, co daje dostęp
% do polskich znaków.
\usepackage{lmodern}  % Wprowadza fonty Latin Modern.
\usepackage[T1]{fontenc}  % Potrzebne do używania fontów Latin Modern.



% ----------------------------
% Podstawowe pakiety (niezwiązane z ustawieniami języka)
% ----------------------------
\usepackage{microtype}  % Twierdzi, że poprawi rozmiar odstępów w tekście.
\usepackage{graphicx}  % Wprowadza bardzo potrzebne komendy do wstawiania
% grafiki.
% \usepackage{verbatim}  % Poprawia otoczenie VERBATIME.
% \usepackage{textcomp}  % Dodaje takie symbole jak stopnie Celsiusa,
% wprowadzane bezpośrednio w tekście.
\usepackage{vmargin}  % Pozwala na prostą kontrolę rozmiaru marginesów,
% za pomocą komend poniżej. Rozmiar odstępów jest mierzony w calach.
% ----------------------------
% MARGINS
% ----------------------------
\setmarginsrb
{ 0.7in} % left margin
{ 0.6in} % top margin
{ 0.7in} % right margin
{ 0.8in} % bottom margin
{  20pt} % head height
{0.25in} % head sep
{   9pt} % foot height
{ 0.3in} % foot sep



% ------------------------------
% Często przydatne pakiety
% ------------------------------
% \usepackage{csquotes}  % Pozwala w prosty sposób wstawiać cytaty do tekstu.
% \usepackage{xcolor}  % Pozwala używać kolorowych czcionek (zapewne dużo
% więcej, ale ja nie potrafię nic o tym powiedzieć).



% ------------------------------
% Pakiety do tekstów z nauk przyrodniczych
% ------------------------------
\let\lll\undefined  % Amsmath gryzie się z pakietami do języka
% polskiego, bo oba definiują komendę \lll. Aby rozwiązać ten problem
% oddefiniowuję tę komendę, ale może tym samym pozbywam się dużego Ł.
\usepackage[intlimits]{amsmath}  % Podstawowe wsparcie od American
% Mathematical Society (w skrócie AMS)
\usepackage{amsfonts, amssymb, amscd, amsthm}  % Dalsze wsparcie od AMS
\usepackage{bm}  % Daję komendę \bm do pogrubionej czcionki matematycznej
% Gryzie się z innymi czcionkami
% \usepackage{siunitx}  % Do prostszego pisania jednostek fizycznych
\usepackage{upgreek}  % Ładniejsze greckie litery
% Przykładowa składnia: pi = \uppi
% \usepackage{slashed}  % Pozwala w prosty sposób pisać slash Feynmana.
\usepackage{calrsfs}  % Zmienia czcionkę kaligraficzną w \mathcal
% na ładniejszą. Może w innych miejscach robi to samo, ale o tym nic
% nie wiem.
\usepackage{tikz}  % Potężny pakiet PGF/TikZ.
\usetikzlibrary{decorations.markings}  % Włączenie konkretnych bibliotek
% pakietu TikZ



% ----------
% Tworzenie otoczeń "Twierdzenie", "Definicja", "Lemat", etc.
% ----------
\newtheorem{twr}{Twierdzenie}  % Komenda wprowadzająca otoczenie
% ,,twr'' do pisania twierdzeń matematycznych
\newtheorem{defin}{Definicja}  % Analogicznie jak powyżej
\newtheorem{wni}{Wniosek}



% ----------------------------
% Pakiety napisane przez użytkownika.
% Mają być w tym samym katalogu to ten plik .tex
% ----------------------------
\usepackage{analizamatematyczna}  % Pakiet napisany między innymi
% dla tego pliku.
\usepackage{latexshortcuts}
\usepackage{mathshortcuts}




% --------------------------------------------------------------------
% Dodatkowe ustawienia dla języka polskiego
% --------------------------------------------------------------------
\renewcommand{\thesection}{\arabic{section}.}
% Kropki po numerach rozdziału (polski zwyczaj topograficzny)
\renewcommand{\thesubsection}{\thesection\arabic{subsection}}
% Brak kropki po numerach podrozdziału



% ----------------------------
% Ustawienia różnych parametrów tekstu
% ----------------------------
\renewcommand{\arraystretch}{1.2}  % Ustawienie szerokości odstępów między
% wierszami w tabelach.



% ----------------------------
% Pakiet "hyperref"
% Polecano by umieszczać go na końcu preambuły.
% ----------------------------
\usepackage{hyperref}  % Pozwala tworzyć hiperlinki i zamienia odwołania
% do bibliografii na hiperlinki.





% --------------------------------------------------------------------
% Tytuł, autor, data
\title{Zbiory zadań z~analizy matematycznej --~błędy i~uwagi}

% \author{}
% \date{}
% --------------------------------------------------------------------





% ####################################################################
\begin{document}
% ####################################################################



% ######################################
\maketitle  % Tytuł całego tekstu
% ######################################



% % ######################################
% \section{Analiza matematyczna}

% \vspace{\spaceTwo}
% % ######################################











% ######################################
\newpage
\section{Zbiory zadań z~analizy zespolonej}

\vspace{\spaceTwo}
% ######################################



% ####################
\Work{ % Autor i tytuł dzieła
  Jak Krzyż \\
  ,,Zbiór zadań z~funkcji analitycznych'',
  \cite{KrzyzZbiorZadanZFunkcjiAnalitycznych2005} }


\CenterTB{Uwagi}

\start \StrWg{28}{18} Sens tego niezbyt zgrabnie zbudowanego zdanie
jest prosty. Mając dany dowolny niezerowy kąt\footnote{Czy to
  założenie jest potrzebne?} o~wierzchołku $z = 0$, możemy dobrać
takie $\alpha$, że~funkcja $z^{ \alpha }$ przekształca ten kąt na~kąt półpełny.

\vspace{\spaceFour}


\start \StrWg{31}{14} Stwierdzeni, że~całka
\begin{equation}
  \label{eq:Krzyz-01}
  \int_{ \gamma } f( z )\, dz
\end{equation}
niezależny od parametryzacji, choć zrozumiałe jest niezręczne. Już
następna część zdania przypomina, że~parametryzacji zmieniające
orientację zmienia jej znak na przeciwny. Należałoby więc napisać,
iż~całka ta nie zależy od~wyboru parametryzacji zachowującej
orientację.

\vspace{\spaceFour}


% \start

% \vspace{\spaceFour}


% \start

% \vspace{\spaceFour}


% \start

% \vspace{\spaceFour}


% \start

% \vspace{\spaceFour}


% \start

% \vspace{\spaceFour}


% \start

% \vspace{\spaceFour}


\CenterTB{Błędy}
\begin{center}
  \begin{tabular}{|c|c|c|c|c|}
    \hline
    & \multicolumn{2}{c|}{} & & \\
    Strona & \multicolumn{2}{c|}{Wiersz} & Jest
                              & Powinno być \\ \cline{2-3}
    & Od góry & Od dołu & & \\
    \hline
    28  & 18 & & kąt dowolny & dowolny kąt \\
    66  & 20 & & dwu & dwóch \\
    73  & &  2 & w~skończoności & skończonego \\
    % & & & & \\
    % & & & & \\
    % & & & & \\
    % & & & & \\
    % & & & & \\
    % & & & & \\
    % & & & & \\
    \hline
  \end{tabular}

  % \begin{tabular}{|c|c|c|c|c|}
  %   \hline
  %   & \multicolumn{2}{c|}{} & & \\
  %   Strona & \multicolumn{2}{c|}{Wiersz} & Jest
  %   & Powinno być \\ \cline{2-3}
  %   & Od góry & Od dołu & & \\
  %   \hline
  %   %   & & & & \\
  %   %   & & & & \\
  %   %   & & & & \\
  %   %   & & & & \\
  %   %   & & & & \\
  %   \hline
  % \end{tabular}
\end{center}
% \noi \\
% \StrWg{}{} \\
% \Jest \Powin
% \StrWg{}{} \\
% \Jest \Powin
% \StrWg{}{} \\
% \Jest \Powin
% \StrWg{}{} \\
% \Jest \Powin
% \StrWg{}{} \\
% \Jest \Powin
% \StrWg{}{} \\
% \Jest \Powin
% \StrWg{}{} \\
% \Jest \Powin

\vspace{\spaceTwo}











% ####################################################################
% ####################################################################
% Bibliografia
\bibliographystyle{plalpha} \bibliography{LibDEUSPhil,LibMathInfo}{}



% ############################

% Koniec dokumentu
\end{document}
