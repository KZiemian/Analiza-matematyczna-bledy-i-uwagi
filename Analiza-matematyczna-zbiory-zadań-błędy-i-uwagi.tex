% Autor: Kamil Ziemian

% ---------------------------------------------------------------------
% Podstawowe ustawienia i pakiety
% ---------------------------------------------------------------------
\RequirePackage[l2tabu, orthodox]{nag}  % Wykrywa przestarzałe i niewłaściwe
% sposoby używania LaTeXa. Więcej jest w l2tabu English version.
\documentclass[a4paper,11pt]{article}
% {rozmiar papieru, rozmiar fontu}[klasa dokumentu]
\usepackage[MeX]{polski}  % Polonizacja LaTeXa, bez niej będzie pracował
% w języku angielskim.
\usepackage[utf8]{inputenc} % Włączenie kodowania UTF-8, co daje dostęp
% do polskich znaków.
\usepackage{lmodern}  % Wprowadza fonty Latin Modern.
\usepackage[T1]{fontenc}  % Potrzebne do używania fontów Latin Modern.



% ------------------------------
% Podstawowe pakiety (niezwiązane z ustawieniami języka)
% ------------------------------
\usepackage{microtype}  % Twierdzi, że poprawi rozmiar odstępów w tekście.
\usepackage{graphicx}  % Wprowadza bardzo potrzebne komendy do wstawiania
% grafiki.
% \usepackage{verbatim}  % Poprawia otoczenie VERBATIME.
% \usepackage{textcomp}  % Dodaje takie symbole jak stopnie Celsiusa,
% wprowadzane bezpośrednio w tekście.
\usepackage{vmargin}  % Pozwala na prostą kontrolę rozmiaru marginesów,
% za pomocą komend poniżej. Rozmiar odstępów jest mierzony w calach.
% ------------------------------
% MARGINS
% ------------------------------
\setmarginsrb
{ 0.7in} % left margin
{ 0.6in} % top margin
{ 0.7in} % right margin
{ 0.8in} % bottom margin
{  20pt} % head height
{0.25in} % head sep
{   9pt} % foot height
{ 0.3in} % foot sep



% ------------------------------
% Często przydatne pakiety
% ------------------------------
% \usepackage{csquotes}  % Pozwala w prosty sposób wstawiać cytaty do tekstu.
% \usepackage{xcolor}  % Pozwala używać kolorowych czcionek (zapewne dużo
% więcej, ale ja nie potrafię nic o tym powiedzieć).



% ------------------------------
% Pakiety do tekstów z nauk przyrodniczych
% ------------------------------
\let\lll\undefined  % Amsmath gryzie się z pakietami do języka
% polskiego, bo oba definiują komendę \lll. Aby rozwiązać ten problem
% oddefiniowuję tę komendę, ale może tym samym pozbywam się dużego Ł.
\usepackage[intlimits]{amsmath}  % Podstawowe wsparcie od American
% Mathematical Society (w skrócie AMS)
\usepackage{amsfonts, amssymb, amscd, amsthm}  % Dalsze wsparcie od AMS
\usepackage{bm}  % Daję komendę \bm do pogrubionej czcionki matematycznej
% Gryzie się z innymi czcionkami
% \usepackage{siunitx}  % Do prostszego pisania jednostek fizycznych
\usepackage{upgreek}  % Ładniejsze greckie litery
% Przykładowa składnia: pi = \uppi
% \usepackage{slashed}  % Pozwala w prosty sposób pisać slash Feynmana.
\usepackage{calrsfs}  % Zmienia czcionkę kaligraficzną w \mathcal
% na ładniejszą. Może w innych miejscach robi to samo, ale o tym nic
% nie wiem.
\usepackage{tikz}  % Potężny pakiet PGF/TikZ.
\usetikzlibrary{decorations.markings}  % Włączenie konkretnych bibliotek
% pakietu TikZ



% ---------------
% Tworzenie otoczeń "Twierdzenie", "Definicja", "Lemat", etc.
% ---------------
\newtheorem{theorem}{Twierdzenie}  % Komenda wprowadzająca otoczenie
% „theorem” do pisania twierdzeń matematycznych
\newtheorem{definition}{Definicja}  % Analogicznie jak powyżej
\newtheorem{corollary}{Wniosek}



% ------------------------------
% Pakiety których pliki *.sty mają być w tym samym katalogu co ten plik
% ------------------------------
\usepackage{latexgeneralcommands}
\usepackage{mathcommands}
% \usepackage{analizamatematyczna}  % Pakiet napisany między innymi
% % dla tego pliku.





% ---------------------------------------------------------------------
% Dodatkowe ustawienia dla języka polskiego
% ---------------------------------------------------------------------
\renewcommand{\thesection}{\arabic{section}.}
% Kropki po numerach rozdziału (polski zwyczaj topograficzny)
\renewcommand{\thesubsection}{\thesection\arabic{subsection}}
% Brak kropki po numerach podrozdziału



% ------------------------------
% Ustawienia różnych parametrów tekstu
% ------------------------------
\renewcommand{\arraystretch}{1.2}  % Ustawienie szerokości odstępów między
% wierszami w tabelach.



% ------------------------------
% Pakiet „hyperref”
% Polecano by umieszczać go na końcu preambuły
% ------------------------------
\usepackage{hyperref}  % Pozwala tworzyć hiperlinki i zamienia odwołania
% do bibliografii na hiperlinki










% ---------------------------------------------------------------------
% Tytuł, autor, data
\title{Zbiory zadań z~analizy matematycznej --~błędy i~uwagi}

% \author{}
% \date{}
% ---------------------------------------------------------------------










% ####################################################################
\begin{document}
% ####################################################################





% ######################################
\maketitle  % Tytuł całego tekstu
% ######################################





% ######################################
\newpage
\section{Zbiory zadań z~analizy matematycznej}

\vspace{\spaceTwo}
% ######################################



% ############################
\Work{
  Włodzimierz Krysicki, L. Włodarski \\
  „Analiza matematyczna w zadaniach. Tom~I”,
  \cite{KrysickiWlodarskiAnalizaWZadaniachVolI2005}}


% ##################
\CenterBoldFont{Uwagi}


\start \Str{89} Problemy takie jak policzenie granicy
$\lim\limits_{ x \to 0 } \frac{ \sin 3 x }{ x }$, sugerują następujące
postępowanie. Wiemy, że~$\lim\limits_{ x \to 0 } 3x = 0$ oraz
że~$\lim\limits_{ x \to 0 } \frac{ \sin x }{ x } = 1$, chcielibyśmy
przekształcić wyrażenie do postaci $3 \frac{ \sin 3x }{ 3x }$
i~skorzystać z~tego, że~$\lim\limits_{ x \to 0 } \frac{ \sin 3x }{ 3x } = 0$,
pytanie jedna czy ta ostania równość zachodzi? Pozytywną odpowiedź
na~to~pytanie daje poniższe twierdzenie.

Funkcje tu użyte wciąż nie są zdefiniowane

\vspace{\spaceFour}



\start \Str{96} Można
zauważyć, że~funkcje $\artanh$ i~$\arcoth$ mają taką samą pochodną,
więc powinny różnią~się o~stałą, co~jest nonsensem. Wyjaśnieniem
problemu jest fakt, że~te dwie funkcje~są określone na rozłącznych
dziedzinach.

\vspace{\spaceFour}


% \start \Str{125} Warto przedyskutować, dlaczego tak ważnej jest
% założenie o~tym, że~$\dd{}{ x }{ t } \neq 0$. Jeżeli mamy dane dwie
% funkcje $y( t )$ i~$x( t )$, to~nie musi istnieć funkcja $y( x )$.
% Jeżeli jednak dla jakiegoś $t_{ 0 }$ mamy
% $\dd{}{ x }{ t }( t_{ 0 } ) \neq 0$ to~na mocy twierdzenie
% o~odwracaniu funkcji klasy $\Cj$ istnieje\footnote{Nie wiem czy
%   założenie o~klasie różniczkowalności jest konieczne, bowiem
%   w~przypadku funkcji rzeczywistej jednej zmiennej istnieje wiele
%   wariantów twierdzenia o~funkcji uwikłanej i~funkcji odwrotnej, więc
%   może~się okazać, iż~jeden z~nich pozwala osłabić ten warunek.
%   Szeroką dyskusję tych twierdzeń można znaleźć w~książce Fichtenholza
%   \cite{FichtenholzRachunekRiCTomI2005}.} funkcja $t( x )$ w~pewnym
% otoczeniu $t_{ 0 }$ i~w~tym otoczeniu $y( x ) = y( t( x ) )$.

% Co~jednak dzieje~się, gdy~$\dd{}{ x }{ t }( t_{ 0 } ) = 0$, czy
% funkcja $y( x )$ wówczas nie istnieje? Następujący przypadek pokazuje,
% że~tak nie musi być. Rozpatrzmy funkcje $x( t ) = t^{ 3 }$,
% $y( t ) = t$. Pomimo, że~$\dd{}{ x }{ t }( 0 ) = 0$~to, można to
% zauważyć np. rysując wykresy obu funkcji, można odwikłać $t( x )$
% i~wówczas
% \begin{equation*}
%   y( x ) =
%   \begin{cases}
%     \sqrt[1 / 3]{ x } & x \geq 0, \\
%     -\sqrt[1 / 3]{ x } & x < 0.
%   \end{cases}
% \end{equation*}
% Funkcja ta jednak nie jest różniczkowalna dla $x = 0$. Nie potrafię
% stwierdzić, czy znikanie pochodnej $\dd{}{ x }{ t }( t_{ 0 } )$ musi
% pociągać za sobą, że~jeśli funkcja $y( x )$ istnieje to jest
% nieróżniczkowalna po $x$ w~punkcie $x( t_{ 0 } )$. Wątpię jednak,
% aby~tak było.

% \vspace{\spaceFour}


% \start \Str{130} Autorzy popełnili tu błąd przyjmując, że~moduł
% pochodnej równy jest pochodnej modułu:
% \begin{equation*}
%   \left| \dd{}{ f }{ t } \right| = \dd{}{ \abso{ f } }{ t }.
% \end{equation*}
% Że~jest inaczej można~się przekonać rozważając znany przykłady ruchu
% jednostajnego po okręgu, gdzie prędkość ma stałą długość, a~jednak
% moduł przyśpieszenia nie jest równy 0, lecz $\fr{ v^{ 2 } }{ \rho }$,
% gdzie $\rho$ to promień krzywizny.

% Przeprowadzając proste obliczenia dostajemy poprawne wzory na składowe
% i~moduł przyśpieszenia:
% \begin{align*}
%   a_{ x } &= 50 ( \cos 5t^{ 2 } - 100 t^{ 2 } \sin 5t^{ 2 } ), \\
%   a_{ y } &= -50 ( \sin 5t^{ 2 } + 100 t^{ 2 } \cos 5t^{ 2 } ), \\
%   a &= 50 \sqrt{ 1 + 100 t^{ 4 } }.
% \end{align*}
% Widzimy więc, że~moduł rośnie w~czasie. Należało~się tego spodziewać,
% bowiem przyśpieszenie normalne do toru dane jest wzorem
% $\fr{ v^{ 2 } }{ \rho }$, więc jeśli prędkość rośnie liniowo, to ta
% składowa przyśpieszenia również musi rosnąć.

% % Uwagi:
% % \begin{itemize}
% % \item
% % \item
% % \item
% % \item
% % \end{itemize}

% \CenterTB{Błędy}
% \begin{center}
%   \begin{tabular}{|c|c|c|c|c|}
%     \hline
%     & \multicolumn{2}{c|}{} & & \\
%     Strona & \multicolumn{2}{c|}{Wiersz} & Jest
%                               & Powinno być \\ \cline{2-3}
%     & Od góry & Od dołu & & \\
%     \hline
%     46  &  8 & & $\sqrt[n]{ u_{ n } } < p$ & $\sqrt[n]{ u_{ n } } \leq p$ \\
%     79  & &  8 & $-b \, a$ & $-b / a$ \\
%     80  & 10 & & $= 2$ & $x = 2$ \\
%     88  & &  2 & $\fr{ x^{ 2 } - 1 }{ x - 2 }$ & $\fr{ x^{ 2 } - 4 }
%                                                  { x - 2 }$ \\
%     89  &  5 & & $\fr{ ( x - 3 )( -1 )^{ [ x ] } }{ x^{ 2 } - 9 }$
%            & $\fr{ ( x + 3 )( -1 )^{ [ x ] } }{ x^{ 2 } - 9 }$ \\
%     96  & &  3 & $\fr{ -1 }{ 1 - x^{ 2 } }$
%            & $\fr{ 1 }{ 1 - x^{ 2 } }$ \\
%     116 &  2 & & $\left[ 2 \tan \fr{ x }{ 3 } + 1 \right)$
%            & $ \left( 2 \tan \fr{ x }{ 3 } + 1 \right)$ \\
%     129 & &  5 & $a = 796$ & $a = 7.96$ \\
%     231 & 11 & & $\bigg| \Sum_{ k = 0 }^{ n } u( x ) - S( x ) < \bigg|
%                  \veps$
%            & $\bigg| \Sum_{ k = 0 }^{ n } u( x )
%              - S( x ) \bigg| < \veps$ \\
%     234 & & 11 & $\sin{ 1 \atop n }$ & $\sin \fr{ 1 }{ n }$ \\
%     241 & &  6 & $f( 0 )$ & $f'( 0 )$ \\
%     241 & &  4 & $2^{ 3 }$ & $2^{ 2 }$ \\
%     256 & &  4 & $\fr{ f'( x ) }{ {}'( x ) }$ & $\fr{ f'( x ) }
%                                                 { g'( x ) }$ \\
%     258 & &  4 & ${ 1 \atop t^{ 2 } }$ & $\fr{ 1 }{ t^{ 2 } }$ \\
%     259 &  8 & & oraz \emph{istnieje} & \emph{ale istnieje} \\
%     441 & &  7 & $\arctan^{ 3 }x$ & $\arctan x^{ 3 }$ \\
%     442 & &  6 & $\fr{ 1 }{ ( 1 - x ) \sqrt{ x } }$
%            & $\fr{ 1 }{ x \ln x \ln( \ln x ) }$ \\
%     436 & &  6 & $-4$ & $-2$ \\
%     % & & & & \\
%     % & & & & \\
%     % & & & & \\
%     \hline
%   \end{tabular}
% \end{center}

% \begin{itemize}

% \item[--] Str. 325. 16.32.
%   $\int \frac{ 6 x - 13 }{ x^{ 2 } - \frac{ 7 }{ 2 } x + \frac{ 3 }{ 2
%     } } dx$.

% \item[--] Str. 378. 19.15.
%   $\int \limits_{ 0 }^{ a } 3x \sqrt{ x^{ 2 } + 4 a^{ 2 } } dx, a > 0
%   \textrm{.}$

% \item[--] Str. 436. 5.38. $\frac{ 2 }{ 3 }$.

% \item[--] Str. 438. 6.50.
%   $y' = 7 x^{ 4 / 3 } - 13 x^{ 9 / 4 } - \frac{ 2 }{ 7 } x^{ -3 / 2 }
%   \textrm{.}$

% \item[--] Str. 438. 6.53.
%   $y' = x^{ -2 / 3 } - 3 x^{ 2 } + \frac{ 1 }{ 2 } \frac{ 1 }{ \sqrt[
%     4 ]{ x } }$.

% \item[--] Str. 438. 6.56. \ldots
%   $y' = \frac{ -5 }{ 7 \sqrt[ 7 ]{ x^{ 8 } } } - 14 x^{ 6 } - \frac{ 3
%   }{ 4 \sqrt{ x^{ 3 } } }$.

% \item[--] Str. 438. 6.89. \ldots
%   $z' = \frac{ -2 a x }{ ( a^{ 2 } + x^{ 2 } ) \sqrt{ a^{ 4 } - x^{ 4
%       } } }$.

% \item[--] Str. 440. 6.113. $\cos x \neq 0$,
%   $y' = \frac{ 7 \sin^{ 3 } x }{ \cos^{ 8 } x }$.

% \item[--] Str. 441. 6.129. $x > 1$,
%   $y' = \frac{ x \ln x }{ \sqrt{ ( x^{ 2 } - 1 )^{ 3 } } }$.

% \item[--] Str. 441. 6.131.
%   $y' = x^{ 4 } \arctan x + \frac{ x^{ 5 } - x }{ 5 ( 1 + x^{ 2 } ) }
%   - \frac{ 1 }{ 5 } x^{ 3 } + \frac{ 1 }{ 5 } x$.
% \item[--] Str. 478. \ldots
%   $I = \frac{ 1 }{ 3 } \ln | a^{ 3 } + x^{ 3 } |$.

% \item[--] Str. 480. 16.26. $I = \frac{ 1 }{ 8 } ( 2 x + 1 )^{ 4 }$.

% \item[--] Str. 480. 16.27. $x \neq \frac{ 2 }{ 3 }$;
%   $I = \frac{ -1 }{ 9 ( 3 x - 2 )^{ 3 } }$.

% \item[--] Str. 480. 16.37. $x \neq \frac{ 2 }{ 3 }$,
%   $x \neq \frac{ 3 }{ 2 }$;
%   $I = \frac{ 1 }{ 5 } \ln | \frac{ 2 x - 3 }{ 3 x - 2 } |$.

% \end{itemize}

\vspace{\spaceTwo}
% ############################










% ######################################
\newpage
\section{Zbiory zadań z~analizy zespolonej}

\vspace{\spaceTwo}
% ######################################



% ####################
\Work{ % Autor i tytuł dzieła
  Jak Krzyż \\
  „Zbiór zadań z~funkcji analitycznych”,
  \cite{KrzyzZbiorZadanZFunkcjiAnalitycznych2005} }


% ##################
\CenterBoldFont{Uwagi}


\start \StrWg{28}{18} Sens tego niezbyt zgrabnie zbudowanego zdanie
jest prosty. Mając dany dowolny niezerowy kąt\footnote{Czy to
  założenie jest potrzebne?} o~wierzchołku $z = 0$, możemy dobrać
takie $\alpha$, że~funkcja $z^{ \alpha }$ przekształca ten kąt na~kąt półpełny.

\vspace{\spaceFour}



\start \StrWg{31}{14} Stwierdzeni, że~całka
\begin{equation}
  \label{eq:KrzyzZZzFA-01}
  \int_{ \gamma } f( z ) \, dz
\end{equation}
niezależny od parametryzacji, choć zrozumiałe jest niezręczne. Już
następna część zdania przypomina, że~parametryzacji zmieniające
orientację zmienia jej znak na przeciwny. Należałoby więc napisać,
iż~całka ta nie zależy od~wyboru parametryzacji zachowującej
orientację.

\vspace{\spaceFour}


% \start

% \vspace{\spaceFour}


% \start

% \vspace{\spaceFour}


% \start

% \vspace{\spaceFour}


% \start

% \vspace{\spaceFour}


% \start

% \vspace{\spaceFour}


% \start

% \vspace{\spaceFour}


% ##################
\CenterBoldFont{Błędy}


\begin{center}

  \begin{tabular}{|c|c|c|c|c|}
    \hline
    & \multicolumn{2}{c|}{} & & \\
    Strona & \multicolumn{2}{c|}{Wiersz} & Jest
                              & Powinno być \\ \cline{2-3}
    & Od góry & Od dołu & & \\
    \hline
    28  & 18 & & kąt dowolny & dowolny kąt \\
    66  & 20 & & dwu & dwóch \\
    73  & &  2 & w~skończoności & skończonego \\
    % & & & & \\
    % & & & & \\
    % & & & & \\
    % & & & & \\
    % & & & & \\
    % & & & & \\
    % & & & & \\
    \hline
  \end{tabular}





  % \begin{tabular}{|c|c|c|c|c|}
  %   \hline
  %   & \multicolumn{2}{c|}{} & & \\
  %   Strona & \multicolumn{2}{c|}{Wiersz} & Jest
  %   & Powinno być \\ \cline{2-3}
  %   & Od góry & Od dołu & & \\
  %   \hline
  %   %   & & & & \\
  %   %   & & & & \\
  %   %   & & & & \\
  %   %   & & & & \\
  %   %   & & & & \\
  %   \hline
  % \end{tabular}

\end{center}

% \noi \\
% \StrWg{}{} \\
% \Jest \Powin
% \StrWg{}{} \\
% \Jest \Powin
% \StrWg{}{} \\
% \Jest \Powin
% \StrWg{}{} \\
% \Jest \Powin
% \StrWg{}{} \\
% \Jest \Powin
% \StrWg{}{} \\
% \Jest \Powin
% \StrWg{}{} \\
% \Jest \Powin

\vspace{\spaceTwo}
% ############################










% ####################################################################
% ####################################################################
% Bibliografia
\bibliographystyle{plalpha}

\bibliography{MathComScienceBooks}{}





% ############################

% Koniec dokumentu
\end{document}
