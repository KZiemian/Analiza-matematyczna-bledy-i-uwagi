% ------------------------------------------------------------------------------------------------------------------
% Basic configuration and packages
% ------------------------------------------------------------------------------------------------------------------
% Package for discovering wrong and outdated usage of LaTeX.
% More information to be found in l2tabu English version.
\RequirePackage[l2tabu, orthodox]{nag}
% Class of LaTeX document: {size of paper, size of font}[document class]
\documentclass[a4paper,11pt]{article}



% ------------------------------------------------------
% Packages not tied to particular normal language
% ------------------------------------------------------
% This package should improved spaces in the text
\usepackage{microtype}
% Add few important symbols, like text Celcius degree
\usepackage{textcomp}



% ------------------------------------------------------
% Polonization of LaTeX document
% ------------------------------------------------------
% Basic polonization of the text
\usepackage[MeX]{polski}
% Switching on UTF-8 encoding
\usepackage[utf8]{inputenc}
% Adding font Latin Modern
\usepackage{lmodern}
% Package is need for fonts Latin Modern
\usepackage[T1]{fontenc}



% ------------------------------------------------------
% Setting margins
% ------------------------------------------------------
\usepackage[a4paper, total={14cm, 25cm}]{geometry}



% ------------------------------------------------------
% Setting vertical spaces in the text
% ------------------------------------------------------
% Setting space between lines
\renewcommand{\baselinestretch}{1.1}

% Setting space between lines in tables
\renewcommand{\arraystretch}{1.4}



% ------------------------------------------------------
% Packages for scientific papers
% ------------------------------------------------------
% Switching off \lll symbol, that I guess is representing letter "Ł"
% It collide with `amsmath' package's command with the same name
\let\lll\undefined
% Basic package from American Mathematical Society (AMS)
\usepackage[intlimits]{amsmath}
% Equations are numbered separately in every section
\numberwithin{equation}{section}

% Other very useful packages from AMS
\usepackage{amsfonts}
\usepackage{amssymb}
\usepackage{amscd}
\usepackage{amsthm}

% Package with better looking calligraphy fonts
\usepackage{calrsfs}

% Package with better looking greek letters
% Example of use: pi -> \uppi
\usepackage{upgreek}
% Improving look of lambda letter
\let\oldlambda\Lambda
\renewcommand{\lambda}{\uplambda}




% ------------------------------------------------------
% BibLaTeX
% ------------------------------------------------------
% Package biblatex, with biber as its backend, allow us to handle
% bibliography entries that use Unicode symbols outside ASCII
\usepackage[
language=polish,
backend=biber,
style=alphabetic,
url=false,
eprint=true,
]{biblatex}

\addbibresource{Analiza-zespolona-Bibliography.bib}





% ------------------------------------------------------
% Defining new environments (?)
% ------------------------------------------------------
% Defining enviroment "Wniosek"
\newtheorem{corollary}{Wniosek}
\newtheorem{definition}{Definicja}
\newtheorem{theorem}{Twierdzenie}





% ------------------------------------------------------
% Local packages
% You need to put them in the same directory as .tex file
% ------------------------------------------------------
% Package containing various command useful for working with a text
\usepackage{./Local-packages/general-commands}
% Package containing commands and other code useful for working with
% mathematical text
\usepackage{./Local-packages/math-commands}





% ------------------------------------------------------
% Package "hyperref"
% They advised to put it on the end of preambule
% ------------------------------------------------------
% It allows you to use hyperlinks in the text
\usepackage{hyperref}










% ------------------------------------------------------------------------------------------------------------------
% Title and author of the text
\title{Rachunek wariacyjny \\
  {\Large Błędy i~uwagi}}

\author{Kamil Ziemian}


% \date{}
% ------------------------------------------------------------------------------------------------------------------










% ####################################################################
\begin{document}
% ####################################################################





% ######################################
\maketitle  % Tytuł całego tekstu
% ######################################





% % ######################################
% \section{XIX szkoła analizy matematycznej}

% \vspace{\spaceTwo}
% % ######################################





% ############################
\section{I.M.~Gelfand, S.W.~Fomin \textit{Rachunek
    wariacyjny},
  \cite{GelfandFominRachunekWariacyjny1972}}

\vspace{0em}


% ##################
\CenterBoldFont{Uwagi do konkretnych stron}

\vspace{0em}


\noindent
\Str{10} W~tym miejscu pierwszy raz autorzy książki używają pewnego
skrótu myślowego. Zapisują oni bowiem funkcjonał długości
krzywej\footnote{Dokładniej, chodzi o~klasę krzywych, które da się zapisać za
  pomocą zależności typu $x \mapsto ( x, y( x ) )$, gdzie $y( x )$ jest normalną
  funkcją zmiennej~$x$. Krzywą, która nie jest tego typu, jest choćby okrąg
  o~środku w~punkcie $( 0, 0 )$ i~promieniu jeden. Dla punktu tego okręgu
  o~współrzędnych $( x, y )$ spełnione jest równanie
  $x^{ 2 } + y^{ 2 } = 1$. Przy czym nie zakładamy, że~$y$ jest funkcją $x$,
  ani odwrotnie.} pewnej klasy krzywych na płaszczyźnie jako
\begin{equation}
  \label{eq:Gelfand-Fomin-Year1972-01}
  \int_{ a }^{ b } \sqrt{ 1 + y'^{ \, 2 } } \,\: dx,
\end{equation}
pomijając jawną zależność funkcji $y'( x )$ od zmiennej niezależnej~$x$.
Bardziej precyzyjny zapis tego funkcjonału miałby postać
\begin{equation}
  \label{eq:Gelfand-Fomin-Year1972-02}
  \int_{ a }^{ b } \sqrt{ 1 + \big( y'( x ) \big)^{ 2 } } \,\: dx.
\end{equation}
Innym przykładem tego skrótu myślowego jest pojawiający~się też na tej
stronie zapis innego funkcjonału
\begin{equation}
  \label{eq:Gelfand-Fomin-Year1972-03}
  \int_{ a }^{ b } F\left( x, y, y' \right) dx,
\end{equation}
który w~bardziej precyzyjnej notacji zapisalibyśmy jako
\begin{equation}
  \label{eq:Gelfand-Fomin-Year1972-04}
  \int_{ a }^{ b } F\left( x, y( x ), y'( x ) \right) dx.
\end{equation}
Dla większej przejrzystości prowadzonych rozważań, w~dalszym ciągu tych
notatek będziemy~się starali możliwie jawnie zapisywać zależność wszystkich
funkcji od~ich argumentów w~sytuacjach, gdy wykonujemy operacje takie
jak całkowanie po~tych zmiennych.

\VerSpaceFour





\noindent
\Str{15} W~definicji liniowego funkcjonału nad liniową przestrzenią
unormowaną~$R$ nie pojawia się warunek, by był on jednorodny, czyli
\begin{equation}
  \label{eq:Gelfand-Fomin-Year1972-05}
  \varphi( \alpha \, h ) = \alpha \, \varphi( h ), \quad \alpha \in \Cbb, \, h \in R.
\end{equation}
Nie wiem, czy jest to przeoczenie ze~strony autorów, czy też addytywność
i~ciągłość takiego funkcjonału od razu gwarantuje jego jednorodność.

\VerSpaceFour





\noindent
\Str{15} W~tym miejscu autorzy wprowadzają jedno z~kluczowych dla
rachunku wariacyjnego zagadnień. Mianowicie przyjmijmy, iż~dana jest pewna
przestrzeń funkcji~$H$. Jeśli dla funkcji $f( x )$ równość
\begin{equation}
  \label{eq:Gelfand-Fomin-Year1972-06}
  \int_{ x_{ 1 } }^{ x_{ 2 } } f( x ) h( x ) \, dx = 0
\end{equation}
zachodzi dla każdego $h( x ) \in H$, to co możemy o~niej na podstawie tego
powiedzieć?

????
Z braku czasu mogę tylko zaznaczyć kilka problemów jakie mogą się w~związku
z~tym pojawić. Jaka całka Riemanna/Lebesgue’a/jakaś inna występuje w
powyższym wzorze? Co musimy założyć o funkcji $f( x )$ by odpowiednie całki
istniały? Dla jakich przestrzeni $H$ chcielibyśmy udowodnić twierdzenia,
która dowodzą, np. że funkcja $f( x )$ jest tożsamościowo równa 0?

????
Te i inne pytanie musimy na razie odłożyć na bok, licząc iż przyjdą kiedyś
lepsze czasy, gdy będziemy się mogli nimi zająć.

\VerSpaceFour





\noindent
\Str{17} W~tym miejscu potrzebny nam jest rozkład funkcji ciągłej
$f : [ a, b ] \to \Rbb$ (lub $f : [ a, b ] \to \Cbb$), postaci
\begin{equation}
  \label{eq:Gelfand-Fomin-Year1972-07}
  f( x ) = \lambda( x ) + \alpha,
\end{equation}
gdzie $\alpha$ jest stałą, zaś $\lambda( x )$ jest funkcją posiadającą własność
\begin{equation}
  \label{eq:Gelfand-Fomin-Year1972-08}
  \int_{ a }^{ b } \lambda( x ) \, dx = 0.
\end{equation}
Taki rozkład da się bardzo łatwo podać. Weźmy mianowicie
\begin{subequations}
  \begin{align}
    \label{eq:Gelfand-Fomin-Year1972-09-A}
    &\alpha =
      \frac{ 1 }{ \absOne{ b - a } } \,
      \int_{ a }^{ b } f( x ) \, dx, \\[0.3em]
    \label{eq:Gelfand-Fomin-Year1972-09-B}
    &\lambda( x ) = f( x ) - \alpha.
  \end{align}
\end{subequations}
Widzimy więc, że~wprowadzenie funkcji $\lambda( x )$ sprowadza~się do sprytnej
zmiany oznaczeń. Co nie zmienia faktu, że~jest to bardzo użyteczny trik.

\VerSpaceFour





% % ##################
% \newpage

% \CenterBoldFont{Błędy}

% \vspace{\spaceFive}


% \begin{center}

%   \begin{tabular}{|c|c|c|c|c|}
%     \hline
%     Strona & \multicolumn{2}{c|}{Wiersz} & Jest
%                               & Powinno być \\ \cline{2-3}
%     & Od góry & Od dołu & & \\
%     \hline
%     % & & & & \\
%     % & & & & \\
%     % & & & & \\
%     % & & & & \\
%     % & & & & \\
%     % & & & & \\
%     % & & & & \\
%     \hline
%   \end{tabular}

% \end{center}

\VerSpaceTwo


% ############################










% ############################
\newpage

\section{I.M.~Gelfand, S.W.~Fomin \textit{Rachunek
    wariacyjny},
  \cite{GelfandFominRachunekWariacyjny1979}}

\vspace{0em}


% ##################
\CenterBoldFont{Błędy}

\VerSpaceFive


\begin{center}

  \begin{tabular}{|c|c|c|c|c|}
    \hline
    Strona & \multicolumn{2}{c|}{Wiersz} & Jest
                              & Powinno być \\ \cline{2-3}
    & Od góry & Od dołu & & \\
    \hline
    25  & & \hphantom{0}1 & $\lim\limits_{ \Delta x \to 0 }
                 \frac{ \Delta y' }{ \Delta x } = \tilde{ F }_{ y' y' }$
           & $\lim\limits_{ \Delta x \to 0 }
             \frac{ \Delta y' }{ \Delta x } \tilde{ F }_{ y' y' }$ \\
    27  & \hphantom{0}7 & & $\frac{ d }{ dx }( F - y' F_{ y' } )$
           & $\frac{ d }{ dx }( F - y' F_{ y' } ) = 0$ \\
           % & & & & \\
    \hline
  \end{tabular}

\end{center}

\VerSpaceTwo



% ############################







% % ##################
% \CenterBoldFont{Błędy}


% \begin{center}

%   \begin{tabular}{|c|c|c|c|c|}
%     \hline
%     & \multicolumn{2}{c|}{} & & \\
%     Strona & \multicolumn{2}{c|}{Wiersz} & Jest
%                               & Powinno być \\ \cline{2-3}
%     & Od góry & Od dołu & & \\
%     \hline
%     % & & & & \\
%     % & & & & \\
%     \hline
%   \end{tabular}





%   % \begin{tabular}{|c|c|c|c|c|}
%   %   \hline
%   %   & \multicolumn{2}{c|}{} & & \\
%   %   Strona & \multicolumn{2}{c|}{Wiersz} & Jest
%   %                             & Powinno być \\ \cline{2-3}
%   %   & Od góry & Od dołu & & \\
%   %   \hline
%   %   & & & & \\
%   %   \hline
%   % \end{tabular}

% \end{center}


% \noindent
% \StrWd{}{} \\
% \Jest   \\
% \Powin  \\


% \vspace{\spaceTwo}
% % ############################










% ############################
\newpage

\section{ % Autorzy i tytuł dzieła
  J. Jost, X. Li-Jost
  \textit{Calculus of Variations},
  \cite{JostLiJostCalculusOfVariations1998}}

\vspace{0em}

% ##################
\CenterBoldFont{Uwagi}

\vspace{0em}


\noindent
\Str{6} Dowód podstawowego lematu rachunku wariacyjnego jest
poprawny, acz mało elegancko zrobiony.


% % ##################
% \CenterBoldFont{Błędy}


% \begin{center}

%   \begin{tabular}{|c|c|c|c|c|}
%     \hline
%     & \multicolumn{2}{c|}{} & & \\
%     Strona & \multicolumn{2}{c|}{Wiersz} & Jest
%                               & Powinno być \\ \cline{2-3}
%     & Od góry & Od dołu & & \\
%     \hline
%     & & & & \\
%     & & & & \\
%     \hline
%   \end{tabular}

% \end{center}

\VerSpaceTwo


% ############################










% ####################################################################
% ####################################################################
% Bibliography

\printbibliography





% ############################
% End of the document

\end{document}
